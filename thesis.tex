%% 
%% This is a sample doctoral dissertation.  It shows the appropriate
%% structure for your dissertation.  It should handle most of the
%% strange requirements imposed by the Grad school; like the different
%% handling of titles of one/many appendices.  It will automatically
%% handle the linespacing changes.  The body default is double-spaced
%% (except when you use the singlespace or condensed options).  The
%% default for quotations is single-space, and the default for tabular
%% environments is also single-space.  
%%
%% This class adds the following commands and environments to the
%% report class, upon which it is based:
%% Commands
%% ------------
%% \degree{name}{abbrv} -- Sets the name and abbreviation for the degree.
%%                         These default to ``Doctor of Philosopy''
%%                         and ``Ph.D.'', respectively.
%% \copyrightyear{year} -- for the copyright page.
%% \bachelors{degree}{institution} -- for the abstract
%% \masters{degree}{institution}   --  "
%%     if you have other degrees you may use
%% \secondbachelors{degree}{institution}
%% \thirdbachelors{degree}{institution}
%% \secondmasters{degree}{institution}
%% \thirdmasters{degree}{institution}
%% \priordoctorate{degree}{institution}
%%
%% \committeechair{name}           -- for the signature page
%% or, if you have two co-chairs:
%% \cochairs{first name}{second name}
%%
%% \firstreader{name}              --  "
%% \secondreader{name}             --  "
%% \thirdreader{name}              -- (optional)
%% \fourthreader{name}             --  "
%% \fifthreader{name}              --  "
%% \sixthreader{name}              --  "
%% \departmentchair{name}          -- for the signature page
%% \departmentname{name}           --  "
%%
%% \copyrightpage                  -- produces the copyright page
%% \signaturepage                  -- produces the signature page
%%
%% \frontmatter                    -- these are required in their various
%% \mainmatter                     -- appropriate locations
%% \backmatter                     --
%%
%% \unnumberedchapter[toc]{name}   -- like \chapter, except that it
%%                                    produces an unnumbered chapter;
%%                                    alternatively, like \chapter*,
%%                                    except that it lists the chapter
%%                                    in the table of contents.
%%
%% New environments:
%%   dedication  -- for the dedication
%%   abstract    -- for the abstract
%%
%% The thesis documentclass is built on top of the report document class.
%% It accepts all of the options that the report class accepts, plus the
%% following:
%%     doublespace -- the default, indicates double spacing as per U.Mass.
%%                    requirements.  You will need this when you do your
%%                    final copy.
%%     singlespace -- for earlier work, not acceptable to the Grad school
%%     condensed   -- for earlier work, not acceptable to the Grad school,
%%                    creates condensed versions of the frontmatter. 
%%                    Condensed implies singlespace.
%%     dissertation - the default, indicates that this document is a
%%                    dissertation.
%%     proposal    -- indicates that this document is a dissertation proposal,
%%                    rather than a dissertation.  This will only change the
%%                    wording on the title and signature pages.
%%     thesis      -- indicates that this document is a Master's thesis 
%%                    rather than a doctoral dissertation.  This also changes
%%                    the default for \degree to Master of Science, M.S.
%%     allowlisthypenation -- (the default), allows hyphenation of words in
%%                    the table of contents, the list of figures, and the list
%%                    of tables.  I believe that this is acceptable to the 
%%                    Graduate School.
%%     nolisthyphenation -- disallows hyphenation of words in the table of
%%                    contents and the list of figures and tables.  Use this 
%%                    option if the Grad School doesn't like your hyphenation.
%%     nicerdraft  -- relaxes some of the Grad School's rules for working with
%%                    drafts -- has no effect when doublespace is in effect
%%     nonicerdraft -- the default, leaves things in draft as they will be in
%%                     the final version
%% umassthesis changes the default font size to 12pt, but you may specify 10pt or
%%   11pt in the options.
\documentclass[proposal,singlespace]{umassthesis}          % for Ph.D. dissertation or proposal
% \documentclass[thesis]{umassthesis}  % for Master's thesis

%%
%% If you have enough figures or tables that you run out of space for their
%% numbers in the List of Tables or List of figures, you can use the following
%% command to adjust the space left for numbers.  The default is shown:
%%
%% \setlength{\tablenumberwidth}{2.3em}

%% Use the hyperref package if you're producing a version for online
%% distribution and you want hyperlinks.  Note that the Grad School doesn't want
%% their PDF viewers to colorize or otherwise highlight the links; use the
%% hidelinks option to hyperref to avoid decorating links.
%\usepackage[hidelinks]{hyperref}

%% One way of formatting the epigraph/frontispiece is to use this package.
%\usepackage{epigraph}


\usepackage{verbatim}


%vertex connectivity preamble stuff

\usepackage{url}
%\usepackage{color,fullpage,microtype}
%\usepackage{graphicx,wrapfig,subfigure,titling}
\usepackage{amssymb,amsmath,amssymb,amsfonts}
\usepackage{amsthm}
\usepackage{textcomp}
\usepackage{multirow}
%\usepackage{gensymb}
\usepackage{array}
\usepackage{times}
\usepackage[sort]{cite}

\newtheorem{theorem}{Theorem}%[section]
\newtheorem{definition}[theorem]{Definition}

\newtheorem{proposition}[theorem]{Proposition}%[section]
\newtheorem{corollary}[theorem]{Corollary}%[section]
\newtheorem{lemma}[theorem]{Lemma}%[section]
\newtheorem{conjecture}[theorem]{Conjecture}%[section]
\newtheorem{fact}{Fact}
\newtheorem{problem}{Problem}
%\newtheorem{example}{Example}
\newtheorem{claim}{Claim}

%\newcommand{\vecx}{\mathbf{x}} 
\newcommand{\vecy}{\mathbf{y}} 

\newcommand{\expec}[1]{\mathbb E\left [ #1 \right ]}
\newcommand{\rounddown}[1]{\left \lfloor  #1 \right \rfloor}
\newcommand{\roundup}[1]{\left \lceil  #1 \right \rceil}
\DeclareMathOperator{\polylog}{polylog}
\DeclareMathOperator{\poly}{poly}
\DeclareMathOperator{\light}{{light}}
%\DeclareMathOperator{\outdeg}{out-deg}
%\newcommand{\inneigh}{\Gamma^-}
%\newcommand{\outneigh}{\Gamma^+}

\newcommand{\todo}[1]{{\color{blue}[\textit{#1}]}}
\newcommand{\reals}{{\mathbb R}}
\newcommand{\prob}[1]{\mathbb P \left [ #1 \right ]}
%\newcommand{\Egt}{U}
%\newcommand{\Eslt}{D}

\newcommand{\vecf}{\mathbf{f}} 

%\newcommand{\sfv}{\begin{proof}See full version.\end{proof}}
\newcommand{\sfv}{}
\newcommand{\R}{{\mathbb R}}
\renewcommand{\H}{{\cal H}}
\renewcommand{\O}{{\cal O}}
\newcommand{\etal}{{et al.~}}
\newcommand{\rk}{\mbox{\sc Rk}}
\newcommand{\hatyi}{\hat{y}_i}
\newcommand{\hatk}{\hat{k}} 
\newcommand{\veca}{\mathbf{a}} 
\newcommand{\veczero}{\mathbf{0}}
\newcommand{\vecone}{\mathbf{1}} 
\newcommand{\vecx}{\mathbf{x}} 
\newcommand{\delc}{{\Delta_C(i)}}
\newcommand{\alg}{{\cal A}}
\newcommand{\obj}{{\cal O}}
\newcommand{\y}{{\mathbf Y}}
\newcommand{\bin}{{\mathbf{Bin}}}
\newcommand{\eat}[1]{}
\newcommand{\acc}{E}
\renewcommand{\k}{\kappa}

\newcommand{\A}{{\cal A}}
\newcommand{\B}{{\cal B}}

\newcommand{\I}{{\cal I}}
\newcommand{\D}{{\cal D}}
\newcommand{\X}{{\cal X}}
\newcommand{\Y}{{\cal Y}}
\newcommand{\Z}{{\cal Z}}
\renewcommand{\P}{{\cal P}}
\renewcommand{\S}{{\cal S}}
\newcommand{\cc}{cc}

\newcommand{\E}{{\mathbf E}}
%\newcommand{\comment}[1]{\marginpar{!!!}[#1]}

\newcommand{\err}{{\mathcal E}}
%\newcommand{\comment}[1]{{\color{red} #1} \marginpar{!!!}}


\newcommand{\pparagraph}[1]{\vspace{0.1in}\noindent{\bf \boldmath #1}}




% mesh preamble stuff

%\usepackage{minted}
%\RequirePackage[outputdir=latex.out,cache=false]{minted}
%\RequirePackage[cache=false]{minted}
\usepackage{listings}
\usepackage{microtype}
\usepackage{hyperref}
\usepackage{xspace}
\usepackage{graphicx}
%\usepackage{url}
%\usepackage{amsmath}
%\usepackage{amssymb}
\usepackage{relsize}
\usepackage{tikz}
\usepackage{clrscode3e}
%\usepackage{amsthm}
\usepackage{color}

\newtheorem{observation}{Observation}%[section]


\newcommand{\sm}{\proc{SplitMesher}\xspace}

%% \setmonofont[Scale=MatchLowercase]{DejaVu Sans Mono}
%% \setmonofont{DejaVu Sans Mono}

\SetTracking{encoding={*}, shape=sc}{0} % No tracking for smallcaps

\newcommand{\Mesh}{\textsc{Mesh}\xspace}
\newcommand{\mesh}{Mesh\xspace}
\newcommand{\ndegree}{\textup{d}} % Command \deg already defined by some dependency
%\newcommand{\comments}[1]{[#1]}


%% Some recommended packages.
\usepackage{booktabs}   %% For formal tables:
                        %% http://ctan.org/pkg/booktabs
%\usepackage{subcaption} %% For complex figures with subfigures/subcaptions
                        %% http://ctan.org/pkg/subcaption
%\captionsetup[subfloat]{font={small,sf}}
% Settings for figures and tables. Figure captions are placed below the figure,
% while table captions are placed above the table. All captions are sans-serif.
%\RequirePackage[font={normalsize,sf,bf}]{caption}
\RequirePackage[font={footnotesize,rm}]{caption}
\RequirePackage[position=bottom]{subfig}
\captionsetup[table]{aboveskip=0.5em, belowskip=0.5em}
\captionsetup[figure]{aboveskip=0.5em, belowskip=0em}
\captionsetup[subfloat]{font={footnotesize,rm}}
\setcounter{topnumber}{2}
\setcounter{dbltopnumber}{2}
\setcounter{bottomnumber}{2}
\setcounter{totalnumber}{4}

\renewcommand{\topfraction}{0.85}
\renewcommand{\dbltopfraction}{0.9}
\renewcommand{\bottomfraction}{0.85}
\renewcommand{\textfraction}{0.07}
\renewcommand{\floatpagefraction}{0.85}
\renewcommand{\dblfloatpagefraction}{0.85}

\setlength{\floatsep}{0.5em plus 0.2em minus 0.2em}
\setlength{\dblfloatsep}{0.5em plus 0.2em minus 0.2em}
\setlength{\textfloatsep}{0.5em plus 0.2em minus 0.2em}
\setlength{\dbltextfloatsep}{0.5em plus 0.2em minus 0.2em}

% Utility packages for floats and tables.
\RequirePackage{float}
\RequirePackage{graphicx}
\RequirePackage{booktabs}
\RequirePackage{multirow}




\begin{document}

%%
%% You must fill in all of these appropriately
\title{Algorithms for Massive, Expensive, or Otherwise Inconvenient Graphs}
\author{David Tench}
\date{Jun 2020} % The date you'll actually graduate -- must be
                     % February, May, or September
\copyrightyear{2020}
\bachelors{B.Sc.}{Lehigh University}
\masters{M.Sc.}{University of Massachusetts Amherst}
%\secondmasters{M.Ed.}{Antioch College}
%\priordoctorate{M.D.}{University of Never-never-land}
\committeechair{Andrew McGregor}
%\cochairs{B. B. Bahh}{I. M. A. Wolf}
\firstreader{Phillipa Gill}
\secondreader{Cameron Musco}
\thirdreader{Markos Katsoulakis}
%\fourthreader{Mary Lamb}   % Optional
%\fifthreader{}            % Optional
%\sixthreader{}            % Optional
\departmentchair[Chair of the Faculty]{James Allan} % Default uses "Department Chair" as the title. To
% use an alternate title, such as "Chair", use \departmentchair[Chair]{Pete Shearer}
% CICS uses "Chair of the Faculty" as of 2019.
\departmentname{College of Information and Computer Sciences}

%% If your degree is something other than a Ph.D. (for a dissertation), or
%% an M.S. (for a thesis), you will need to uncomment the appropriate
%% following line:
%%
%% \degree{Doctor of Education}{Ed.D.}
\degree{Doctor of Philosophy}{Ph.D.}
%%
%% \degree{Master of Arts}{M.A.}
%% \degree{Master of Arts in Teaching}{M.A.T.}
%% \degree{Master of Business Administration}{M.B.A.}
%% \degree{Master of Education}{M.Ed.}
%% \degree{Master of Fine Arts}{M.F.A.}
%% \degree{Master of Landscape Architecture}{M.L.A.}
%% \degree{Master of Music}{M.M.}
%% \degree{Master of Public Administration}{M.P.A.}
%\degree{Master of Public Health}{M.P.H.}
%% \degree{Master of Regional Planning}{M.R.P.}
%% \degree{Master of Science}{M.S.}
%% \degree{Master of Science in Accounting}{M.S. Acctg.}
%% \degree{Master of Science in Chemical Engineering}{M.S. Ch.E.}
%% \degree{Master of Science in Civil Engineering}{M.S.C.E.}
%% \degree{Master of Science in Electrical and Computer Engineering}{M.S.E.C.E.}
%% \degree{Master of Science in Engineering Management}{M.S. Eng. Mgt.}
%% \degree{Master of Science in Environmental Engineering}{M.S. Env. E.}
%% \degree{Master of Science in Industrial Engineering and Operations Research}{M.S.I.E.O.R.}
%% \degree{Master of Science in Manufacturing Engineering}{M.S. Mfg. Eng.}
%% \degree{Master of Science in Mechanical Engineering}{M.S.M.E.}
%%
%% \degree{Professional Master of Business Administration}{P.M.B.A.}


%%
%% These lines produce the title, copyright, and signature pages.
%% They are Mandatory; except that you could leave out the copyright page
%% if you were preparing an M.S. thesis instead of a PhD dissertation.
\frontmatter
\maketitle
\copyrightpage     %% not required for an M.S. thesis
\signaturepage

%%
%% Dedication is optional -- but this is how you create it
\begin{dedication}              % Dedication page
  \begin{center}
    \emph{To my mother and father.}
  \end{center}
\end{dedication}

%%
%% Epigraph (aka frontispiece) is also optional, but this is one way you
%% can create it
%\begin{frontispiece}
%  %% Format to your liking -- see documentation of epigraph package
%  \setlength{\epigraphrule}{0pt}
%
%  \begin{epigraphs}
%    \qitem{%
%      \itshape
%      Mary had a little lamb,\\
%      Her fleece was white as snow.\\
%      \vspace{\baselineskip}
%      And everywhere that Mary went\\
%      The lamb was sure to go.
%      \vspace{\baselineskip}}
%    {Sarah Josepha Hale}
%
%    \vspace{2\baselineskip}
%    \qitem{%
%      \itshape
%      Baa, baa, black sheep,\\
%      Have you any wool?\\
%      Yes, sir, yes, sir,\\
%      Three bags full;\\
%      One for the master,\\
%      And one for the dame,\\
%      And one for the little boy\\
%      Who lives down the lane.
%      \vspace{\baselineskip}}
%    {English Nursery Rhyme}
%
%  \end{epigraphs}
%\end{frontispiece}

%%
%% Acknowledgements are optional...yeah, right.
\chapter{Acknowledgments}             % Acknowledgements page
  Thanks to all those fine shepherds. Not to mention all the great
  border collies and suchlike fine animals.

%%
%% Abstract is MANDATORY. -- Except for MS theses
\begin{abstract}                % Abstract

  A long-standing assumption common in algorithm design is that any part of the input is accessible at any time for unit cost.  However, as we work with increasingly large data sets, or as we build smaller devices, we must revisit this assumption.  In this proposal, I present of my work on graph algorithms designed for circumstances where traditional assumptions about inputs do not apply.

1. Classical graph algorithms require direct access to the input graph and this is not feasible when the graph is too large to fit in memory.  For computation on massive graphs we consider the dynamic streaming graph model.  Given an input graph defined by as a stream of edge insertions and deletions, our goal is to approximate properties of this graph using space sublinear in the size of the stream.  In this proposal, I present algorithms for approximating vertex connectivity and hypergraph edge connectivity in graph streams.

2. In certain applications the input graph is not explicitly represented, but its edges may be discovered via queries which require costly computation or measurement.  I present Mesh, a memory manager which compacts memory efficiently by finding an approximate graph matching subject to stringent time and edge query restrictions.

\end{abstract}

%%
%% Preface goes here...would be just like Acknowledgements -- optional
%% \chapter{Preface} 
%% ...


%%
%% Table of contents is mandatory, lists of tables and figures are 
%% mandatory if you have any tables or figures; must be in this order.
\tableofcontents                % Table of contents
\listoftables                   % List of Tables
\listoffigures                  % List of Figures

%%
%% We don't handle List of Abbreviations
%% We don't handle Glossary

%%%%%%%%%%%%%%%%%%%%%%%%%%%%%%%%%%%%%%%%%%%%%%%%%%%%%%%%%%%%%%%%%%%%%%%%%
%% Time for the body of the dissertation
\mainmatter   %% <-- This line is mandatory

%%
%% If you want an introduction, which is not a numbered chapter, insert
%% the following two lines.  This is OPTIONAL:
\unnumberedchapter{Introduction}
Why on earth do I want to study sheep anyway?

%%
%% Some sample text

\chapter{Vertex and Hyperedge Connectivity in Dynamic Graph Streams}
\label{chap:vertex}


\iffalse
%\documentclass[11pt]{article}  
%\documentclass[twoside,leqno,twocolumn]{article}  
%\usepackage{ltexpprt} 

%\documentclass{sig-alternate-2013}

\usepackage{url}
%\usepackage{color,fullpage,microtype}
%\usepackage{graphicx,wrapfig,subfigure,titling}
\usepackage{amssymb,amsmath,amssymb,amsfonts}
%\usepackage{amsthm}
\usepackage{textcomp}
\usepackage{multirow}
\usepackage{gensymb}
\usepackage{array}
\usepackage{times}
\usepackage[sort]{cite}

\newtheorem{theorem}{Theorem}%[section]
\newtheorem{definition}[theorem]{Definition}

\newtheorem{proposition}[theorem]{Proposition}%[section]
\newtheorem{corollary}[theorem]{Corollary}%[section]
\newtheorem{lemma}[theorem]{Lemma}%[section]
\newtheorem{conjecture}[theorem]{Conjecture}%[section]
\newtheorem{fact}{Fact}
\newtheorem{problem}{Problem}
\newtheorem{example}{Example}
\newtheorem{claim}{Claim}

%\newcommand{\vecx}{\mathbf{x}} 
\newcommand{\vecy}{\mathbf{y}} 

\newcommand{\expec}[1]{\mathbb E\left [ #1 \right ]}
\newcommand{\rounddown}[1]{\left \lfloor  #1 \right \rfloor}
\newcommand{\roundup}[1]{\left \lceil  #1 \right \rceil}
\DeclareMathOperator{\polylog}{polylog}
\DeclareMathOperator{\poly}{poly}
\DeclareMathOperator{\light}{{light}}
%\DeclareMathOperator{\outdeg}{out-deg}
%\newcommand{\inneigh}{\Gamma^-}
%\newcommand{\outneigh}{\Gamma^+}

\newcommand{\todo}[1]{{\color{blue}[\textit{#1}]}}
\newcommand{\reals}{{\mathbb R}}
\newcommand{\prob}[1]{\mathbb P \left [ #1 \right ]}
%\newcommand{\Egt}{U}
%\newcommand{\Eslt}{D}

\newcommand{\vecf}{\mathbf{f}} 

%\newcommand{\sfv}{\begin{proof}See full version.\end{proof}}
\newcommand{\sfv}{}
\newcommand{\R}{{\mathbb R}}
\renewcommand{\H}{{\cal H}}
\renewcommand{\O}{{\cal O}}
\newcommand{\etal}{{et al.~}}
\newcommand{\rk}{\mbox{\sc Rk}}
\newcommand{\hatyi}{\hat{y}_i}
\newcommand{\hatk}{\hat{k}} 
\newcommand{\veca}{\mathbf{a}} 
\newcommand{\veczero}{\mathbf{0}}
\newcommand{\vecone}{\mathbf{1}} 
\newcommand{\vecx}{\mathbf{x}} 
\newcommand{\delc}{{\Delta_C(i)}}
\newcommand{\alg}{{\cal A}}
\newcommand{\obj}{{\cal O}}
\newcommand{\y}{{\mathbf Y}}
\newcommand{\bin}{{\mathbf{Bin}}}
\newcommand{\eat}[1]{}
\newcommand{\acc}{E}
\renewcommand{\k}{\kappa}

\newcommand{\A}{{\cal A}}
\newcommand{\B}{{\cal B}}

\newcommand{\I}{{\cal I}}
\newcommand{\D}{{\cal D}}
\newcommand{\X}{{\cal X}}
\newcommand{\Y}{{\cal Y}}
\newcommand{\Z}{{\cal Z}}
\renewcommand{\P}{{\cal P}}
\renewcommand{\S}{{\cal S}}
\newcommand{\cc}{cc}

\newcommand{\E}{{\mathbf E}}
%\newcommand{\comment}[1]{\marginpar{!!!}[#1]}

\newcommand{\err}{{\mathcal E}}
\newcommand{\comment}[1]{{\color{red} #1} \marginpar{!!!}}

\title{Vertex and Hyperedge Connectivity\\
 in Dynamic Graph Streams}

\numberofauthors{3} 
\author{
% You can go ahead and credit any number of authors here,
% e.g. one 'row of three' or two rows (consisting of one row of three
% and a second row of one, two or three).
%
% The command \alignauthor (no curly braces needed) should
% precede each author name, affiliation/snail-mail address and
% e-mail address. Additionally, tag each line of
% affiliation/address with \affaddr, and tag the
% e-mail address with \email.
%
% 1st. author
\alignauthor
Sudipto Guha\titlenote{Supported by NSF Awards CCF-1117216.}\\
       \affaddr{University of Pennsylvania}\\
       \email{sudipto@seas.upenn.edu}
% 2nd. author
\alignauthor
Andrew McGregor\titlenote{Supported by NSF Awards CCF-0953754, IIS-1251110,  CCF-1320719, and a Google Research Award.}\\
       \affaddr{University of Massachusetts}\\
       \email{mcgregor@cs.umass.edu}
% 3rd. author
\alignauthor David Tench\\
       \affaddr{University of Massachusetts}\\
       \email{dtench@cs.umass.edu}
}

%
%\author{
%Sudipto Guha\thanks{University of Pennsylvania. 
%Supported by NSF Awards CCF-1117216.
%\texttt{sudipto@seas.upenn.edu}}
%\and 
%Andrew McGregor\thanks{University of Massachusetts Amherst. Supported by NSF CAREER Award CCF-0953754.
% \texttt{mcgregor@cs.umass.edu}. }
% \and 
%David Tench\thanks{University of Massachusetts Amherst. Supported by NSF CAREER Award CCF-0953754.
% \texttt{dtench@cs.umass.edu}. }
% }

\newcommand{\pparagraph}[1]{\vspace{0.1in}\noindent{\bf \boldmath #1}}
\date{}

\newfont{\mycrnotice}{ptmr8t at 7pt}
\newfont{\myconfname}{ptmri8t at 7pt}
\let\crnotice\mycrnotice%
\let\confname\myconfname%

\permission{Permission to make digital or hard copies of all or part of this work for personal or classroom use is granted without fee provided that copies are not made or distributed for profit or commercial advantage and that copies bear this notice and the full citation on the first page. Copyrights for components of this work owned by others than ACM must be honored. Abstracting with credit is permitted. To copy otherwise, or republish, to post on servers or to redistribute to lists, requires prior specific permission and/or a fee. Request permissions from permissions@acm.org.}
\conferenceinfo{PODS'15,}{May 31--June 4, 2015, Melbourne, Victoria, Australia.}
\copyrightetc{Copyright \copyright~2015 ACM \the\acmcopyr}
\crdata{978-1-4503-2757-2/15/05\ ...\$15.00.\\
http://dx.doi.org/10.1145/2745754.2745763}

\clubpenalty=10000 
\widowpenalty = 10000

\begin{document}

\maketitle

\fi

%\begin{abstract}
A growing body of work addresses the challenge of processing dynamic graph streams: a graph is defined by a sequence of edge insertions and deletions and the goal is to construct synopses and compute properties of the graph while using only limited memory. 
Linear sketches have proved to be a powerful technique in this model and can also be used  to minimize communication in distributed graph processing. 

We present the first linear sketches for estimating vertex connectivity and constructing hypergraph sparsifiers. 
Vertex connectivity exhibits markedly different combinatorial structure than edge connectivity and appears to be harder to estimate in the dynamic graph stream model. Our hypergraph result generalizes the work of Ahn et al.~\cite{AhnGM12a} on graph sparsification and has the added benefit of significantly simplifying the previous results. One of the main ideas is related to the problem of reconstructing subgraphs that satisfy a specific sparsity property. We introduce a more general notion of graph degeneracy and extend the graph reconstruction result of Becker et al.~\cite{BeckerMNRST11}.
%Vertex connectivity is significantly different from edge connectivity 
%and even the proof of existence of sparse certificates requires nontrivial algorithms (Cheriyan et al. SICOMP 1993). However existing techniques 
%do not provide sublinear space algorithms in the dynamic setting -- which is addressed by the results herein. 
%\end{abstract}

%\category{F.2}{Analysis of Algorithms \& Problem Complexity}{}

%\terms{Algorithms, Theory}

%\keywords{data streams; graph sketching; vertex connectivity; hypergraphs; sparsification}

\section{The Dynamic Graph Streaming Setting}

Massive graphs arise in many applications. Popular examples include the web-graph, social networks, and biological networks but, more generally, graphs are a natural abstraction whenever we have information about both a set of basic entities and relationships between these entities. Unfortunately, it is not possible to use existing algorithms to process many of these graphs; many of these graphs are too large to be stored in main memory and are constantly changing. Rather, there is a growing need to design new algorithms for even basic graph problems in the relevant computational models. 

In this chapter, we consider algorithms in the dynamic data stream and linear sketching models. In the dynamic data stream model, a sequence of edge insertions and deletions defines an input graph and the goal is to solve a specific problem on this graph given only one-way access to the input sequence and limited working memory. While insert-only graph streaming has been an active area of research for over a decade, it is only relatively recently algorithms have been found that handle insertions and deletions\cite{AhnGM12a,AhnGM12b,AhnGM13,KapralovLMMS14,KapralovW14,GoelKP12,KutzkovP14a}. The main technique used in these algorithms is \emph{linear sketching} where a random linear projection of the input graph is maintained as the graph is updated. To be useful, we need to be able to a) store the projection of the graph in small space and b) solve the problem of interest given only the projection of the graph. While linear sketching is a classic technique for solving statistical problems in the data stream model, it was long thought unlikely to be useful in the context of combinatorial problems on graphs. Not only do linear sketches allow us to process edge deletions (a deletion can just be viewed as a ``negative" insertion) but the linearity of the resulting data structures enables a rich set of algorithmic operations to be performed after the sketch has been generated. Linear sketches are also a useful technique for reducing communication when processing distributed graphs. For a recent survey of graph streaming and sketching  see  \cite{McGregor14}.
 

\subsection{Our Contributions and Related Work}
We present sketch-based dynamic graph algorithms for three basic graph problems: computing vertex connectivity, graph reconstruction, and hypergraph sparsification. All our algorithms run in (low) polynomial time, typically linear in the number of edges. However, our primary focus is on space complexity, as is the convention in much of the data streams literature. In what follows, let $n$ denote the number of vertices in the graph.
% algorithms, is  focus our attention on space complexity which is more critical in our setting.
%All the algorithms use $O(n\polylog n)$ space.

\pparagraph{Vertex Connectivity.} To date, the main success story for graph sketching has been  about edge connectivity, i.e., estimating how many edges need to be removed to disconnect the graph, and estimating the size of cuts. In this chapter we present the first dynamic graph stream algorithms for vertex connectivity, i.e., estimating how many \emph{vertices} need to be removed to disconnect the graph. While it can be shown that edge connectivity is an upper bound for vertex connectivity, the vertex connectivity of a graph can be much smaller. Furthermore, the combinatorial structure relevant to both quantities is very different. For example, edge-connectivity is transitive\footnote{If it takes at least $ k$ edge deletions to disconnect $u$ and $v$ and it takes at least $ k$ edge deletions to disconnect $v$ and $w$, then it takes at least $ k$ edge deletions to disconnect $u$ and $w$.} whereas vertex-connectivity is not. A celebrated result by Karger \cite{karger1994} bounds the number of near minimum cuts whereas no analogous bound is known for vertex removal.  Feige et al.~\cite{FeigeHL05} discuss issues that arise specific to vertex connectivity in the context of approximation algorithms and embeddings.


In Section \ref{sec:vertex}, we present two sketch-based algorithms for vertex connectivity. The first algorithm uses $O(kn\polylog n)$ space and constructs a data structure such that, at the end of the stream, it is possible to test whether the removal of a queried set of at most $k$ vertices would disconnect the graph. We  prove that this algorithm is optimal in terms of its space use. The second algorithm estimates the vertex connectivity up to a  $(1+\epsilon)$ factor using $O(\epsilon^{-1} kn\polylog n)$ space where $k$ is an upper bound on the vertex connectivity. 

No stream algorithms were previously known that supported both edge insertions and deletions. Existing approaches either use $\Omega(n^2)$ space \cite{Sankowski07} or only handle insertions \cite{EppsteinGIN97}. With only insertions, Eppstein et al.~\cite{EppsteinGIN97} proved that $O(kn \polylog n)$ space was sufficient. Their algorithm drops an inserted edge $\{u,v\}$ iff there already exists $k$ vertex-disjoint paths between $u$ and $v$ amongst the edges stored thus far. Such an algorithm fails in the presence of edge deletions since some of the vertex disjoint paths that existed when an edge was ignored need not exist if edges are subsequently deleted.

\pparagraph{Graph Reconstruction.} Our next result relates to reconstructing graphs rather than estimating properties of the graph. Becker et al.~\cite{BeckerMNRST11} show that is possible to reconstruct a $d$-degenerate graph given an $O(d \polylog n)$ size sketch of each row of the adjacency matrix of the graph. In Section \ref{sec:recon}, we define the  $d$-cut-degeneracy and show that the strictly larger class of graphs that satisfy this property can also be reconstructed given an $O(d \polylog n)$-size sketch of each row. Moreover, even if the graph is not $d$-cut-degenerate we show that we can find all edges with a certain connectivity property. This will be an integral part of our algorithm for hypergraph sparsification. For this purpose, we also prove the first dynamic graph stream algorithms for hypergraph connectivity in this section. We also extend the vertex connectivity results to hypergraphs.


\pparagraph{Hypergraph Sparsification.}
Hypergraph sparsification is a natural extension of graph sparsification. Given a hypergraph, the goal is to find a sparse weighted subgraph such that the weight of every cut in the subgraph is within a $(1+\epsilon)$ factor of the weight of the corresponding cut in the original hypergraph. 
Estimating hypergraph cuts has applications in video object segmentation \cite{HuangLM09}, network security analysis \cite{Yamaguchi14},  load balancing in parallel computing \cite{CatalyurekBDBHR09}, and
modelling communication in parallel sparse-martix vector multiplication 
\cite{CatalyurekA99}.

Kogan and Krauthgamer \cite{KoganK14} recently presented the first stream algorithm for hypergraph sparsification in the insert-only model. In Section \ref{sec:hypesparsification}, we present the first algorithm that supports both edge insertions and deletions. The algorithm uses $O(n\polylog n)$ space assuming that size of the hyperedges is bounded by a  constant. This result is part of a growing body of work on processing hypergraphs in the data stream model\cite{SahaG09,EmekR14,RadhakrishnanS11,Sun13,KoganK14}. There are numerous challenges in extending previous work on graph sparsification \cite{AhnGM12b,AhnGM13,KapralovLMMS14,KapralovW14,GoelKP12} to hypergraph sparsification and we discuss these in Section \ref{sec:hypesparsification}. In the process of overcoming these challenges, we also identify a simpler approach for graph sparsification in the data stream model.

\section{Models and Preliminaries}

\pparagraph{Graphs Preliminaries.}
A hypergraph is specified by a set of vertices $V=\{v_1, \ldots, v_n\}$ and a set of subsets of $V$ called hyperedges. In this chapter we assume all hyperedges have cardinality at most $r$ for some constant $r$. The special case when all hyperedges have cardinality exactly two corresponds to the standard definition of a graph. All graphs and hypergraphs discussed in this chapter will be undirected.
It will be convenient to define the following notation: Let $\delta_G(S)$ be the set of hyperedges that cross the cut $(S,V\setminus S)$ in the hypergraph $G$ where we say a hyperedge $e$ crosses $(S,V\setminus S)$ if $e\cap S\neq \emptyset$ and $e\cap (V\setminus S)\neq \emptyset$. For any hyperedge $e$, define $\lambda_e(G)$ to be the minimum cardinality of a cut that includes $e$. A \emph{spanning graph} $H=(V,E)$ of a hypergraph $G=(V,E)$ is a subgraph such that  $|\delta_{H}(S)|\geq \min (1, |\delta_G(S)|)$ for every $S\subset V$.

%Something about induced graphs and subgraphs on hypergraphs?


\pparagraph{Linear Sketches and Applications.}
All the algorithms presented in this chapter use linear sketches.

\begin{definition}[Linear Sketches]
A \emph{linear measurement} of a hypergraph on $n$ vertices is defined by a set of coefficients $\{c_e : e\in \P_r(V)\}$ where $\P_r(V)$ is the set of all subsets of $V$ of size at most $r$. Given a hypergraph $G=(V,E)$, the evaluation of this measurement is defined as 
$\sum_{e\in E} c_e$.
A \emph{sketch} is a collection of (non-adaptive) linear measurements. The cardinality of this collection is referred to as the \emph{size} of the sketch. We will assume that the magnitude of the coefficients $c_e$ is $\poly(n)$. We say a linear measurement is \emph{local} for node $v$ if the measurement only depends on hyper-edges incident to $v$, i.e., $c_e=0$ for all hyper-edges that do not include $v$. We say a sketch is \emph{vertex-based} if every linear measurement is local to some node.\end{definition}

Linear sketches have long been used in the context of data stream models because it is possible to maintain a sketch of the stream incrementally. Specifically, if the next stream update is an insertion or deletion of an edge, we can update the sketch by simply adding or subtracting the appropriate set of coefficients.
Sketches are also useful in distributed settings. In particular, the model considered by Becker et al.~\cite{BeckerMNRST11} was as follows: suppose there are $n+1$ players $P_1, \ldots, P_n$ and $Q$. The input for player $P_i$ is the set of (hyper-)edges that include the $i$th vertex of a graph $G$. Player $Q$ wants to compute something about this graph such as determining whether $G$ connected. To enable this, each of the players $P_1, \ldots, P_n$  simultaneously sends a message about their input to $Q$ such that the set of these $n$ messages contains sufficient information to complete $Q$'s computation. In the case of randomized protocols, we assume that all players have access to public random bits. The goal is to minimize the maximum length of the $n$ messages that are sent to $Q$. If a vertex-based sketch exists for the problem under consideration, then for each linear measurement, there is a single player that can evaluate this message and send it to $Q$.

%\begin{definition}[Sketch and vertex-based Sketch]
%
%\end{definition}

\section{Vertex Connectivity}\label{sec:vertex}

A natural approach to determining vertex connectivity could be to try to mimic the algorithm of Cheriyan et al.~\cite{CheriyanKT93}. They showed that the union of $k$ disjoint ``scan first search trees" (a generalization of breadth-first search trees) can be used to determine if a graph is $k$ vertex connected. A similar approach worked in data stream model for the case of edge-connectivity (which we discuss in further detail in the next section) but in that case the trees to be constructed could be arbitrary. Unfortunately, we can show %(see appendix) 
that any algorithm for constructing a scan-first search tree in the data stream model requires $\Omega(n^2)$ space even when there are no edge deletions.\\

%
%%The same difficulty arises in the context of proving that sparse certificates of vertex connectivity exist. 
%It is not very difficult to show that the union of $k$ edge disjoint spanning trees prove that a graph is $k$ edge connected -- proving something similar in the context of vertex $k$-connectivity  is nontrivial. It was first shown by Cheriyan et al. \cite{CheriyanKT93} that the union of $k$ ``scan first search'' trees (where each such tree is constructed on the residual graph defined by the removal of the edges in the previous trees) provide a sparse certificate of vertex $k$-connectivity with $O(kn)$ edges. 

%Breadth first search trees are scan first search trees, but there exists scan first search trees that not breadth first search trees. This fact is highlighted and exploited in \cite{CheriyanKT93} from the perspective of parallel algorithms -- while the construction of breadth first is difficult to achieve in a parallel setting; the construction of scan first trees was shown to be much easier. However from the perspective of space complexity (the parallel algorithms in \cite{CheriyanKT93} used shared memory) the construction of a single scan first tree under edge insertions only requires $\Omega(n^2)$ space. 
%
%

A scan first search tree (SFST) of a graph \cite{CheriyanKT93} is defined as follows: The tree is initially empty, all vertices except the root (chosen arbitrarily) are \emph{unmarked}, and 
 all vertices are \emph{unscanned}. At each step we \emph{scan} an marked but unscanned vertex. For each vertex $x$ that is being scanned, all edges from $x$ to unmarked neighbors of $x$ are added to the tree and the unmarked neighbors are marked. This continues until no marked but unscanned vertices remain.

\begin{theorem}
Any data stream algorithm that constructs a SFST with probability at least $3/4$ requires $\Omega(n^2)$ space.
\end{theorem}
\begin{proof}
The proof is by a reduction from the communication problem of indexing \cite{Ablayev96}. Suppose Alice has a binary string $x\in \{0,1\}^{n^2}$ indexed by $[n]\times [n]$ and Bob wants to compute $x_{i,j}$ for some index $(i,j)\in [n]\times [n]$ that is unknown to Alice. This requires $\Omega(n^2)$ bits to be communicated from Alice to Bob if Bob is to learn $x_{i,j}$ with probability at least $3/4$.
Suppose we have a data stream algorithm for constructing an SFST. Alice creates a graph on nodes $T\cup U\cup V \cup W$ where $T=\{t_1, \ldots, t_n\}, U=\{u_1, \ldots, u_n\}, V=\{v_1, \ldots, v_n\}$, and $W=\{w_1, \ldots, w_n\}$. She adds edges $\{t_k,u_\ell\}$ and $\{v_\ell,t_k\}$ for each $\ell,k$ such that $x_{\ell,k}=1$.  Alice runs the scan-first search algorithm and sends the contents of her memory to Bob. Bob adds the edge $\{u_i,v_i\}$. Note that any SFST includes all neighbors of $u_i$ or $v_i$. In particular, $x_{i,j}=1$ iff at least one of $\{t_j,u_i\}$ or $\{v_i,w_j\}$ is present in the SFST constructed. Hence, the algorithm must have used $\Omega(n^2)$ space.
\end{proof}


To avoid this issue, we take a different approach based on finding arbitrary spanning trees for the induced graph on a random subset of vertices.\footnote{We note that the idea of subsampling vertices
% in the context of vertex connectivity 
 was recently explored by Censor-Hillel et al.~\cite{Censor-HillelGK14,Censor-HillelGGHK15}. They showed that if each vertex of a $k$-vertex-connected graph is subsampled with probability $p=\Omega(\sqrt{\log n / k})$ then the resulting graph has vertex connectivity $\Omega(kp^2)$. We do not make use of this result in our work as it does not lead to an approximation factor better than $\sqrt{k}$.
} We will use the following result for finding these spanning trees.

\begin{theorem}[Ahn et al.~\cite{AhnGM12a}]\label{thm:spantree}
For a graph on $n$ vertices, there exists a vertex-based sketch of size $O(n\polylog n)$ from which we can construct a spanning forest with high probability.
\end{theorem}

%roughly $1/\sqrt{k}$ is sufficient to distinguish the cases when the vertex connectivity is at least $k$ from the case when the vertex connectivity is at most $\sqrt k$. 
%Combining that analysis with the above theorem leads to a $O(n {k}^{-1} \polylog n)$ space algorithm. 

Note that in this section we restrict our attention to graphs rather than hypergraphs. However, in the next section we will explain how the vertex connectivity results extend to hypergraphs.

\subsection{Warm-Up: Vertex Connectivity Queries }

For $i=1,2,\ldots, R:=16\cdot k^2 \ln n$, let $G_i$ be a graph formed by deleting each vertex in $G$ with probability $1-1/k$. Let $T_i$ be an arbitrary spanning forest of $G_i$ and define $H=T_1\cup T_2 \cup \ldots \cup T_R$.

\begin{lemma}
Let $S$ be an arbitrary collection of at most $k$ vertices. With high probability, $H\setminus S$ is connected iff $G\setminus S$ is connected.
\end{lemma}

\begin{proof}
First we note that $H$ has the same set of vertices as $G$ with high probability. This follows because the probability a given vertex is not in $H$ is $(1-1/k)^R\leq \exp(-16\cdot k \cdot \ln n)=n^{-16k}$ and hence by an application of the union bound, all vertices in $G$ are also in $H$ with probability at least $1-n^{-(16k-1)}$. Then since $H$ is a subgraph of $G$, then $G\setminus S$ disconnected implies  $H\setminus S$ disconnected. It remains to prove that $G\setminus S$ connected implies $H\setminus S$ connected.

Assume $G\setminus S$ is connected. Consider an arbitrary pair of vertices $s,t \not \in S$ and let 
$s=v_0\rightarrow v_1 \rightarrow v_2 \rightarrow \ldots \rightarrow v_\ell=t$
be a  path between $s$ and $t$ in $G\setminus S$.
Then note that there is a path between $v_i$ and $v_{i+1}$ in $H\setminus S$ if there exists $G_i$ such that $G_i\cap S=\emptyset$ and $v_i,v_{i+1}\in H\setminus S$. This follows because if $\{v_i,v_{i+1}\}\in G_i$ and $G_i\cap S=\emptyset$ then $T_i\setminus S$ either contains $\{v_i,v_j\}$ or a path between between $v_i$ and $v_j$. Hence,
\[
\prob{\mbox{$v_i$ and $v_{i+1}$ are connected in $T_i\setminus S$}} \geq 1/k^2 (1-1/k)^k
\]
and therefore 
$$\prob{\mbox{$v_i$ and $v_{i+1}$ are disconnected in $T_i\setminus S$ for all $i\in [R]$}} \leq (1-1/k^2 (1-1/k)^k)^R \leq 1/n^4 \ .$$

\begin{comment}
\begin{eqnarray*}
& & \prob{\mbox{$v_i$ and $v_{i+1}$ are disconnected in $T_i\setminus S$ for all $i\in [R]$}} \\
&\leq & (1-1/k^2 (1-1/k)^k)^R 
%\\
%&\leq &\exp(-R/k^2 (1-1/k)^k)
\leq 1/n^4 \ .
\end{eqnarray*}
\end{comment}

Taking the union bound over all $\ell <n$ pairs $\{v_i, v_{i+1}\}$, we conclude that 
$s$ and $t$ are connected in $H\setminus S$ with probability at least $1-1/n^3$. By applying the union bound again, with probability at least $1-1/n^2$, $s$ is connected in $H\setminus S$ to all other vertices.
\end{proof}

Our algorithm constructs a spanning forest for each of $G_1, \ldots, G_R$ using the algorithm referenced in Theorem \ref{thm:spantree}. Note that since each $G_i$ has $O(n/k)$ vertices with high probability, we can construct these $R$ trees in $R\times O(n/k \polylog n)=O(nk\polylog n)$ space. This gives us the following theorem.

\begin{theorem}
There is a sketch-based dynamic graph algorithm that uses $O(kn \polylog n)$ space to test whether a set of vertices $S$ of size at most $k$ disconnects the graph. The query set $S$ is specified at the end of the stream.
\end{theorem}

We next prove that the above query algorithm is space-optimal.

\begin{theorem}
Any dynamic graph algorithm that allows us to test, with probability at least $3/4$, whether a queried set of at most $k$ vertices disconnects the graph requires $\Omega(kn)$ space.
\end{theorem}
\begin{proof}
The proof is by a reduction from the communication problem of indexing \cite{Ablayev96}. Suppose Alice has a binary string $x\in \{0,1\}^{(k+1)\times n}$ indexed by $[k+1]\times [n]$ and Bob wants to compute $x_{i,j}$ for some  index $(i,j)\in [k+1]\times [n]$ that is unknown to Alice. This requires $\Omega(nk)$ bits to be communicated from Alice to Bob if Bob is to be successful with probability at least $3/4$. Consider the protocol where the players create a bipartite graph on vertices $L\cup R$ where $L=\{l_1, \ldots, l_{k+1}\}$ and $R=\{r_1, \ldots, r_n\}$. Alice adds edges $\{l_i,r_j\}$ for all pairs $(i,j)$ such that $x_{i,j}=1$. 
Alice runs the algorithm and sends the state to Bob. Bob adds edges $\{r_\ell,r_{\ell'}\}$ for all $\ell,\ell'\neq j$ and deletes all vertices in $L$ except $l_i$. Now $r_j$ is connected to the rest of the graph iff the $x_{i,j}=1$.
\end{proof}


\subsection{Vertex Connectivity}

For $i=1,2,\ldots, R:=160\cdot k^2 \epsilon^{-1}  \ln n$, let $G_i$ be a graph formed by deleting each vertex in $G$ with probability $1-1/k$. As before, let $T_i$ be an arbitrary spanning forest of $G_i$ and define $H=T_1\cup T_2 \cup \ldots \cup T_R$.

%Let 
%Let $R=8\times k^2\epsilon^{-1} \ln n$.

\begin{theorem}\label{thm:everyonesgonetosleep}
Let $S$ be a subset of $V$ of size $k$. Consider any pair of vertices $u,v\in V\setminus S$ such that there are at least $(1+\epsilon)k$ vertex-disjoint paths between $u$ and $v$ in $G$. Then, 
\[\prob{u \mbox{ and }v \mbox{ are connected in } G_S}\geq 1-4/n^{10k}\] 
where $G_S=\cup_{i\in U(S)} G_i$ and $U(S)=\{i:G_i\cap S=\emptyset\}$ is the set of sampled graphs with no vertices in $S$. 
\end{theorem}
\begin{proof}
We first argue that $|U(S)|$ is large with high probability. Then $\expec{|U(S)|}=(1-1/k)^{k} R \geq R/4$. By an application of the Chernoff bound:
\[
\prob{|U(S)|\leq 1/2\times R/4}\leq  e^{-1/4\times R/4 \times 1/3}<1/n^{10k} \ .
\]
In the rest of the proof we condition on event $|U(S)|\geq r:=R/8$.
% and note that the set of $G_i$ such that $G_i\cap S=\emptyset$ are inpendent.

Note that there are $t\geq \epsilon k$ vertex-disjoint paths between $u$ and $v$ in $G\setminus S$. Call these paths $P_1, \ldots, P_{t}$. For each $P_i$, let $a_i$ be the edge incident to $u$, let $c_i$ be the edge incident to $v$, and let $B_i$ be the remaining edges in $P_i$. Note that $a_i$ and $c_i$ need not be distinct and $B_i$ could be empty.

\begin{claim} The followings three probabilities are each larger than $1-1/n^{10k}$:
\[
\prob{a_i \in G_S \mbox{ for at least $3t/4$ values of $i$}}\]
\[\prob{B_i \subseteq G_S \mbox{ for at least $3t/4$ values of $i$}}\]
\[\prob{c_i \in G_S \mbox{ for at least $3t/4$ values of $i$}} \ .
\]
%\begin{align*}
%\prob{B_i \subseteq G_S \mbox{ for at least $3t/4$ values of $i$}}>& 1-1/n^{3k} \\
%\prob{a_i \in G_S \mbox{ for at least $3t/4$ values of $i$}} >& 1-1/n^{3k}\\
%\prob{c_i \in G_S \mbox{ for at least $3t/4$ values of $i$}}>& 1-1/n^{3k} \ .
%\end{align*}
% and
%
%\[\prob{\mbox{for at least $t/4$ values of $i\in [t]$}~,~ c_i\not \in G_S}<1/n^{2k} \ .\]
\end{claim}
\begin{proof}
Each edge in $B_i$ is not present in $G_S$ with probability $(1-1/k^2)^{r}$. Hence, by the union bound,
$\prob{B_i\not \subseteq G_S}\leq |B_i| (1-1/k^2)^{r}$.
Also by the union bound,
%\begin{eqnarray*}
%\prob{B_i \not \subseteq G_S \mbox{ for more than $t/4$ values of $i$}}
%&<&  {t\choose t/4} (n (1-1/k^2)^{r})^{t/4} \\
%&<& e^{t \ln 2+(\ln n-r/k^2)t/4}<1/n^{2k} \ .
%\end{eqnarray*}
\begin{eqnarray*}
& & \prob{B_i \not \subseteq G_S \mbox{ for more than $t/4$ values of $i$}}
\\ &<&   {t\choose t/4} (n (1-1/k^2)^{r})^{t/4} \\
&< &  e^{t \ln 2+(\ln n-r/k^2)t/4}<1/n^{10k} \ .
\end{eqnarray*}

The proofs for $a_i$ and $c_i$ are entirely symmetric so we just consider $a_i$. 
Consider the set $U'(S)=U(S)\cap \{j: u\in G_j\}$. Note that for $j\in U'(S)$ we have
$\prob{a_i\in G_j}=1/k$ and by the union bound,
\begin{eqnarray*}
& & \prob{a_i\not \in \cup_{j\in U'(S)}G_j \mbox{ for at least $t/4$ values of $i$}}\\
& \leq & {t \choose t/4} (1-1/k)^{|U'(S)|  t/4} \\
& \leq & 2^t \exp \left (\frac{-|U'(S)|  t}{(4k)}\right )
%<1/n^{3k} 
\ .
\end{eqnarray*}

%We next argue that $|U'(S)|$ is large with high probability. 
%%We first argue that $s\in G_i$ for a large fraction of $i\in U(S)$ with high probability. 
%By an application of the Chernoff bound,
%\[
%\prob{|U'(S)|\leq |U(S)|/(2k)}
%%&=&
%%\prob{\mbox{for at most $|U(S)|/(2k)$ values of $j\in U(S)$}~,~s\not \in G_j}\\
%\leq  \exp(-1/4 \times |U(S)|/k \times 1/3)\]
%%\end{eqnarray*}
Let $E$ be the event that $|U'(S)|\leq |U(S)|/(2k)$. Then, by an application of the Chernoff bound:
\begin{eqnarray*}
& & \prob{a_i\not \in G_S \mbox{ for at least $t/4$ values of $i$}} \\
& \leq &
\prob{E}   \\
& & + \prob{a_i\not \in \cup_{j\in U'(S)}G_j  \mbox{ for at least $t/4$ values of $i$} \mid \neg E} \\
& \leq &
\exp(-1/4 \times |U(S)|/k \times 1/3)   \\
& & \quad + \prob{a_i\not \in \cup_{j\in U'(S)}G_j  \mbox{ for at least $t/4$ values of $i$} \mid \neg E} \\
& \leq &
\exp(-1/4 \times r/k \times 1/3) +2^t \exp(-r/(2k)\times  t/(4k)) \\
&<&  1/n^{10k}\ .
%& \leq &
%e^{-1/4 \times |U(S)|/k \times 1/3} +2^t \exp(-|U'(S)|  t/(4k))
\end{eqnarray*}
\end{proof}
It follows from the claim that there exists $i$ such that $P_i\in G_S$ (and therefore $u$ and $v$ are connected in $G_S$) with probability at least $1-3/n^{10k}$. The conditioning on $|U(S)|\geq r$ decreases this by another $1/n^{10k}$.
\end{proof}

\begin{corollary}
If $G$ is $(1+\epsilon)k$-vertex-connected then $H$ is $k$-vertex-connected with high probability. If $H$ is  $k$-vertex connected then $G$ is $k$-vertex connected.
% at least $1-1/n$.
\end{corollary}
\begin{proof}
The first part of the corollary follows from Theorem \ref{thm:everyonesgonetosleep} by applying the union bound over all $O(n^k)$ subsets of size at most $k$ and $O(n^2)$ choices of $u$ and $v$. Note that $u$ and $v$  connected in $G_S$ implies $u$ and $v$ are connected in $H$ since $H$ includes a spanning forest of $G_S$. The second part  is implied by the fact $H$ is a subgraph of $G$.
\end{proof}

As in the previous section, our algorithm is simply to construct $H$ be  using the algorithm referenced in Theorem \ref{thm:spantree} to construct $T_1, \ldots, T_R$. We can then run any vertex connectivity algorithm on $H$ in post-processing. Since each $G_i$ has $O(n/k)$ vertices with high probability, we can construct these $R$ trees in $R\times O(n/k \cdot \polylog n)=O(nk\epsilon^{-1}\polylog n)$ space. This gives us the following theorem.

\begin{theorem}
There is a sketch-based dynamic graph algorithm that uses $O(kn\epsilon^{-1} \polylog n)$ space to distinguish $(1+\epsilon)k$-vertex connected graphs from $k$-connected graphs.
\end{theorem}
%
%Applying the above algorithm with $\epsilon=1/k$ yields the following theorem.
%
%\begin{corollary}
%Testing $k$-vertex connectivity is possible in $O(k^2 n \polylog n)$ space.
%\end{corollary}
%
%



\section{Reconstructing Hypergraphs}\label{sec:recon}

We next present sketches for reconstructing cut-degenerate hypergraphs. Recall that a hypergraph is $d$-degenerate if all induced subgraphs have a vertex of degree at most $d$. Cut-degeneracy is defined as follows.

\begin{definition}
A hypergraph is \emph{$d$-cut-degenerate} if every induced subgraph has a cut of size at most $d$.
\end{definition}

The following lemma establishes that this is a strictly weaker property than $d$-degeneracy.

\begin{lemma}
Any hypergraph that is $d$-degenerate is also $d$-cut-degenerate. There exists graphs that are    $d$-cut-degenerate but not $d$-degenerate.
\end{lemma}
\begin{proof}
Since the degree of a vertex $v$ is exactly the size of the cut $(\{v\},V\setminus \{v\})$ it is immediate that $d$-degeneracy implies $d$-cut-degeneracy. For an example that $d$-cut-degenerate does not imply it is $d$-degenerate consider the graph $G$ on eight vertices $\{v_1,v_2,v_3,v_4,u_1,u_2,u_3, u_4\}$ with edges $\{v_i,v_j\}, \{u_i,u_j\}$ for all $i,j$ except $i=1,j=4$ and edges $\{v_1,u_1\}$ and $\{v_4,u_4\}$. 
Then $G$ has minimum degree 3 and is therefore not 2-degenerate while it is $2$-cut-degenerate.
\end{proof}


Becker et al.~\cite{BeckerMNRST11} showed how to reconstruct a $d$-degenerate graph in the simultaneous communication model if each player sends an $O(d\polylog n)$ bit message. We will show that it is also possible to reconstruct any $d$-cut-degenerate with the same message complexity. Even if the graph is not cut-degenerate, we show that is possible to reconstruct all edges with a certain connectivity property. We will subsequently use this fact in Section \ref{sec:hypesparsification}.


\subsection{Skeletons for Hypergraphs}
\label{sec:kskeletons}
We first review the existing results on constructing $k$-skeletons \cite{AhnGM12a} that we will need for our new results. In doing so, we generalize the previous work to the case of hypergraphs.  In particular, this leads to the first dynamic graph algorithm for determining hypergraph connectivity.

\begin{definition}[$k$-skeleton]
Given a hypergraph $H=(V,E)$, a subgraph $H'=(V,E')$ is a \emph{$k$-skeleton} of $H$ if for any $S\subset V$, $|\delta_{H'}(S)| \geq \min (|\delta_{H}(S)|,k)$.
\end{definition}

In particular, any spanning graph is a $1$-skeleton and it can be shown that $F_1\cup F_2 \cup \ldots \cup F_k$ is  a $k$-skeleton \cite{AhnGM12a} of $G$ if $F_i$ is a spanning graph of $G\setminus   (\cup_{j=1}^{i-1} F_{j} )$. The next lemma establishes that given an arbitrary $k$-skeleton of a graph we can exactly determine the set of edges with $\lambda_e(G)\leq k-1$.

\begin{lemma}\label{lem:skeletonlambda}
Let $H$ be a $k$-skeleton of $G$ then $\lambda_e(H)\leq k-1$ iff $\lambda_e(G)\leq k-1$.
\end{lemma} 
\begin{proof}
Since $H$ is a subgraph $\lambda_e(H)\leq \lambda_e(G)$ and hence $\lambda_e(G)\leq k-1$ implies $\lambda_e(H)\leq k-1$. Using the fact that $H$ is a $k$-skeleton $\lambda_e(H)\geq \min(k,\lambda_e(G))$ and hence, if $  \lambda_e(H)\leq k-1$ it must be that $\lambda_e(G)\leq k-1$.
\end{proof}

%We present 
%
%We start with a simple algorithm for finding a spanning forest of a graph and then show how to emulate this algorithm via sketches.

%Of course this is a trivial algorithm! The challenge is to perform it via a set of simultaneous linear measurements.

\pparagraph{Constructing Spanning Graphs.}
For each vertex $v_i\in V$, define the vector $\veca^i\in \{-1,0,1,2, $ \\ $\ldots, r-1\}^d$ where $d=\sum_{i=2}^r {n\choose i}$ is the number of possible hyperedges of size at most $r$:
% whose $\{j,k\}\in {[n]\choose 2}$ entry is defined as  
 \[
  \veca^i_{e}=
\begin{cases}
   |e|-1     & \mbox{if } i=\min e \mbox{ and } e \in E \\
   -1     & \mbox{if } i\in e\setminus \min e  \mbox{ and }  e \in E \\
   0      & \mbox{otherwise}
\end{cases}
 \]
 where $e$ ranges over all subsets of $V$ of size between $2$ and $r$ and $\min e$ denotes the smallest ID of a node in $e$.
Observe that these vectors have the property that
%. First note that the support of $\veca_i$ corresponds  exactly to the set of neighbors of $v_i$, i.e., $\support(\veca_i)=\{(i,j):v_j\in \Gamma(v_i)\}$.
%Furthermore, 
for any subset of vertices $\{v_i\}_{i\in S}$, the non-zero entries of $\sum_{i\in S} \veca^i$ correspond exactly to $\delta(S)$. This follows because the only subsets of 
\[\{|e|-1, \underbrace{-1, -1, \ldots, -1}_{|e|-1}\}\]
that sum to zero are the empty set and the entire set. Hence, the $e$-th coordinate of $\sum_{i\in S} \veca^i$ is zero iff either $e\not \in E$ or $e\subset S$ or $e\subset V\setminus S$.

The rest of algorithm proceeds exactly as in the case of (non-hyper) graphs \cite{AhnGM12a} and a reader that is very familiar with the previous work should feel free to skip the remainder of Section \ref{sec:kskeletons}. We construct the sketches $M\veca^1,  \ldots, M\veca^n$ 
where $M$ is chosen according to a distribution over matrices $\reals^{k \times d}$ where $k=\polylog (d)$. The distribution has the property that for any $\veca\in \reals^d$, it is possible to determine the index of a non-zero entry of $\veca$ given $M\veca$ with probability $1-1/\poly(n)$. Such as distribution is known to exist by a result of Jowhari et al.~\cite{JowhariST11}.
%
%
%We next use the following result by Jowhari et al.~\cite{JowhariST11}:
%%\begin{theorem}%[Jowhari et al.~\cite{JowhariST11}]
%There exists a distribution $\mu$ over  $\reals^{(\polylog d) \times d}$ matrices such that for any $\veca \in \reals^d$, the index of a non-zero entry of $\veca$ can be determined given $M\veca$ with probability $1-1/\poly(n)$ where $M\sim \mu$. 
%Note that we do not get to specify which non-zero entry is found if there is more than one.
%\end{theorem}
%As mentioned earlier, the goal of $\ell_0$-sampling is to take a non-zero vector $\bold{x}\in \reals^{d}$ and return a sample $j$ where
%\[
%\Pr_r[\mbox{sample equals $j$}] =
%\begin{cases}
%\frac{1}{|F_0(\vecx)|} & \mbox{ if } \vecx_j\neq 0\\
%0 & \mbox{ if } \vecx_j= 0
%\end{cases} \ .
%\]
Given $M\veca^1,  \ldots, M\veca^n$ we can find an edge across an arbitrary cut $(S,V\setminus S)$. To do this, we compute $\sum_{i\in S} M\veca^i=M( \sum_{i\in S} \veca^i)$. We can then determine the index of a non-zero entry of $\sum_{i\in S} \veca^i$ which corresponds to an element of $\delta(S)$ as required. It may appear that to test connectivity we need to test all $2^{n-1}-1$ possible cuts. Since the failure probability for each cut is only inverse polynomial in $n$ this would be problematic. However, it is possible to be more efficient and only test $O(n)$ cuts. See Ahn et al.~\cite{AhnGM12a} for details.

\begin{theorem}[Spanning Graph Sketches]
There exists a vertex-based sketch $\alg$ of size $O(n\polylog n)$ such that we can find a spanning graph of a hypergraph $G$ from $\alg(G)$ with high probability.
\end{theorem}

Note the above theorem can be substituted for Theorem \ref{thm:spantree} and the resulting algorithms for vertex connectivity go through for hypergraphs unchanged. 
%
%A useful feature of existing work \cite{JowhariST11} on $\ell_0$ sampling is that it can be performed via linear projections, i.e., for any  string $r$ there exists $M_r\in \reals^{k\times d}$ such that the sample can be reconstructed  from $M_r \vecx$. For the process to be successful with constant probability $k=O(\log^2 n)$ suffices. Consequently, given $M_r\vecx$ and $M_r \vecy$ we have enough information to determine a random sample from the set $\{i: x_i+y_i\neq 0\}$ since 
%\[M_r(\vecx+\vecy)=M_r \vecx+M_r \vecy \ .\]

%
%For example, for the graph in Figure~\ref{fig:example}, we have
%\[
%\begin{array}{rccccccl}  
%\veca^1 ~=~ (\! & 1 & 1 & 0& 0& 0 & 0&\!   ) \\
%\veca^2 ~=~ (\!  & -1 & 0& 0& 1& 0 & 0&\!  ) \\
%\veca^3 ~=~ (\!  & 0 & -1 & 0& -1 & 0 & 1&\!  ) \\
%\veca^4 ~=~ (\!  & 0 & 0 & 0 & 0 & 0 & -1 &\!  ) 
%\end{array}
%\]
%where the entries correspond to the sets 
%$\{1,2\}, $ $  \{1,3\}, $ $ \{1,4\}, $ $  \{2,3\}, $ $  \{2,4\}, $ $  \{3,4\}$ in that order. Note that the non-zero entries of 
%\[
%\begin{array}{rccccccl}  
%\veca^1+\veca^2  ~=~ (\! & 0 & 1 & 0& 1& 0 & 0&\!   )
%\end{array}
%\]
%correspond to $\{1,3\}$ and $\{2,3\}$ which are exactly the edges across the cut $(\{1,2\}, \{3,4\})$.

%Given the carefully chosen ingredients, 
 
\pparagraph{Constructing $k$-skeletons.}
As mentioned above, it suffices to find $F_1, \ldots, F_k$ such that $F_i$ is a spanning graph of $G\setminus   (\cup_{j=1}^{i-1} F_{j} )$. 
Do to this we use  $k$ independent spanning graph sketches $\alg^1(G), $ $ \alg^2(G), \ldots, $ $ \alg^k(G)$ as described in the previous section. We may construct $F_1$ from $\alg^1(G)$ because this is the functionality of a spanning graph sketch. Assuming we have already constructed $F_1, \ldots, F_{i-1}$ we can construct $F_i$ from:
\[
\alg^i(G-F_1-F_2\ldots - F_{i-1})=\alg^i(G)-\sum_{j=1}^{i-1} \alg^i(F_j) \ .
\]


\begin{theorem}[$k$-Skeleton Sketches]
There exists a vertex-based sketch $\B$ of size \\ $O(kn\polylog n)$ such that we can  find of a $k$-skeleton  a hypergraph $G$ from $\B(G)$ with high probability.
\end{theorem}

\subsection{Beyond $k$-Skeletons}
\label{sec:fleshingk}

One might be tempted as ask whether it was necessary to use $k$ independent spanning graph sketches $\alg^1, \ldots, \alg^k$  rather that reuse a single sketch $\alg$. If each application of the sketch $\alg$ fails to return a spanning graph with probability $\delta$, one might hope to use the union bound to argue that the probability that $\alg$ fails on any of the inputs $G, G-F_1, G-F_1-F_2, \ldots, G-F_1-\ldots -F_{k-1}$ is at most $k\delta$. \emph{But this would not be a valid application of the union bound!} The union bound states that for any \emph{fixed} set of $t$  events $B_1, \ldots, B_t$, we  have $\prob{B_1\cup \ldots \cup B_t}\leq \sum_i \prob{B_i}$. The issue is that the events in the above example are not fixed, i.e., they can not be specified a priori, since spanning graph $F_i$ is determined by the randomness in the sketch.\footnote{Another way to see that using the same sketch cannot work is that if it were possible to repeatedly remove each spanning graph from the sketch of the original graph, we would be able to reconstruct the entire graph using only a sketch of size $O(n \polylog n)$. Clearly this is not possible because it requires at $\Omega(n^2)$ bits to specify an arbitrary graph on $n$ vertices.} We belabor this point because, while the union bound was not applicable in the above case,  we will need it to prove our next result in a situation that is only subtly different and yet the union bound \emph{is} valid.

\subsubsection{Finding the light edges}
Given a graph $G=(V,E)$ and a postive integer $k$, recursively define 
\[E_i=\{e\in E: \lambda_e(G\setminus \bigcup_{j=1}^{i-1} E_i) \leq k\}\] and denote the union of these sets as:
\[\light_k(G)=\bigcup_{i\geq 1} E_i\ .\]
Note that if $G$ is $d$ cut-degenerate then $\light_d(G)=E$. Furthermore, there is at most $n$ values of $i$ such that $E_i$ is non-empty since removing each non-empty set $E_i$ from the graph increases the number of connected components.

Suppose $\B(G)$ is a sketch that returns an arbitrary $(k+1)$-skeleton of $G$ with failure probability $\delta=1/\poly(n)$. Then, since $E_1, E_2,\ldots, E_n$ are sets defined solely by the input graph (and not any randomness in a sketch) we can specify the fixed events 
\[
\begin{split}
B_i=\mbox{``We fail to return a $(k+1)$-skeleton sketch of } \\
\mbox{$G-E_1-\ldots -E_i$ given $\B(G-E_1-\ldots -E_i)$"}
\end{split}\]
and therefore use the union bound to establish that the probability that we find a $(k+1)$-skeleton of each of the relevant graphs with failure probability at most $n\delta=1/\poly(n)$.

We can therefore find the sets $E_1, E_2, \ldots, E_n$ as follows. Let
$S_i$ be an arbitrary $(k+1)$ skeleton of $G-E_1-\ldots E_{i-1}$. Assuming we have already determined $E_1, \ldots, E_{i-1}$, we can find $S_i$ using:
\[
\B(G-E_1-E_2\ldots - E_{i-1})=\B(G)-\sum_{j=1}^{i-1} \B(E_j) \ .
\]
Then, by appealing to Lemma~\ref{lem:skeletonlambda}, we know that we can then uniquely determine $E_i$ given $S_i$.

\begin{theorem}
There exists a vertex-based sketch of size $\tilde{O}(kn)$ from which  $\light_k(G)$ can be reconstructed for any hypergraph $G$. 
In the case of a $k$-cut-degenerate graph, this is the entire graph.
\end{theorem}

\subsubsection{What are the light edges?}
In this section, we restrict our attention to graphs rather than hypergraphs and show that the set of edges in $\light_k(G)$ can be defined in terms of the notion of \emph{strong connectivity} introduced by Bencz{\'u}r and Karger \cite{BenczurK96}. 


\begin{lemma} $\light_k(G)=\{e:k_e\leq k\}$ where $k_{\{u,v\}}$ is the maximum $k$ such that there is a set $S\subset V$ including $u$ and $v$ such that the induced graph on $S$ is $k$-edge-connected.
\end{lemma}
\begin{proof}
Define $t_e$ to be the minimum value of $k$ such that $e\in  \light_k(G)$. We  prove that $t_e=k_e$ and the result  follows.
To show $k_e \geq t_e$ suppose $t_e = t$ and then note that $e$ survives when we recursively remove edges with edge connectivity $t-1$. But the remaining components in this graph are at least $(t-1)+1=t$ connected so $k_e\geq t$.
To show that $k_e\leq t_e$, suppose $k_e=k$. Then there exists a vertex induced subgraph $H$ containing $e$ that is $k$-connected. But when we recursively remove edges with edge connectivity at most $k-1$ then no edge in $H$ can be removed. Hence, $t_e> (k-1)$ and so $t_e\geq k$.
\end{proof}

\section{Hypergraph Sparsification}
\label{sec:hypesparsification}

In this final section, we present a vertex-based sketch for constructing a sparsifier of a hypergraph. This yields the first dynamic graph stream algorithm for constructing a sparsifier of a hypergraph.  As an added bonus, our approach gives an algorithm and analysis that is significantly simpler than previous work on the specific case of graph sparsification \cite{AhnGM12b,GoelKP12}.
% In particular, our proofs are almost entirely self-contained and does not 


\begin{definition}[Hypergraph Sparsifier]
A weighted subgraph $H=(V,E',w)$ of a hypergraph $G=(V,E)$ is a \emph{sparsfier} if for all $S\subset V$,
$
\sum_{e\in \delta_H(S)} w(e) =(1\pm \epsilon) | \delta_G(S) |$.  
\end{definition}

Previous approaches to sparsification in the dynamic stream model relied on work by Fung et al.~\cite{FungHHP11}. To construct a \emph{graph} sparsifier, they showed that it was sufficient to independently sample every edge in the graph with probability $O(\epsilon^{-2} \lambda_e^{-1} \log n)$. Using their work required coopting their machinery and modifying it appropriately (e.g., replacing Chernoff arguments with careful Martingale arguments). Another downside to the previous approach is that the Fung et al.~result  does not seem to extend to the case of hypergraphs.\footnote{For the reader familiar with  Fung et al.~\cite{FungHHP11}, the issue is finding a suitable definition of cut-projection for hypergraphs and then proving a bound on the number of distinct cut-projections.}

Using our new-found ability (see the previous section) to find  the entire set of edges that are not $k$-strong, we present an algorithm that a) has a simpler, and almost self-contained, analysis and b) extends to hypergraphs. Our approach is closer in spirit to Bencz{\'u}r and Karger's original work on sparsification \cite{BenczurK96} which in turn is based on the following result by Karger \cite{Karger94}: if we sample each edge with probability $p\geq p^* =c \epsilon^{-2} \lambda^{-1} \log n$ where $\lambda$ is the cardinality of the minimum cut and $c\geq 0$ is some constant, and weight the sampled edges by $1/p$ then the resulting graph is a sparsifier with high probability.

The idea behind our algorithm is  as follows. For a hypergraph $G$, if we remove the hyperedges $\light_k(G)$ where $k=2 c \epsilon^{-2}\log n$, then every connected component in the remaining hypergraph has minimum cut of size greater than $2 c \epsilon^{-2}\log n$. Hence, for each of these components $p^*\leq 1/2$. Therefore, the graph formed by sampling the hyperedges in $G\setminus \light_k(G)$ with probability 1/2 (and doubling the weight of sampled hyperedges) and adding the set of hyperedges in $\light_k(G)$ with unit weights is a sparsifier of $G$. We then repeat this process until there are no hyperedges left to sample. 

\pparagraph{Algorithm.}
\begin{enumerate}
\item Generate a series of graphs $G_0, G_1, G_2\ldots $ where $G_i$ is formed by deleting each hyperedge in $G_{i-1}$ independently with probability 1/2 and $G_0=G$.
\item For $i=0,1,2,\ldots, \ell=3\log n$:
\begin{enumerate}
\item Let $F_i=\light_k(H_i)$ where $k=O(\epsilon^{-2} ( \log n + r))$ where 
$H_i=G_i \setminus (F_0\cup F_1 \cup F_2 \cup \ldots \cup F_{i-1})$
\end{enumerate}
\item Return $\bigcup_{i=0}^\ell 2^i \cdot F_i $ where $2^i \cdot F_i$ is the set of hyperedges in $F_i$ where each is given weight $2^i$.
\end{enumerate}

\pparagraph{Analysis.} The following lemma uses an argument due to Karger \cite{karger1994} combined with a hypergraph cut counting result by Kogan and Krauthgamer \cite{KoganK14}.

\begin{lemma}\label{lem:onestep}
$2 H_{i+1} \cup F_i$ is a $(1+\epsilon)$-sparsifier for $H_i$.
\end{lemma} 
\begin{proof}
It suffices to prove that $2 H_{i+1}$ is a $(1+\epsilon)$-sparsifier for $H_i\setminus F_i$. Furthermore, it suffices to consider each connected component of $H_i\setminus F_i$ separately. 

Let $C$ be an arbitrary connected component of $H_i\setminus F_i$ and note that $C$ has a minimum cut of size at least $k$. Let $C'$ be the graph formed by deleting each hyperedge in $C$ with probability $1/2$. Consider a cut of size $t$ in $C$ and let $X$ be the number of hyperedges in this cut that are in $C'$. Then $\expec{X}=t/2$ and by an application of the Chernoff bound,
$\prob{|X-t/2| \geq \epsilon t/2}\leq 2\exp(-\epsilon^2 t/6)
$.

%Kogan and Krauthgamer proved that  
The number of cuts of size at most $t$ is $\exp(O(rt/k+ t/k \cdot \log n))$  by appealing to a result by Kogan and Krauthgamer \cite{KoganK14}. By an application of the union bound, the probability that there exists a cut of  size $t$ such that the number of hyperedges in corresponding cut in $C'$ is not $(1\pm \epsilon)t/2$ is at most 
\[2\exp(-\epsilon^2 t/6) \cdot \exp(O(rt/k+ t/k \cdot \log n)) \ .\]
This probability is less than $1/n^{10}$
if $k\geq c\epsilon^{-2} (\log n + r)$ for some sufficiently large constant $c$. Hence, taking the union bound over all $t\geq k$ ensures that with probability at least $1/n^8$, for every cut in $C$, the fraction of edges in the corresponding cut in $C'$ is $(1\pm \epsilon)/2$.
\end{proof}

\begin{theorem}
$\bigcup_{i=0}^\ell 2^i \cdot F_i $ is a $(1+\epsilon)^\ell$-sparsifier of $G$ where $\ell=3\log n$.
\end{theorem}
\begin{proof}
The theorem follows by repeatedly applying Lemma \ref{lem:onestep}. Specifically,
\begin{enumerate}
\item $F_{\ell-1}$ is a $(1+\epsilon)$ sparsifier for $H_{\ell-1}$ since $H_\ell$ is the empty graph with high probability. 
\item $2 H_{\ell-1} \cup F_{\ell-2}$ is a $(1+\epsilon)$-sparsifier for $H_{\ell-2}$ and so $2 F_{\ell-1} \cup F_{\ell-2}$ is a $(1+\epsilon)^2$-sparsifier for $H_{\ell-2}$
\item $2 H_{\ell-2} \cup F_{\ell-3}$ is a $(1+\epsilon)$-sparsifier for $H_{\ell-3}$
and so $4 F_{\ell-1} \cup 2 F_{\ell-2} \cup F_{\ell-3}$ is a $(1+\epsilon)^3$-sparsifier for $H_{\ell-3}$
\end{enumerate}
We continue in this way until we deduce $\bigcup_{i=0}^\ell 2^i \cdot F_i $ is a $(1+\epsilon)^\ell$-sparsifier for $H_0=G_0$.
\end{proof}

By re-parameterizing $\epsilon \leftarrow \epsilon/(2\ell)$ and using the sketches from Section \ref{sec:recon}, we establish the next theorem.
\begin{theorem}There exists a vertex-based sketch of size $\tilde{O}(\epsilon^{-2} n)$ from which we can construct a $(1+\epsilon)$ hypergraph sparsifier.\end{theorem}

%\section{Conclusions}

%\pparagraph{Acknowledgements.} We thank Jennifer Chayes for prompting us to investigate  hypergraph connectivity.
%{ 
%\bibliographystyle{abbrv} \bibliography{dynamic}
%}
%
%\appendix
%
%\section{Sketches for Constructing $k$ Skeletons}
%
%\pparagraph{Basic Non-Sketch Algorithm}
%The algorithm is based on the following simple $O(\log n)$ stage process. In the first stage, we find an arbitrary incident edge for each vertice. We then collapse each of the resulting connected components into a ``supernode". In each subsequent stage, we find an edge from every supervertex to another supervertex (if one exists) and collapse the connected components into new supervertices. It is not hard to argue that this process terminates after $O(\log n)$ stages and that the set of edges used to connect supervertices in the different stages include  a spanning forest of the graph. From this we can obviously deduce whether the graph is connected.
%
%The resulting algorithm for connectivity is relatively simple but makes use of linearity in an essential way:
%\begin{enumerate}
%\item {\em In a single pass, compute the sketches:} Choose $t=O(\log n)$ random strings $r_1, \ldots, r_t$ and construct the $\ell_0$-sampling projections $M_{r_j} \veca^i$ for $i\in [n]$, $j\in [t]$. Then,
%\[\alg_i(\vecf^{v_i})  = \left (M_{r_1} \veca^i \right ) \circ \left (M_{r_2} \veca^i \right ) \ldots \circ \left (M_{r_t} \veca^i \right ) \ .\]
%\item {\em In post-processing, emulate the original algorithm:}
%\begin{enumerate}
%\item Let $\hat{V}=V$ be the initial set of ``supervertices".
%\item For $i= 1, \ldots, t$: for each supervertex $S\in \hat{V}$, use $\sum_{i\in S} M_{r_j} \veca^i=M_{r_j}(\sum_{i\in S} \veca^i)$ to sample an edge between $S$ and another supervertice. Collapse the connected supervertices to form a new set of supervertices. 
%\end{enumerate}
%\end{enumerate}
%Since each sketch $\alg_i$ has dimension $O(\polylog n)$ and there are $n$ such sketches to be computed, the final connectivity algorithm uses $O(n \cdot \polylog n)$ space.
%

%\appendix


%In the centralized RAM model, there exist alternate algorithms provided by Eppstein et al \cite{EppsteinGIN97} that construct a $O(kn)$ size certificates in the edge insertion model only.  
%This algorithm accepts every incoming edge $(u,v)$ iff and only if there does not already exist $k$ vertex disjoint paths from $u$ to $v$. This theorem relies on a deep splitting off theorem of Mader to show that the number of edges never exceed $O(nk)$. However it is impossible to run the algorithm (meaningfully) in the dynamic setting where an edge used in the $k$ vertex disjoint paths from $u$ to $v$ may later be deleted, but we do not have have access to edge $(u,v)$ at that time. Therefore no previous algorithms answers $k$-vertex connectivity short of storing every edge in the graph in the dynamic setting.


%\section{Scan-First Trees}


%\end{document}

 % Introduction

\chapter{Mesh}
\label{chap:mesh}

\iffalse
%% For double-blind review submission, w/o CCS and ACM Reference (max submission space)
%\documentclass[sigplan,review,anonymous]{acmart}\settopmatter{printfolios=true,printccs=false,printacmref=false}
%\documentclass[sigplan]{acmart}\settopmatter{printfolios=true,printccs=false,printacmref=false}
%% For double-blind review submission, w/ CCS and ACM Reference
%\documentclass[sigplan,review,anonymous]{acmart}\settopmatter{printfolios=true}
%% For single-blind review submission, w/o CCS and ACM Reference (max submission space)
%\documentclass[sigplan,review]{acmart}\settopmatter{printfolios=true,printccs=false,printacmref=false}
%% For single-blind review submission, w/ CCS and ACM Reference
%\documentclass[sigplan,review]{acmart}\settopmatter{printfolios=true}
%% For final camera-ready submission, w/ required CCS and ACM Reference
\documentclass[sigplan,screen]{acmart}\settopmatter{}

\setcopyright{acmlicensed}
\acmPrice{15.00}
\acmDOI{10.1145/3314221.3314582}
\acmYear{2019}
\copyrightyear{2019}
\acmISBN{978-1-4503-6712-7/19/06}
\acmConference[PLDI '19]{Proceedings of the 40th ACM SIGPLAN Conference on Programming Language Design and Implementation}{June 22--26, 2019}{Phoenix, AZ, USA}
\acmBooktitle{Proceedings of the 40th ACM SIGPLAN Conference on Programming Language Design and Implementation (PLDI '19), June 22--26, 2019, Phoenix, AZ, USA}


%% \startPage{1}

%% Bibliography style
\bibliographystyle{ACM-Reference-Format}
%% Citation style
%\citestyle{acmauthoryear}  %% For author/year citations
%\citestyle{acmnumeric}     %% For numeric citations
%\setcitestyle{nosort}      %% With 'acmnumeric', to disable automatic
                            %% sorting of references within a single citation;
                            %% e.g., \cite{Smith99,Carpenter05,Baker12}
                            %% rendered as [14,5,2] rather than [2,5,14].
%\setcitesyle{nocompress}   %% With 'acmnumeric', to disable automatic
                            %% compression of sequential references within a
                            %% single citation;
                            %% e.g., \cite{Baker12,Baker14,Baker16}
                            %% rendered as [2,3,4] rather than [2-4].


%%%%%%%%%%%%%%%%%%%%%%%%%%%%%%%%%%%%%%%%%%%%%%%%%%%%%%%%%%%%%%%%%%%%%%
%% Note: Authors migrating a paper from traditional SIGPLAN
%% proceedings format to PACMPL format must update the
%% '\documentclass' and topmatter commands above; see
%% 'acmart-pacmpl-template.tex'.
%%%%%%%%%%%%%%%%%%%%%%%%%%%%%%%%%%%%%%%%%%%%%%%%%%%%%%%%%%%%%%%%%%%%%%

\pdfinfoomitdate=1
\pdftrailerid{}

\usepackage[utf8]{inputenc}
\usepackage[english]{babel}

%\usepackage{minted}
%\RequirePackage[outputdir=latex.out,cache=false]{minted}
\RequirePackage[cache=false]{minted}
\usepackage{microtype}
\usepackage{hyperref}
\usepackage{xspace}
\usepackage{graphicx}
\usepackage{url}
\usepackage{amsmath}
\usepackage{amssymb}
\usepackage{relsize}
\usepackage{tikz}
\usepackage{clrscode3e}
\usepackage{amsthm}
\usepackage{color}

\newtheorem{observation}{Observation}%[section]


\newcommand{\sm}{\proc{SplitMesher}\xspace}

%% \setmonofont[Scale=MatchLowercase]{DejaVu Sans Mono}
%% \setmonofont{DejaVu Sans Mono}

\microtypecontext{spacing=nonfrench}
\microtypesetup{
  final,
  tracking=true,
  kerning=true,
  spacing=true,
  factor=1100,
  stretch=20,
  shrink=20
}
\SetTracking{encoding={*}, shape=sc}{0} % No tracking for smallcaps

\newcommand{\Mesh}{\textsc{Mesh}\xspace}
\newcommand{\mesh}{Mesh\xspace}
\newcommand{\ndegree}{\textup{d}} % Command \deg already defined by some dependency
\newcommand{\comments}[1]{[#1]}


%% Some recommended packages.
\usepackage{booktabs}   %% For formal tables:
                        %% http://ctan.org/pkg/booktabs
%\usepackage{subcaption} %% For complex figures with subfigures/subcaptions
                        %% http://ctan.org/pkg/subcaption
%\captionsetup[subfloat]{font={small,sf}}
% Settings for figures and tables. Figure captions are placed below the figure,
% while table captions are placed above the table. All captions are sans-serif.
\RequirePackage[font={normalsize,sf,bf}]{caption}
\RequirePackage[position=bottom]{subfig}
\captionsetup[table]{aboveskip=0.5em, belowskip=0.5em}
\captionsetup[figure]{aboveskip=0.5em, belowskip=0em}
\captionsetup[subfloat]{font={small,sf}}
\setcounter{topnumber}{2}
\setcounter{dbltopnumber}{2}
\setcounter{bottomnumber}{2}
\setcounter{totalnumber}{4}
\renewcommand{\topfraction}{0.85}
\renewcommand{\dbltopfraction}{0.9}
\renewcommand{\bottomfraction}{0.85}
\renewcommand{\textfraction}{0.07}
\renewcommand{\floatpagefraction}{0.85}
\renewcommand{\dblfloatpagefraction}{0.85}

\setlength{\floatsep}{0.5em plus 0.2em minus 0.2em}
\setlength{\dblfloatsep}{0.5em plus 0.2em minus 0.2em}
\setlength{\textfloatsep}{0.5em plus 0.2em minus 0.2em}
\setlength{\dbltextfloatsep}{0.5em plus 0.2em minus 0.2em}

% Utility packages for floats and tables.
\RequirePackage{float}
\RequirePackage{graphicx}
\RequirePackage{booktabs}
\RequirePackage{multirow}



\begin{document}

%% Title information
\title[Mesh]{\Mesh: Compacting Memory Management \\ for C/C++ Applications} % Unmanaged Languages}


%% Author information NOOP
%% Contents and number of authors suppressed with 'anonymous'.
%% Each author should be introduced by \author, followed by
%% \authornote (optional), \orcid (optional), \affiliation, and
%% \email.
%% An author may have multiple affiliations and/or emails; repeat the
%% appropriate command.
%% Many elements are not rendered, but should be provided for metadata
%% extraction tools.

%% Author with single affiliation.
\author{Bobby Powers}
%\authornote{with author1 note}          %% \authornote is optional;
                                        %% can be repeated if necessary
%\orcid{nnnn-nnnn-nnnn-nnnn}             %% \orcid is optional
\affiliation{
  %\position{Position1}
  \department{College of Information and Computer Sciences}              %% \department is recommended
  \institution{University of Massachusetts Amherst}            %% \institution is required
  \streetaddress{140 Governors Drive}
  \city{Amherst}
  \state{MA}
  \postcode{01003}
  \country{USA}                    %% \country is recommended
}
\email{bpowers@cs.umass.edu}          %% \email is recommended


%% Author with single affiliation.
\author{David Tench}
%\authornote{with author1 note}          %% \authornote is optional;
                                        %% can be repeated if necessary
%\orcid{nnnn-nnnn-nnnn-nnnn}             %% \orcid is optional
\affiliation{
  %\position{Position1}
  \department{College of Information and Computer Sciences}              %% \department is recommended
  \institution{University of Massachusetts Amherst}            %% \institution is required
  \streetaddress{140 Governors Drive}
  \city{Amherst}
  \state{MA}
  \postcode{01003}
  \country{USA}                    %% \country is recommended
}
\email{dtench@cs.umass.edu}          %% \email is recommended



%% Author with single affiliation.
\author{Emery D. Berger}
%\authornote{with author1 note}          %% \authornote is optional;
                                        %% can be repeated if necessary
%\orcid{nnnn-nnnn-nnnn-nnnn}             %% \orcid is optional
\affiliation{
  %\position{Position1}
  \department{College of Information and Computer Sciences}              %% \department is recommended
  \institution{University of Massachusetts Amherst}            %% \institution is required
  \streetaddress{140 Governors Drive}
  \city{Amherst}
  \state{MA}
  \postcode{01003}
  \country{USA}                    %% \country is recommended
}
\email{emery@cs.umass.edu}          %% \email is recommended

%% Author with single affiliation.
\author{Andrew McGregor}
%\authornote{with author1 note}          %% \authornote is optional;
                                        %% can be repeated if necessary
%\orcid{nnnn-nnnn-nnnn-nnnn}             %% \orcid is optional
\affiliation{
  %\position{Position1}
  \department{College of Information and Computer Sciences}              %% \department is recommended
  \institution{University of Massachusetts Amherst}            %% \institution is required
  \streetaddress{140 Governors Drive}
  \city{Amherst}
  \state{MA}
  \postcode{01003}
  \country{USA}                    %% \country is recommended
}
\email{mcgregor@cs.umass.edu}          %% \email is recommended

\fi

\input{Chapters/mesh/abstract}

\iffalse
%% 2012 ACM Computing Classification System (CSS) concepts
%% Generate at 'http://dl.acm.org/ccs/ccs.cfm'.
\begin{CCSXML}
<ccs2012>
<concept>
<concept_id>10011007.10010940.10010941.10010949.10010950.10010953</concept_id>
<concept_desc>Software and its engineering~Allocation / deallocation strategies</concept_desc>
<concept_significance>500</concept_significance>
</concept>
</ccs2012>
\end{CCSXML}

\ccsdesc[500]{Software and its engineering~Allocation / deallocation strategies}
%% End of generated code


\keywords{Memory management, runtime systems, unmanaged languages}


%% \maketitle
%% Note: \maketitle command must come after title commands, author
%% commands, abstract environment, Computing Classification System
%% environment and commands, and keywords command.
\maketitle


\fi

% TODO: make sure to talk about virtual compaction from Hound in related work!
% TODO: Make repo public, put in redacted citation.
% TODO: Rust maybe later
% TODO: mention evaluation results





\section{Introduction}
\label{sec:introduction}

\begin{comment}
Memory consumption is a serious concern across the spectrum of modern
computing platforms, from mobile to desktop to datacenters. For
example, on low-end Android devices, Google reports that more than 99
percent of Chrome crashes are due to running out of memory when
attempting to display a web page~\cite{hara:stateofblink}. On
desktops, the Firefox web browser has been the subject of a five-year
effort to reduce its memory footprint~\cite{awsy}. In datacenters,
developers implement a range of techniques from custom allocators to
other \emph{ad hoc} approaches in an effort to increase memory
utilization~\cite{jemalloc:exposehints,redis:announcement}.
%% while
%% Google reports that 10\% of CPU cycles in their clusters are spent on
%% memory allocation~\cite{kanev:2015:warehouse-scale}.
\end{comment}



\begin{comment}
A key challenge is that, unlike in garbage-collected environments,
automatically reducing a C/C++ application's memory footprint
via compaction is not possible. Because the addresses of allocated
objects are directly exposed to programmers, C/C++ applications can
freely modify or hide addresses.  For example, a program may stash
addresses in integers, store flags in the low bits of aligned
addresses, perform arithmetic on addresses and later reference them,
or even store addresses to disk and later reload them.  This hostile
environment makes it impossible to safely relocate objects: if an
object is relocated, all pointers to its original location must be
updated. However, there is no way to safely update \emph{every}
reference when they are ambiguous, much less when they are absent.

Existing memory allocators for C/C++ employ a variety of
best-effort heuristics aimed at reducing average
fragmentation~\cite{johnstone:1998:fragmentation}. However, these
approaches are inherently limited. In a classic result, Robson showed
that all such allocators can suffer from catastrophic
memory fragmentation~\cite{robson:1977:worstcasefrag}. This increase
in memory consumption can be as high as the $\log$ of the ratio
between the largest and smallest object sizes allocated. For example,
for an application that allocates 16-byte and 128KB objects, it is
possible for it to consume $13\times$ more memory than required.
\end{comment}

%Embedded systems designed for the Internet-of-Things (IoT), such as the Raspberry Pi Zero W, ship with wireless networking, 3D graphics, and complete operating system stacks but only hundreds of megabytes of memory~\cite{rpi:zero}, placing memory at a premium.

%% TODO: talk about / cite Detlefs paper on precise GC for C/C++

%% address manipulations.  XOR encoding of linked lists

% These addresses can then be freely manipulated: it is legal for C and C++ programs to

% Here, we define fragmentation as the
%amount of memory actually consumed divided by the amount of memory
%actually needed (the total live size).

\begin{comment}
  The result is that C and C++ programs can suffer from fragmentation
(both internal and external). In a setting where compaction
is possible, fragmentation can always be eliminated by squeezing out
space between objects. Since relocating objects is impossible,
fragmentation in C and C++ can become problematic.
\end{comment}

\begin{comment}
  % Push to related work
In languages like LISP, Java and
~\cite{hansen:1969:compaction,fenichel:1969:compaction}, the runtime
system can, as part of garbage collection, periodically compact memory
by moving live objects together. Compaction reduces the working set of
an application and can ensure that an application's footprint never
exceeds some pre-established maximum. Contemporary runtimes like the
Hotspot JVM~\cite{microystems2006memory}, the .NET
VM~\cite{microsoft:dotnet-gc}, and the SpiderMonkey JavaScript
VM~\cite{mozilla:spidermonkey-compaction} implement compaction as part
of their garbage collection algorithms.
\end{comment}


% additionally: build invalid addresses but never dereference them

% emacs pickle state?

%For example, on modern
%systems with a 4 KiB page size and a 16-byte minimum object size, this
%factor corresponds to a worst-case fragmentation of approximately
%$7\times$. In practice, of course, fragmentation is rarely so extreme,
%b
%% TODO: Memory remains one of the scarcest resources across the spectrum of modern computing devices, ranging from servers to desktops to mobile platforms.

%% Not sure - For example, Google reports that \emph{99 percent} of Chrome crashes on low-end Android devices are caused by it running out of memory when attempting to display the page~\cite{hara:stateofblink}.

%% TODO: cost of memory on Amazon servers!

Despite nearly fifty years of conventional wisdom indicating that
compaction is impossible in unmanaged languages, this chapter shows that
it is not only possible but also practical. It introduces
\Mesh, a memory allocator that effectively and efficiently performs
compacting memory management to reduce memory usage in unmodified
C and C++ applications.

Crucially and counterintuitively, \Mesh performs compaction without
relocation; that is, without changing the addresses of objects. This
property is vital for compatibility with arbitrary C/C++
applications. To achieve this, \Mesh{} builds on a mechanism which we
call \emph{meshing}, first introduced by Novark et al.'s Hound memory
leak detector~\cite{1542521}. Hound employed meshing in an effort to avoid
catastrophic memory consumption induced by its memory-inefficient
allocation scheme, which can only reclaim memory when every object on
a page is freed. Hound first searches for pages whose live objects do
not overlap. It then copies the contents of one page onto the other,
remaps one of the \emph{virtual} pages to point to the single
\emph{physical} page now holding the contents of both pages, and
finally relinquishes the other physical page to the
OS. Figure~\ref{fig:meshing} illustrates meshing in action.

\begin{figure*}[t!]
  \centering
  \subfloat[\textbf{Before:} these pages are candidates for
      ``meshing'' because their allocated objects do not
      overlap.\vspace{2em}]{
      \includegraphics[width=0.45\textwidth]{Chapters/mesh/figures/mesh-diagram-1}
      \label{pre-meshing}
  }
  ~~~~~
  \centering
  \subfloat[\textbf{After:} both virtual pages now point to the
      first physical page; the second page is now freed.]{
      \includegraphics[width=0.45\textwidth]{Chapters/mesh/figures/mesh-diagram-2}
      \label{post-meshing}
  }

  \caption{\textbf{\Mesh{} in action.} \Mesh{} employs novel
    randomized algorithms that let it efficiently find and then
    ``mesh'' candidate pages within \emph{spans} (contiguous 4K pages)
    whose contents do not overlap.  In this example, it increases
    memory utilization across these pages from 37.5\% to 75\%, and
    returns one physical page to the OS (via \texttt{munmap}),
    reducing the overall memory footprint. \Mesh{}'s randomized
    allocation algorithm ensures meshing's effectiveness with high
    probability.}

  \label{fig:meshing}
\end{figure*}

\Mesh{} overcomes two key technical challenges of meshing that previously made
it both inefficient and potentially entirely ineffective. First,
Hound's search for pages to mesh involves a linear scan of pages on
calls to \texttt{free}. While this search is more efficient than a
naive $O(n^2)$ search of all possible pairs of pages, it remains
prohibitively expensive for use in the context of a general-purpose
allocator. Second, Hound offers no guarantees that \emph{any} pages
would ever be meshable.  Consider an application that happens to
allocate even one object in the same offset in every page. That layout
would preclude meshing altogether, eliminating the possibility of
saving any space.

\Mesh makes meshing both efficient and provably effective (with high
probability) by combining it with two novel randomized
algorithms. First, \Mesh uses a space-efficient randomized
allocation strategy that effectively scatters objects within each
virtual page, making the above scenario provably exceedingly
unlikely. Second, \Mesh incorporates an efficient randomized
algorithm that is guaranteed with high probability to quickly find
candidate pages that are likely to mesh. These two algorithms work in
concert to enable formal guarantees on \Mesh's effectiveness. Our
analysis shows that \Mesh breaks the above-mentioned Robson worst
case bounds for fragmentation with high
probability~\cite{robson:1977:worstcasefrag}, as memory reclaimed by meshing is available for use by any size class.
This ability to redistribute memory from one size class to another enables Mesh to adapt to changes in an application's allocation behavior in a way other segregated-fit allocators cannot.


We implement \Mesh as a library for C/C++ applications running on
Linux or Mac OS X. \Mesh{} interposes on memory management operations,
making it possible to use it without code changes or
recompilation by setting the appropriate environment variable to load
the \Mesh{} library (e.g., \texttt{export
  LD\_PRELOAD=libmesh.so} on Linux). Our evaluation demonstrates that
our implementation of \Mesh{} is both fast and efficient in
practice. It generally matches the performance of state-of-the-art
allocators while guaranteeing the absence of catastrophic
fragmentation with high probability. In addition, it occasionally
yields substantial space savings: replacing the standard allocator
with \Mesh{} automatically reduces memory consumption by 16\%
(Firefox) to 39\% (Redis).
%In
%general, the longer-lived and more memory-intensive the application,
%the more memory \Mesh can save.


\subsection{Contributions}
\label{sec:contributions}

This chapter describes the \Mesh system, focusing on its core meshing algorithm.  It presents theoretical results that guarantee \Mesh{}'s efficiency and effectiveness with high probability (\S\ref{sec:theory}).  Other components of the \Mesh system design and empirical evaluation of its performance are briefly summarized to contextualize the algorithm and analysis.

\begin{comment}
\begin{itemize}

\item It introduces \textbf{\Mesh}, a novel memory allocator that acts
  as a plug-in replacement for \texttt{malloc}. \Mesh{} combines
  remapping of virtual to physical pages (meshing) with randomized
  allocation and search algorithms to enable safe and effective
  \emph{compaction without relocation} for C/C++
  (\S\ref{sec:meshing}, \S\ref{sec:algorithms},
  \S\ref{sec:allocator}).

\item It presents theoretical results that guarantee \Mesh{}'s
    efficiency and effectiveness with high probability (\S\ref{sec:theory}).

\item It evaluates \Mesh{}'s performance empirically, demonstrating \Mesh{}'s ability to reduce
    space consumption while generally imposing low runtime
    overhead (\S\ref{sec:evaluation}).

\end{itemize}
\end{comment}

\section{Overview}
\label{sec:meshing}

\begin{comment}
  \begin{figure}[!t]
\centering
\includegraphics[width=.3\textwidth]{figures/bitmap_bitstring}
\caption{The bitmaps managing the allocated space in a span
  (visualized as allocated objects in the span, top) can be
  represented as bitstrings of 0s and 1s (bottom), where a 1
  corresponds to an allocated object and 0 to free space.}
\label{fig:bitmap-bitstring}
\end{figure}
\end{comment}

This section provides a high-level overview of how \Mesh{} works and
gives some intuition as to how its algorithms and implementation
ensure its efficiency and effectiveness, before diving into detailed
description of \Mesh{}'s algorithms (\S\ref{sec:algorithms}),
implementation (\S\ref{sec:allocator}), and its theoretical analysis
(\S\ref{sec:theory}).


\subsection{Remapping Virtual Pages}

\Mesh{} enables compaction without relocating object addresses; it
depends only on hardware-level virtual memory support, which is
standard on most computing platforms like x86 and ARM64. \Mesh{} works
by finding pairs of pages and merging them together \emph{physically}
but not \emph{virtually}: this merging lets it relinquish
physical pages to the OS.

Meshing is only possible when no objects on the pages occupy the same
offsets.  A key observation is that as fragmentation increases (that
is, as there are more free objects), the likelihood of successfully
finding pairs of pages that mesh also increases.

Figure~\ref{fig:meshing} schematically illustrates the meshing
process. \Mesh{} manages memory at the granularity of \textit{spans},
which are runs of contiguous 4K pages (for purposes of illustration,
the figure shows single-page spans). Each span only contains
same-sized objects. The figure shows two spans of memory with low
utilization (each is under $40\%$ occupied) and whose allocations are at non-overlapping offsets.


\begin{comment}

More formally, we can say that $t$ spans
\textit{\mesh} if and only if the sum of each offset across the $t$
spans sum to 1 or less:

\begin{align}
  \forall k \in [0, b-1]. \sum_{0 \leq i \leq t} s_i[k] \leq 1
\end{align}

where $b$ is span length, and $s_i[k] = 1$ iff there is an object at the $k$th offset of span i.
\end{comment}

Meshing consolidates allocations from each span onto one physical span.
Each object in the resulting meshed span resides at the same offset as
it did in its original span; that is, its virtual addresses are
preserved, making meshing invisible to the application. Meshing then
updates the virtual-to-physical mapping (the page tables) for the
process so that both virtual spans point to the same physical
span. The second physical span is returned to the OS.  When average
occupancy is low, meshing can consolidate many pages, offering the
potential for considerable space savings.
% consolidated to a single span and all other spans can be reused.


\subsection{Random Allocation}

A key threat to meshing is that pages could contain objects at the
same offset, preventing them from being meshed. In the worst case, all
spans would have only one allocated object, each at the same offset,
making them non-meshable. \Mesh{} employs randomized allocation to
make this worst-case behavior exceedingly unlikely. It allocates
objects uniformly at random across all available offsets in a span. As
a result, the probability that all objects will occupy the same offset
is $\left({1}/{b}\right)^{n-1}$, where $b$ is the number of objects in
a span, and $n$ is the number of spans.

%Meshing employs random allocation to minimize the likelihood of this happening too often. The use of a randomized algorithm lets us reason about the effectiveness of meshing in expectation.

%Meshing introduces a limited amount of randomness to ensure that we
%can reason about meshing spans in expectation.

In practice, the resulting probability of being unable to mesh many
pages is vanishingly small. For example, when meshing 64 spans with
one 16-byte object allocated on each (so that the number of objects
$b$ in a 4K span is $256$), the likelihood of being unable to mesh any
of these spans is $10^{-152}$. To put this into perspective, there are
estimated to be roughly $10^{82}$ particles in the universe.

We use randomness to guide the design of \Mesh{}'s algorithms
(\S\ref{sec:algorithms}) and implementation (\S\ref{sec:allocator});
this randomization lets us prove robust guarantees of its performance
(\S\ref{sec:theory}), showing that \Mesh{} breaks the Robson bounds with
high probability.

% ,
%subject to an assumption that application allocation behavior does not
%depend on the addresses returned by the allocator.

%% TODO: should insert a treatment of Emery's ideas about address obliviousness somewhere.


\subsection{Finding Spans to Mesh}

Given a set of spans, our goal is to mesh them in a way that frees as
many physical pages as possible. We can think of this task as that of
partitioning the spans into subsets such that the spans in each subset
mesh. An optimal partition would minimize the number of such subsets.

%% TODO: fix ref to point at exact subsection in algorithms.

Unfortunately, as we show, optimal meshing is not feasible
(\S\ref{sec:theory}). Instead, the algorithms in
Section~\ref{sec:algorithms} present practical methods for finding
high-quality meshes under real-world time constraints. We show that
solving a simplified version of the problem (\S\ref{sec:algorithms})
is sufficient to achieve reasonable meshes with high probability
(\S\ref{sec:theory}).


\begin{comment}
Memory fragmentation, introduced in Section~\ref{sec:introduction},
occurs when the ratio of program-allocated memory to operating-system
reserved memory becomes high:

\begin{align*}
\text{Fragmentation} = \frac{\text{Reserved}}{\text{Allocated}}
\end{align*}

\end{comment}

%%\textit{XXX: I think it is worth noting that an allocator that stored a very high amount of metadata per allocation (e.g. tracking the   meshability graph explicitly) would be indistinguishable from a   highly-fragmented allocator from the perspective of the OS.  Not   sure how to work that in.}

%% Informally, fragmentation occurs because the application manages
%% memory in bytes, while the operating system manages memory at page
%% granularity -- 4 KiB on most current architectures.  Sparsely allocated
%% application addresses ``pin down'' otherwise vacant OS pages.

%% Many managed languages like Java and JavaScript do not experience
%% memory fragmentation due to the combination of language soundness and
%% garbage collection implementations.  Languages are designed in such a
%% way that implementations can enumerate and update all object
%% references, enabling garbage collectors to periodically compact live
%% objects.  This isn't possible for unmanaged languages like C, C++ and
%% Rust, as it is not possible to soundly enumerate all object
%% references.


%% XXX: we should unify talking about ``objects'' and ``allocations''
%% -- allocations is probably better for unmanaged languages where
%% objects have a specific interpretation.

%% Meshing works on \textit{spans} -- contiguous regions of memory where
%% the size of a span is a multiple of the page size between 4 KiB and
%% 128 KiB.  Each span allocates objects of a single-size only, for
%% example a 4 KiB span might hold 32 objects of size 128 bytes.  We can
%% represent a span by a \textit{bitstring}, as in
%% Figure~\ref{fig:bitmap-bitstring}, a string with a 1 for an allocated
%% object at that offset from the start of the span and 0 otherwise.  The
%% length of the bitstring is the number of objects that span holds.

%% This definition characterizes the constraints of the technique by
%% which meshing is possible, wherein two or more spans are meshed or
%% "stacked" on top of each other.

%% The layout and management of a program's heap guide how we consider
%% meshing.  In a running program, the heap is managed as a number of
%% different \textit{size classes} along with a region consisting of
%% large allocations.  Allocations are fulfilled from the smallest size
%% class they fit in (e.g. an allocation request for 50 bytes is
%% satisfied by the 64-byte size class), and objects larger than 16 KiB
%% are individually served from the large allocation region.

%% We treat each size class as an independent instance of the meshing
%% problem, and large allocations are not meshed.  As large allocations
%% are all many multiples of the page size significant fragmentation
%% between them does not exist.  The number of size-classes is fixed at
%% compilation time and constant during the execution of a program.

%% From here, we consider meshing as dealing with a single size-class,
%% and refer to all spans within this size class as $S$.  If we want to
%% mesh the entire heap, this means solving $n$ instances of the meshing
%% problem, where $n$ is the number of size classes.


%% Finally, meshing relies on the fact that there are two types of spans,
%% virtual and physical.  A \textit{virtual} span refers to the memory
%% addresses visible to the program being executed, while a
%% \textit{physical} span corresponds to the area in memory where
%% allocated objects live.  Meshing is concerned with minimizing the
%% count of in-use physical spans without modifying or moving virtual
%% spans.  As noted in Section~\ref{sec:introduction}, we cannot change
%% or modify virtual addresses returned from the allocator, as we do not
%% have a way to enumerate and update all references the program has
%% stored.  Since the last two digits of a virtual address directly
%% specify the offset of the referenced object in the span, objects
%% cannot be safely relocated to a different offset, leading to our

%% Allocated objects and free space within a span are tracked by the
%% memory allocator as a bitmap (see Section~\ref{sec:allocator}) -- the
%% in-memory representation of a bitstring.  For example, for objects of
%% size 32 and a span size of 4 KiB, the span can hold 128 32-byte
%% objects, so allocated objects are be tracked with a 128-bit bitmap.
%% Each allocated object in a running program has a unique
%% \textit{(bitmap, offset)} tuple.  Bitmaps are between 8 and 256-bits
%% in length.

%% We can therefore think of the meshing problem as that of partitioning
%% a set of equal-length bitstrings such that the bitstrings in each subset
%% mesh.  An optimal partition would minimize the number of such subsets.
%% This abstraction allows us to analyze the complexity of this computational
%% task in Section~\ref{sec:theory}.


\section{Algorithms \& System Design}
\label{sec:algorithms}

%This section provides an overview of \Mesh{}'s algorithms.
%Section~\ref{sec:allocator} provides a detailed description of \Mesh's
%implementation, while Section~\ref{sec:theory} presents a theoretical
%analysis of its effectiveness.

\Mesh{} comprises three main algorithmic components: allocation
(\S\ref{sec:allocation-algorithm}), deallocation
(\S\ref{sec:deallocation-algorithm}), and finding spans to mesh
(\S\ref{sec:meshing-algorithm}). Unless otherwise noted and without
loss of generality, all algorithms described here are per size class
(within spans, all objects are same size).

\subsection{Allocation}
\label{sec:allocation-algorithm}
% We want out allocator to enforce random spread of objects so that we can probabilistically guarantee meshing.  Also we want it to reuse memory when available.

Allocation in \Mesh{} consists of two steps: (1) finding a span to
allocate from, and (2) randomly allocating an object from that span.
\Mesh{} always allocates from a thread-local shuffle vector -- a
randomized version of a freelist %(described in detail in\S\ref{sec:shuffle-freelists}). 
The shuffle vector contains offsets
corresponding to the slots of a single span.  We call that span the
\emph{attached} span for a given thread.

If the shuffle vector is empty, \Mesh relinquishes the current
thread's attached span (if one exists) to the \emph{global heap}
(which holds all unattached spans), and asks it to select a new
span. If there are no partially full spans, the global heap returns a
new, empty span.  Otherwise, it selects a partially full span for
reuse. To maximize utilization, the global heap groups spans into bins
organized by decreasing occupancy (e.g., 75-99\% full in one bin,
50-74\% in the next). The global heap scans for the first non-empty
bin (by decreasing occupancy), and randomly selects a span from that
bin.

Once a span has been selected, the allocator adds the offsets
corresponding to the free slots in that span to the thread-local
shuffle vector (in a random order). \Mesh{} pops the first entry off
the shuffle vector and returns it.


\subsection{Deallocation}
\label{sec:deallocation-algorithm}

Deallocation behaves differently depending on whether the free is
local (the address belongs to the current thread's attached span),
remote (the object belongs to another thread's attached span), or if
it belongs to the global heap.

For local frees, \Mesh{} adds the object's offset onto the span's
shuffle vector in a random position and returns. For remote frees,
\Mesh{} atomically resets the bit in the corresponding index in a
bitmap associated with each span. Finally, for an object belonging to
the global heap, \Mesh{} marks the object as free, updates the span's
occupancy bin; this action may additionally trigger meshing.
%, which we
%describe below.

%% Global frees similarly mark the space backing an allocation as free and open for reuse in the owning span, but additionally may trigger meshing.  Meshing happens after a global free if it has been more than a set period of time since the last meshing (settable both at program startup and later by the program through the \texttt{mallctl} API, by default a tenth of a second, and if the allocator thinks there is a reasonable chance of meshing reclaiming space.

%% TODO: free -- right now, we assume unlimited virtual addresses; max limit of ranges in kernel (discuss in implementation)


\subsection{Meshing}
\label{sec:meshing-algorithm}

When meshing, \Mesh{} randomly chooses pairs of spans and attempts to
mesh each pair. The meshing algorithm, which we call \sm
(Figure~\ref{fig:meshalg}), is designed both for practical
effectiveness and for its theoretical guarantees.  The parameter $t$,
which determines the maximum number of times each span is probed (line
\ref{li:outerloop}), enables space-time trade-offs. The parameter $t$
can be increased to improve mesh quality and therefore reduce space,
or decreased to improve runtime, at the cost of sacrificed meshing
opportunities. We empirically found that $t=64$ balances runtime and
meshing effectiveness, and use this value in our implementation.

\sm proceeds by iterating through $S_l$ and checking whether it can
mesh each span with another span chosen from $S_r$ (line
\ref{li:condition}).  If so, it removes these spans from their
respective lists and meshes them (lines
\ref{li:remove}--\ref{li:mesh}). \sm repeats until it has checked $t *
|S_l|$ pairs of spans; \S\ref{sec:meshing-implementation} describes
the implementation of \sm in detail.

\begin{comment}
  At this point, each span has been checked exactly once.  All spans not
removed from the lists are still candidates for meshing.  \sm loops
again through the lists, in the $j$th step comparing the $j$th span in
$S_l$ with the $j+1$th span in $S_r$.  Again, whenever it compares a
pair of spans that can mesh, it removes them from their respective
lists and meshes them.

\sm repeats this looping process $t$ times.  On the $i$th loop through
the lists, the $j$th element in $S_l$ is compared with the $j+i-1$th
element in $S_r$.  After this outer loop (line~\ref{li:outerloop}) has
completed, any remaining span has been compared with $t$ other spans,
each time failing to mesh. \sm stops here and makes no further effort
to mesh these spans.
\end{comment}


\iffalse
\begin{algorithm}[tb]
  \LinesNumbered
  \SetKwData{Frontier}{frontier}\SetKwData{This}{this}\SetKwData{OldNode}{$n_{old}$}\SetKwData{NewNode}{$n_{new}$}
  \SetKw{Continue}{continue} \SetKw{And}{and}
  \SetKwData{Null}{null}
  \SetKwData{Path}{path}
  \SetKwData{OldEdges}{$E_{old}$}
  \SetKwData{NewEdges}{$E_{new}$}
  \SetKwData{OldEdge}{$e_{old}$}
  \SetKwData{NewEdge}{$e_{new}$}
  \SetKwData{Length}{Length}
  \SetKwData{True}{true} \SetKwData{False}{false}
  \SetKwFunction{MakeTuple}{MakeTuple}
  \SetKwFunction{GetRoot}{GetRoot}\SetKwFunction{Enqueue}{enqueue}\SetKwFunction{Length}{length}\SetKwFunction{Dequeue}{dequeue}
  \SetKwFunction{HasVisited}{HasVisited}
  \SetKwFunction{MarkAsVisited}{MarkAsVisited}
  \SetKwFunction{MarkAsGrowing}{MarkAsGrowing}
  \SetKwFunction{MarkNotGrowing}{MarkNotGrowing}
  \SetKwFunction{RecordGrowingPath}{RecordGrowingPath}
  \SetKwFunction{IsGrowing}{IsGrowing}
  \SetKwInOut{Input}{Input}\SetKwInOut{Output}{Output}
  \Input{$G_{final} = (N, E)$}
  \Frontier$\leftarrow$ []\;
  $n_{root} \leftarrow$ \GetRoot{$G_{final}$}\;
  \For{$e = (n_1, n2) \in E : n_1 = n_{root}$}{
    \Frontier.\Enqueue{\MakeTuple{\Null, $e$}}
  }
  \While{\Frontier.\Length $\neq 0$}{
    $T \leftarrow$ \Frontier.\Dequeue{}\;
    $(T_{prev}, e) \leftarrow T$\;
    \eIf{\HasVisited{$e$}}{
      \Continue\;
    }{
      \MarkAsVisited{$e$}\;
    }
    $(n_{from}, n_{to}) \leftarrow e$\;

    \If{\IsGrowing{$n_{to}$}}{
      \RecordGrowingPath{$T$}\;
    }
    \ForEach{$e=(n_1,n_2) \in E : n_1 = n_{to}$}{
      \Frontier.\Enqueue{\MakeTuple{$T$, $e$}}\;
    }
  }
  \caption{FindLeakPaths, which records paths through the heap to leaking nodes. RecordGrowingPath recovers the entire path to the edge $e$ by following the linked list formed by the tuple $T$.}
  \label{algo:findpaths}
\end{algorithm}
\fi

\iffalse
\begin{algorithm}[tb]
  \SetAlgorithmName{\sm}{}{}
  \SetKwInOut{Input}{Input}
  \SetKwInOut{Output}{Output}
  \Input{randomly ordered list of spans S, $|S| = n$; parameter $t$}

  M $ \leftarrow$ []\;

  $S_l,$ $S_r = S[1:\frac{n}{2}],$ $S[\frac{n}{2}+1 : n]$\;
  \ForEach{$i \in [0, t-1]$}{
    len = length$(S_l)$\;
    \ForEach{$j \in [0, \frac{n}{2}-1]$}{
      \If{$S_l(j)$ and $S_r(j+i$ mod len) mesh}{
        $S_l \leftarrow S_l \textbackslash S_l(j)$\;
        $S_r \leftarrow S_r \textbackslash S_r(j+i$ mod len)\;
        M $\leftarrow$ M $\cup$ $S_l(j)$ $\cup$ $S_r(j+i$ mod len)\;
      }
    }
  }
  \Return M\;
  \TitleOfAlgo{Procedure splits span set into halves and probes for meshable pairs between halves.}
%  \label{alg:mesher}
\end{algorithm}
\fi

\begin{figure}[!t]
\begin{codebox}
    \Procname{$\sm(S,t)$}
%    \li M $\gets$ []
    \li $n \gets$ length$(S)$
    \li $S_l,$ $S_r \gets S[1:n/2],$ $S[n/2 +1 : n]$
    \li \For $(i = 0, i<t, i++)$ \label{li:outerloop}
        \li \Do
            $\mbox{len} = |S_l|$\;
            \li \For $(j = 0, j < \mbox{len}, j++)$ \label{li:innerloop}
                \li \Do
                    \If \proc{Meshable} $(S_l(j)$, $S_r(j+i$ \% $\mbox{len}$)) \Then \label{li:condition}
                        \li $S_l \leftarrow S_l \setminus S_l(j)$ \label{li:remove}
                        \li $S_r \leftarrow S_r \setminus S_r(j+i$ \% $\mbox{len}$)
%                        \li M $\leftarrow$ M $\cup$ $S_l(j)$ $\cup$ $S_r(j+i$ mod len)
                        \li \proc{mesh}($ S_l(j)$, $S_r(j+i$ \% $\mbox{len}$)) \label{li:mesh}
                    \End
                \End
        \End
%    \li \Return M
\end{codebox}
\caption{\textbf{Meshing random pairs of spans.} \sm splits the randomly ordered span list $S$ into halves, then probes pairs between halves for meshes.  Each span is probed up to $t$ times.}
\label{fig:meshalg}
\end{figure}


\subsection{Implementation}
\label{sec:allocator}

We implement \Mesh as a drop-in replacement memory allocator that
implements meshing for single or multi-threaded applications written
in C/C++. Its current implementation work for 64-bit Linux and Mac OS
X binaries. \Mesh can be explicitly linked against by passing
\texttt{-lmesh} to the linker at compile time, or loaded dynamically
by setting the \texttt{LD\_PRELOAD} (Linux) or
\texttt{DYLD\_INSERT\_LIBRARIES} (Mac OS X) environment variables to
point to the \Mesh{} library. When loaded, \Mesh interposes on
standard libc functions to replace all memory allocation functions.

\begin{comment}
\begin{figure}
  \includegraphics[width=.5\textwidth]{figures/global_heap}
  \caption{\textbf{Heap organization.} \Mesh's internal allocator
    manages memory for \Mesh-internal dynamic data structures.
    The global heap manages both large objects and the meshable arena that spans are allocated from.  Each thread has a
    local heap which satisfies small allocations from MiniHeaps (one
    per size class).}
  \label{fig:global-heap}
\end{figure}
\end{comment}

\Mesh combines traditional allocation strategies with meshing to
minimize heap usage.  Like most modern memory
allocators~\cite{Novark:2010:DSH:1866307.1866371,1134000,379232,evans2006scalable,ghemawattcmalloc},
\Mesh is a segregated-fit allocator. \Mesh{} employs fine-grained size
classes to reduce internal fragmentation due to rounding up to the
nearest size class. \Mesh{} uses the same size classes as those
used by jemalloc for objects 1024 bytes and
smaller~\cite{evans2006scalable}, and power-of-two size classes for
objects between 1024 and 16K.  Allocations are fulfilled from the
smallest size class they fit in (e.g., objects of size 33--48 bytes
are served from the 48-byte size class); objects larger than 16K are
individually fulfilled from the global arena.  Small objects are
allocated out of \textit{spans} (\S\ref{sec:meshing}), which are
multiples of the page size and contain between 8 and 256 objects of a
fixed size.  Having at least eight objects per span helps
amortize the cost of reserving memory from the global manager for
the current thread's allocator.

Objects of 4KB and larger are always page-aligned and span at least one
entire page. \Mesh does not consider these objects for meshing;
instead, the pages are directly freed to the OS.

\Mesh's heap organization consists of four main components.
\emph{MiniHeaps} track occupancy and other metadata for spans.
%(\S\ref{sec:miniheaps}).  
\textit{Shuffle vectors} enable efficient,
random allocation out of a MiniHeap. %(\S\ref{sec:shuffle-freelists}).
\textit{Thread local heaps} satisfy small-object allocation requests
without the need for locks or atomic operations in the common case.
%(\S\ref{sec:thread-local-heaps}). 
Finally, the \textit{global heap}
%(\S\ref{sec:global-heap})
 manages runtime state shared by all threads,
large object allocation, and coordinates meshing operations.
%(\S\ref{sec:meshing-implementation}).
We omit detailing discussion of these components of \Mesh in this document.


\iffalse

\subsection{MiniHeaps}
\label{sec:miniheaps}

MiniHeaps manage allocated physical spans of memory and are either
\emph{attached} or \emph{detached}.  An attached MiniHeap is owned by
a specific thread-local heap, while a detached MiniHeap is only
referenced through the global heap.  New small objects are
\textit{only} allocated out of attached MiniHeaps.

Each MiniHeap contains metadata that comprises span length, object
size, allocation bitmap, and the start addresses of any virtual spans
meshed to a unique physical span.  The number of objects that can be
allocated from a MiniHeap bitmap is \textit{objectCount = spanSize /
  objSize}.  The allocation bitmap is initialized to
\textit{objectCount} zero bits.

When a MiniHeap is attached to a
thread-local \emph{shuffle vector} (\S\ref{sec:shuffle-freelists}),
each offset that is unset in the MiniHeap's bitmap is added to the
shuffle vector, with that bit now atomically set to one in the bitmap.
This approach is designed to allow multiple threads to free objects
which keeping most memory allocation operations local in the common
case.

When an object is freed and the free is non-local
(\S\ref{sec:deallocation-algorithm}), the bit is reset.  When a new
MiniHeap is allocated, there is only one virtual span that points to
the physical memory it manages. After meshing, there may be multiple
virtual spans pointing to the MiniHeap's physical memory.


\begin{figure}[!t]
  \centering
  \subfloat[A shuffle vector for a span of size 8, where no objects have
      yet been allocated.]{
      \includegraphics[width=.4\textwidth]{Chapters/mesh/figures/shuffle-freelist_a}
  }
  \vspace{-0em}
  \subfloat[The shuffle vector after the first object has been allocated.]{
      \includegraphics[width=.4\textwidth]{Chapters/mesh/figures/shuffle-freelist_b}
  }
  \vspace{-0em}
  \subfloat[On \texttt{free}, the object's offset is pushed onto the
      front of the vector, the allocation index is updated, and the
      offset is swapped with a randomly chosen offset.]{
      \includegraphics[width=.4\textwidth]{Chapters/mesh/figures/shuffle-freelist_c}
  }
  \vspace{-0em}
  \subfloat[Finally, after the swap, new allocations proceed in a
      bump-pointer like fashion.]{
    \centering
    \includegraphics[width=.4\textwidth]{Chapters/mesh/figures/shuffle-freelist_d}
  }
  \vspace{0.5em}
  \caption{\textbf{Shuffle vectors} compactly enable fast random allocation.
    Indices (one byte each) are maintained in random order; allocation is
    popping, and deallocation is pushing plus a random swap (\S\ref{sec:shuffle-freelists}).}
    % When an object is freed and the MiniHeap     it was allocated from is still attached to the local thread,    its
  \label{fig:shuffle-freelists}
\end{figure}


%% TODO: restore locality section if and when we have locality results
\begin{comment}
\subsubsection{Locality}

At first blush, it may appear that \Mesh's use of randomization would
degrade locality and thus degrade performance. Like DieHard (but unlike
many other allocators), \Mesh does not seek to allocate consecutive
objects nearby. However, we expect this approach to have minimal
impact on locality in many cases. First, the increasing size of cache
lines means that inter-object locality is not a concern for most
objects. Modern 64-bit Intel processors have 64-byte cache lines, and
ARM devices have 32 or 64-byte cache lines, meaning that object
locality has little impact on L1 cache locality for objects of that
size or larger. We also expect TLB pressure to be unaffected compared
to other segregated-fit allocators with the same choice of size
classes; \Mesh allocates randomly \emph{within} spans. % \textbf{XXX forward ref to evaluation; we want to say primary cost is allocation mechanism, not locality effects.}
\end{comment}

\subsection{Shuffle Vectors}
\label{sec:shuffle-freelists}

Shuffle vectors are a novel data structure that lets \Mesh
perform randomized allocation out of a MiniHeap efficiently and
with low space overhead.

Previous memory allocators that have employed randomization (for
security or reliability) perform randomized allocation by random
probing into
bitmaps~\cite{1134000,Novark:2010:DSH:1866307.1866371}. In these
allocators, a memory allocation request chooses a random number in the
range $[0,\text{\textit{objectCount}}-1]$. If the associated bit is
zero in the bitmap, the allocator sets it to one and returns the
address of the corresponding offset. If the offset is already one,
meaning that the object is in use, a new random number is chosen and
the process repeated. Random probing allocates objects in $O(1)$
\emph{expected} time but requires overprovisioning memory by a
constant factor (e.g., $2\times$ more memory must be allocated than
needed). This overprovisioning is at odds with our goal of
\emph{reducing} space overhead.

Shuffle vectors solve this problem, combining low space overhead with
worst-case $O(1)$ running time for \texttt{malloc} and
\texttt{free}. Each comprises a fixed-size array consisting of all the
offsets from a span that are \textit{not} already allocated, and an
allocation index representing the head. Each vector is initially
randomized with the Knuth-Fischer-Yates
shuffle~\cite{knuth:1981:semi}, and its allocation index is set to
0. Allocation proceeds by selecting the next available number in the
vector, ``bumping'' the allocation index and returning the
corresponding address. Deallocation works by placing the freed object
at the front of the vector and performing one iteration of the shuffle
algorithm; this operation preserves randomization of the
vector. Figure~\ref{fig:shuffle-freelists} illustrates this process,
while Figure~\ref{fig:malloc} has pseudocode listings for
initialization, allocation, and deallocation.

%\vspace{-1.2em}
%\begin{align*}
 % \text{\textit{ptr}} = \text{\textit{spanStart + off*objSize}}
%\end{align*}

%This freelist is initialized with all of the available offsets $[0
%  .. $ objectCount $]$ and then sorted

%When an object is freed and the owning MiniHeap is still in use, a
%single iteration of the shuffle is performed to re-add the newly freed
%offset back to the list of available offsets.

Shuffle vectors impose far less space overhead than random
probing. First, with a maximum of 256 objects in a span, each offset
in the vector can be represented as an unsigned character (a single
byte). Second, because \Mesh needs only one shuffle vector per
attached MiniHeap, the amount of memory required for vectors is
$256c$, where $c$ is the number of size classes (24 in the current
implementation): roughly 2.8K per thread.  Finally, shuffle vectors
are only ever accessed from a single thread, and so do not require
locks or atomic operations.  While bitmaps must be operated on
atomically (frees may originate at any time from other threads),
shuffle vectors are only accessed from a single thread and do not
require synchronization or cache-line flushes.

\subsection{Thread Local Heaps}
\label{sec:thread-local-heaps}

All malloc and free requests from an application start at the thread's
local heap. Thread local heaps have shuffle vectors for each size
class, a reference to the global heap, and their own thread-local
random number generator.

Allocation requests are handled differently depending on the size of
the allocation.  If an allocation request is larger than 16K, it is
forwarded to the global heap for fulfillment
(\S\ref{sec:global-heap}).  Allocation requests 16K and smaller are
small object allocations and are handled directly by the shuffle
vector for the size class corresponding to the allocation request, as
in Figure~\ref{fig:malloc}a.  If the shuffle vector is empty, it is
refilled by requesting an appropriately sized MiniHeap from the global
heap.  This MiniHeap is a partially-full MiniHeap if one exists, or
represents a freshly-allocated span if no partially full ones are
available for reuse.  Frees, as in Figure~\ref{fig:malloc}d, first
check if the object is from an attached MiniHeap.  If so, it is
handled by the appropriate shuffle vector, otherwise it is passed to
the global heap to handle.

\subsection{Global Heap}
\label{sec:global-heap}

The global heap allocates MiniHeaps for thread-local heaps, handles
all large object allocations, performs non-local frees for both small
and large objects, and coordinates meshing.

\subsubsection{The Meshable Arena}
\label{sec:meshable-arena}

The global heap allocates meshable spans and large objects from a
single, global meshable arena. This arena contains two sets of bins
for same-length spans --- one set is for demand zero-ed spans (freshly
\texttt{mmap}ped), and the other for used spans --- and a mapping of
page offsets from the start of the arena to their owning MiniHeap
pointers.  Used pages are not immediately returned to the OS as they
are likely to be needed again soon, and reclamation is relatively
expensive. Only after 64MB of used pages have accumulated, or whenever
meshing is invoked, \Mesh{} returns pages to OS by calling
\texttt{fallocate} on the heap's file descriptor
(\S\ref{sec:page-table-updates}) with the
\texttt{FALLOC\_FL\_PUNCH\_HOLE} flag.

\begin{comment}
An allocation request for $k$ pages is fulfilled by searching through
the bins of free spans from index $k-1$ to $k_{\text{max}}$ for a free
span of at-least size $k$.  Bins smaller than $k_{\text{max}}$ only
contain spans of a single size, while $k_{\text{max}}$ contains spans
of length 256 or larger.  If no spans are available, the arena is
extended, the extension placed in $k_{\text{max}}$, and bins are
re-searched.  Once a resulting span has been found, if the span is
larger than the $k$ pages requested it is broken in two with the
remainder span placed back in an appropriate bin.  This scheme favors
dirty pages over clean pages to minimize the process's RSS.
\end{comment}

% TODO: mention how this is also how aligned allocations work

\begin{figure}[!t]
  \centering
  \subfloat{\lstinputlisting[language=C++, basicstyle=\footnotesize]{Chapters/mesh/all-code1}}
  \centering
  \subfloat{\lstinputlisting[language=C++, basicstyle=\footnotesize]{Chapters/mesh/all-code2}}
  %\input{Chapters/mesh/all-code1}
  %\input{Chapters/mesh/all-code2}
%  \input{./local-malloc.cc}
%  \input{./freelist-attach.cc}
%  \input{./miniheap-malloc.cc}
%  \input{./local-free.cc}
%  \input{./miniheap-free.cc}
  \caption{Pseudocode for \Mesh's core allocation and deallocation routines.}
  \label{fig:malloc}
\end{figure}



\subsubsection{MiniHeap Allocation}

Allocating a MiniHeap of size $k$ pages begins with requesting $k$
pages from the meshable arena.  The global allocator then allocates
and initializes a new MiniHeap instance from an internal allocator
that \Mesh uses for its own needs. This MiniHeap is kept live so long
as the number of allocated objects remains non-zero, and singleton
MiniHeaps are used to account for large object allocations.  Finally,
the global allocator updates the mapping of offsets to MiniHeaps for
each of the $k$ pages to point at the address of the new MiniHeap.

\subsubsection{Large Objects}

All large allocation requests (greater than 16K) are directly
handled by the global heap. Large allocation requests are rounded up
to the nearest multiple of the hardware page size (4K on x86\_64),
and a MiniHeap for 1 object of that size is requested, as detailed
above.  The start of the span tracked by that MiniHeap is returned to
the program as the result of the malloc call.

\subsubsection{Non-local Frees}

If \texttt{free} is called on a pointer that is not contained in an
attached MiniHeap for that thread, the free is handled by the global
heap.  Non-local frees occur when the thread that frees the object is
different from the thread that allocated it, or if there have been
sufficient allocations on the current thread that the original
MiniHeap was exhaused and a new MiniHeap for that size class was
attached.

Looking up the owning MiniHeap for a pointer is a constant time
operation. The pointer is checked to ensure it falls within the arena,
the arena start address is subtracted from it, and the result is
divided by the page size.  The resulting offset is then used to index
into a table of MiniHeap pointers. If the result is zero, the pointer
is invalid; otherwise, it points to a live MiniHeap.  This enables us
to catch certain application-level memory management errors, like
non-local double frees when a new object hasn't been allocated at the
same address.

Once the owning MiniHeap has been found, that MiniHeap's bitmap is
updated atomically in a compare-and-set loop.  If a free occurs for an
object where the owning MiniHeap is attached to a different thread,
the free atomically updates that MiniHeap's bitmap, but does not
update the other thread's corresponding shuffle vector.


\subsection{Meshing}
\label{sec:meshing-implementation}

\Mesh's implementation of meshing is guided by theoretical results
(described in detail in Section~\ref{sec:theory}) that enable it to
efficiently find a number of spans that can be meshed.

Meshing is rate limited by a configurable parameter, settable at
program startup and during runtime by the application through the
semi-standard \texttt{mallctl} API.  The default rate meshes at most
once every tenth of a second.  If the last meshing freed less than one
MB of heap space, the timer is not restarted until a subsequent
allocation is freed through the global heap.  This approach ensures that \Mesh
does not waste time searching for meshes when the application and heap
are in a steady state.

We implement the \sm algorithm from Section~\ref{sec:algorithms} in
C++ to find meshes.  Meshing proceeds one size class at a time.  Pairs
of mesh candidates found by \sm are recorded in a list, and after \sm
returns candidate pairs are meshed together \emph{en masse}.

Meshing spans together is a two step process. First, \Mesh{}
consolidates objects onto a single physical span. This consolidation
is straightforward: \Mesh{} copies objects from one span into the free
space of the other span, and updates MiniHeap metadata (like the
allocation bitmap).  Importantly, as \Mesh copies data at the physical
span layer, even though objects are moving in memory, no pointers or
data internal to moved objects or external references need to be
updated. Finally, \Mesh{} updates the process's virtual-to-physical mappings
to point all meshed virtual spans at the consolidated physical
span.

Physical memory reclaimed from meshing in one size class is able to be
used to satisfy future allocations in other size classes.

\subsubsection{Page Table Updates}
\label{sec:page-table-updates}

\Mesh updates the process's page tables via calls to \texttt{mmap}.
We exploit the fact that \texttt{mmap} lets the same offset in a file
(corresponding to a physical span) be mapped to multiple
addresses. \Mesh's arena, rather than being an anonymous mapping, as
in traditional \texttt{malloc} implementations, is instead a shared mapping
backed by a temporary file. This temporary file is obtained via the
\texttt{memfd\_create} system call and only exists in memory or on
swap.


\subsubsection{Concurrent Meshing}

Meshing takes place concurrently with the normal execution of other
program threads with \textit{no} stop-the-world phase required.  This
is similar to how concurrent relocation is implemented in low-latency
garbage collector algorithms like Pauseless and
C4~\cite{click:2005:pauseless, tene:2011:c4}, as described below.
\Mesh maintains two invariants throughout the meshing process: reads
of objects being relocated are always correct and available to
concurrently executing threads, and objects are never written to while
being relocated between physical spans.  The first invariant is maintained
through the atomic semantics of \texttt{mmap}, the second through a
write barrier.

\Mesh's write barrier is implemented with page protections and a
segfault trap handler.  Before relocating objects, \Mesh calls
\texttt{mprotect} to mark the virtual page where objects are being
copied from as read-only.  Concurrent reads succeed as normal.  If a
concurrent thread tries to write to an object being relocated, a
\Mesh-controlled segfault signal handler is invoked by a combination
of the hardware and operating system.  This handler waits on a lock
for the current meshing operation to complete, the last step of which
is remapping the source virtual span as read/write.  Once meshing is
done the handler checks if the address that triggered the segfault was
involved in a meshing operation; if so, the handler exits and the
instruction causing the write is re-executed by the CPU as normal
against the fully relocated object.


\subsubsection{Concurrent Allocation}

All thread-local allocation (on threads other than the one running
\sm) can proceed concurrently and independently with meshing, until
and unless a thread needs a fresh span to allocate from.  Allocation
only is performed from spans owned by a thread, and only spans owned
by the global manager are considered for meshing; spans have a single
owner.  The thread running \sm holds the global heap's lock while
meshing.  This lock also synchronizes transferring ownership of a span
from the global heap to a thread-local heap (or vice-versa).  If
another thread requires a new span to fulfill an allocation request,
the thread waits until the global manager finishes meshing and
releases the lock.

\subsection{Huge Pages}

Mesh's heap is not designed to be used in conjunction with transparent
huge pages, where the page table size used by the kernel and hardware
is 2MB rather than 4KB and the kernel runs a garbage collection-like
daemon to coalesce 4KB pages into 2MB pages.  Huge pages reduce TLB
pressure, but necessarily increase the granularity at which the kernel
manages physical memory on behalf of the process. This coarse
granularity is fundamentally at odds with Mesh's focus on minimizing
heap size.  Additionally, the \texttt{madvise} mechanism that Mesh and
other allocators like jemalloc use to return memory to the OS
interacts poorly with transparent huge pages on Linux (causing 2MB
pages to be split into 4KB pages), to the extent that major software
vendors and operators recommend disabling transparent huge pages
altogether~\cite{cloudera:thb,redis:thb,mongodb:thb,oracle:thb,nelson:thb}.
Applications that need to back datasets or data structures with huge
pages can still directly allocate (non-Mesh-managed) memory from Linux
using one of several interfaces~\cite{lwn:hp-interfaces}.

%% Mainstream CPUs have incorporated hierarchical TLBs for 4 KB pages,
%% reducing the performance impact of TLB
%% pressure~\cite{anandtech:nehelem-tlb,vtune:page-walk}.  Additionally,
%% with transparent huge pages on Linux, there is a high cost in finding
%% pages to physically relocate, in order to convert contiguous chunks of
%% memory from traditional to huge pages.  Many databases and server
%% workloads like redis~\cite{redis:thb}, MongoDB~\cite{mongodb:thb}, and
%% Oracle~\cite{oracle:thb} recommend disabling transparent huge pages.


\fi

\section{Analysis}
\label{sec:theory}


%\newenvironment{claim}[1]{\par\noindent\underline{Claim:}\space#1}{}
\newenvironment{claimproof}[1]{\par\noindent\underline{Proof:}\space#1}{\hfill $\blacksquare$}

%\newtheorem{problem}{Problem}

%% \theoremstyle{definition}
%% \newtheorem{definition}{Definition}[section]

%% \theoremstyle{theorem}
%% \newtheorem{theorem}{Theorem}[section]

%% \theoremstyle{lemma}
%% \newtheorem{lemma}{Lemma}[section]

\newcommand{\bigo}{\mathcal{O}}
\newcommand{\page}{\pi}
\newcommand{\str}{s}
\newcommand{\node}{\mathit {v}}
\newcommand{\W}{\mathcal {W}}
\newcommand{\lp}{\left(}
%\newcommand{\rparen}{\right)}
\newcommand{\rparen}{\right)}
\newcommand{\rp}{\right)}

%%caveat required for lemma 4.4 added below.  run by andrew before finalizing
This section shows that the \sm procedure described in
\S\ref{sec:meshing-algorithm} comes with strong formal guarantees on
the \textit{quality} of the meshing found along with bounds on its
\textit{runtime}.  In situations where significant meshing
opportunities exist (that is, when compaction is most desirable), \sm
finds with high probability an approximation arbitrarily close to
$1/2$ of the best possible meshing in $O\lp n/q\rparen$ time, where
$n$ is the number of spans and $q$ is the global probability of two
spans meshing.

To formally establish these bounds on quality and runtime, we show
that meshing can be interpreted as a graph problem, analyze its
complexity (\S\ref{subsec:graph}), show that we can do nearly as well
by solving an easier graph problem instead (\S\ref{subsec:matching}),
and prove that \sm approximates this problem with high probability
(\S\ref{subsec:analysis}).

% \item{Introduce an easily-computable \emph{a priori} lower bound for the effect of meshing (\S\ref{subsec:lowerbound})}
%\end{itemize}

\begin{figure}[!t]
  \centering
  \includegraphics[width=.8\textwidth]{Chapters/mesh/figures/graph-diagram.pdf}
%  \includegraphics[width=.33\textwidth]{figures/meshing_graph.pdf}
%  \vspace{-2em}
%\centering
\caption{\textbf{An example meshing graph.}  Nodes correspond to the spans represented by the strings \texttt{01101000}, \texttt{01010000}, \texttt{00100110}, and \texttt{00010000}.  Edges connect meshable strings (corresponding to non-overlapping spans).}\label{fig:exmesh}
\end{figure}


\subsection{Formal Problem Definitions}
\label{subsec:probdef}
Since \Mesh{} segregates objects based on size, we can limit our
analysis to compaction within a single size class without loss of
generality. For our analysis, we represent spans as binary strings of
length $b$, the maximum number of objects that the span can
store. Each bit represents the allocation state of a single object. We
represent each span $\page$ with string $\str$ such that $\str\lp
i\rparen = 1$ if $\page$ has an object at offset $i$, and 0 otherwise.

\begin{definition}
We say  two  strings $\str_{1}, \str_{2}$ \em{mesh} iff $\sum_i \str_{1}\lp i\rparen \cdot \str_{2} \lp i \rparen = 0$. More generally, a set of binary strings are said to mesh if every pair of strings in this set mesh.
%We say that k binary strings $\str_{1}, ... \str_{k}$ mesh iff $\sum_{j \in [1,k]} \str_{j}(i) \leq 1$ $\forall i \in [0,b-1]$.\\
%Equivalently, k binary strings mesh if each each of said strings are pairwise meshable with all of the other strings.
\end{definition}

When we mesh $k$ spans together, the objects scattered across those
$k$ spans are moved to a single span while retaining their offset from
the start of the span. The remaining $k-1$ spans are no longer needed
and are released to the operating system. We say that we ``release''
$k-1$ \emph{strings} when we mesh $k$ strings together.  Since our
goal is to empty as many physical spans as possible, we can
characterize our theoretical problem as follows:

\begin{problem}
Given a multi-set of $n$ binary strings of length $b$, find a meshing
that releases the maximum number of strings.
\end{problem}

% Note that the total number of strings released is equal to $n - \rho -
% \phi$, where $\rho$ is the number of total meshes performed, and $\phi$
% is the number of strings that remain unmeshed.

\paragraph*{A Formulation via Graphs:}
\label{subsec:graph}

We observe that an instance of the meshing problem, a string multi-set
$S$, can naturally be expressed via a graph $G(S)$ where there is a
node for every string in $S$ and an edge between two nodes iff the
relevant strings can be meshed. Figure~\ref{fig:exmesh} illustrates
this representation via an example.

%Given , we construct a meshing graph $G_S$ as follows:\\
%We add a node $\node$ to $G_S$ for each $\str \in S$.  We say node $\node$ represents $\str$.  For each %node pair $\node_1, \node_2$, we add edge $(\node_1, \node_2)$ iff $\node_1$ and $\node_2$ mesh.

%This meshing problem is reducible to min clique cover (MCC), an NP-Complete problem.  We implicitly show %this reduction through a straightforward graph interpretation of the meshing problem.

If a set of strings are meshable, then there is an edge between every
pair of the corresponding nodes: the set of corresponding nodes is a
\emph{clique}. We can therefore decompose the graph into $k$ disjoint
cliques iff we can free $n-k$ strings in the meshing
problem. Unfortunately, the problem of decomposing a graph into the
minimum number of disjoint cliques (\textsc{MinCliqueCover}) is in
general NP-hard. Worse, it cannot even be approximated up to a factor
$m^{1-\epsilon}$ unless $P=NP$~\cite{zuckerman07}.

While the meshing problem is reducible to {\textsc{MinCliqueCover}},
we have not shown that the meshing problem is NP-Hard.  The meshing
problem is indeed NP-hard for strings of arbitrary length, but in
practice string length is proportional to span size, which is
constant.

\begin{theorem}\label{thm:polytime}
The meshing problem for $S$, a multi-set of strings of constant length, is in $P$. %can be solved in polynomial time.
%is in P
%; equivalently, the min clique cover problem on meshing graphs generated from sets of strings of constant %length is in P.
\end{theorem}

\begin{proof}
We assume without loss of generality that $S$ does not contain the
all-zero string $\str_0$; if it does, since $\str_0$ can be meshed
with any other string and so can always be released, we can solve the
meshing problem for $S \setminus \str_0$ and then mesh each instance
of $\str_0$ arbitrarily.

Rather than reason about {\textsc{MinCliqueCover}} on a
 meshing graph $G$, we consider the equivalent
problem of coloring the complement graph $\bar{G}$ in which there is
an edge between every pair of two nodes whose strings do not mesh. The nodes of $\bar{G}$ can be partitioned into at most $2^b-1$
subsets $N_1 \ldots N_{2^b-1}$ such that all nodes in each
$N_i$ represent the same string $\str_i$.  The induced subgraph of
$N_i$ in $\bar{G}$ is a clique since all its nodes have a 1 in the same position and so cannot be pairwise meshed.  Further, all nodes in $N_i$ have the same set of neighbors.

Since $N_i$ is a clique, at most one node in $N_i$ may be colored with
any color.  Fix some coloring on $\bar{G}$.  Swapping the colors of
two nodes in $N_i$ does not change the validity of the coloring since
these nodes have the same neighbor set.  We can therefore
unambiguously represent a valid coloring of $\bar{G}$ merely by
indicating in which cliques each color appears.

With $2^b$ cliques and a maximum of $n$ colors, there are at most $\lp
n+1 \rparen ^{c}$ such colorings on the graph where $c=2^{2^b}$. This follows because each color used can be associated with a subset of $\{1, \ldots, 2^b\}$ corresponding to which of the cliques have node with this color; we call this subset a \emph{signature} and note there are $c$ possible signatures. A coloring can be therefore be associated with a multi-set of possible signatures where each signature has multiplicity between 0 and $n$; there are $(n+1)^c$ such multi-sets. This is polynomial in $n$ since $b$ is constant and hence $c$ is also constant. So we can simply check each coloring for validity (a coloring is valid iff no color appears in two cliques whose string representations mesh). The algorithm returns a valid coloring with the lowest number of colors from all valid colorings discovered.
\end{proof}

%Unfortunately, the exponent of $n$ runtime grows doubly exponentially in  $b$ and $b$ can be as large as 256.
%Such a runtime,
Note that the runtime of the above algorithm is at least exponential in the string length. While technically polynomial for constant string length, the running time of the above algorithm would obviously be prohibitive in practice and so we never employ it in \Mesh{}. Fortunately, as we show next, we can exploit the randomness in the strings to design a much faster algorithm.

\subsection{Simplifying the Problem: From \textsc{MinCliqueCover} to \textsc{Matching}}
\label{subsec:matching}
We leverage \Mesh{}'s random allocation to simplify meshing; this random
allocation implies a distribution over the graphs that exhibits useful
structural properties. We first make the following important observation:

\begin{observation}
Conditioned on the occupancies of the strings, edges in the meshing graph  are not three-wise independent.
\end{observation}

To see that edges are not three-wise independent consider three random
strings $s_1, s_2, s_3$ of length 4, each with exactly 2 ones. It is
impossible for these strings to all mesh mutually since if we know that $s_1$ and $s_2$ mesh,
and that $s_2$ and $s_3$ mesh, we know for certain that $s_1$ and
$s_3$ cannot mesh. More generally, conditioning on $s_1$ and $s_2$
meshing and $s_1$ and $s_3$ meshing decreases the probability that
$s_1$ and $s_3$ mesh.
%Note that even when we assume bits are 1 or 0
%independently, edges are still not independent.
%The existence of edge
%($s_4,s_5$) is weak evidence of $s_5$'s low occupancy, increasing the
%probability that edge ($s_5,s_6$) exists.
Below, we quantify this
effect to argue that we can mesh near-optimally by solving the much
easier \textsc{Matching} problem on the meshing graph (i.e.,
restricting our attention to finding cliques of size 2) instead of
\textsc{MinCliqueCover}. Another consequence of the above observation is that we will not be able to appeal to  theoretical results on the standard model of random graphs, \emph{Erd\H{o}s-Renyi graphs}, in which  each possible edge is present with some fixed probability and the edges are fully independent. Instead we will need new algorithms and proofs that only require independence of acyclic collections of edges.

\subsubsection{Triangles and Larger Cliques are Uncommon.}
Because of the dependencies across the edges present in a meshing
graph, we can argue that \emph{triangles} (and hence also larger
cliques) are relatively infrequent in the graph and certainly less
frequent than one would expect were all edges independent.  For
example, consider three strings $s_1, s_2, s_3\in \{0,1\}^b$ with
occupancies $r_1, r_2,$ and $r_3$, respectively. The probability they
mesh is
\[
{\binom{b-r_1}{r_2}} \big / {\binom{b}{r_2}} \times {\binom{b-r_{1}-r_2 }{r_3}} \big / {\binom{b}{r_3}} \ . \]

This value is significantly less than would have been the case if the
events corresponding to pairs of strings being meshable were
independent.
%In most practical cases, we will not be interested in meshing strings with low occupancy --- it is often better just to wait until %the last objects on these spans are freed.  For higher occupancies, the expected number of triangles is quite low.
For instance, if $b = 32, r_1=r_2=r_3 = 10$, this probability is so
low that even if there were $1000$ strings, the expected number of
triangles would be less than 2. In contrast, had all meshes been
independent, with the same parameters, there would have been $167$ triangles.

The above analysis suggests that we can focus on finding only cliques
of size 2, thereby solving \textsc{Matching} instead of
\textsc{MinCliqueCover}. The evaluation in
Section~\ref{subsec:exp} vindicates this approach, and we show a
strong accuracy guarantee for \textsc{Matching} in Section~\ref{subsec:analysis}.
% We also verify this empirically in the Appendix.

\subsection{Experimental Confirmation of Maximum Matching/Min Clique Cover Convergence}
\label{subsec:exp}

In Section~\ref{subsec:matching}, we argue that we can approximate the solution to \textsc{MinCliqueCover} on meshing graphs with high probability by instead solving \textsc{MaximumMatching}.

We experimentally verify this result by generating many random constant occupancy graphs and, for each graph, comparing the size of the maximum matching to the size of a greedy (non-optimal) solution for \textsc{MinCliqueCover}.  The results are summarized in Figure~\ref{plot:const}.

\begin{figure}[h]
\includegraphics[scale = 1]{Chapters/mesh/figures/const_match_comp.pdf}
\centering
\caption{\textbf{Min Clique Cover and Max Matching solutions converge.} The average size of Min Clique Cover and Max Matching for randomly generated constant occupancy meshing graphs, plotted against span occupancy.  Note that for sufficiently high-occupancy spans, Min Clique Cover and Max Matching are nearly equal.}
\label{plot:const}
\end{figure}



When we instead assume bits are 1 independently with probability p, we expect the graph to have many more triangles.  For $p = r/b = 10/32, n = 1000$, the expected number of triangles is roughly 36,000.  However, we can see experimentally that these graphs behave quite similarly in Figure~\ref{plot:indep}.


\begin{figure}[h]
\includegraphics[scale = 1]{Chapters/mesh/figures/ind_match_comp.pdf}
\centering
\caption{\textbf{Converge still holds for independent bits assumption.} The average size of Min Clique Cover and Max Matching for randomly generated constant occupancy meshing graphs, plotted against span occupancy.  Note that for sufficiently high-occupancy spans, Min Clique Cover and Max Matching are nearly equal.}
\label{plot:indep}
\end{figure}

While constant occupancy graphs are fairly regular, independent bit graphs may not be.  Since strings have different occupancies, nodes which correspond to strings with relatively low occupancy will tend to have significantly higher degree than other nodes in the graph.  Meanwhile, other nodes may have strings with high occupancy, and therefore only have a few edges (probably with low-occupancy nodes).  So while there are many triangles, when the graph is sparse enough, even meshing cliques of size 3 and 4 will likely "abandon" adjacent high-occupancy nodes, one of which could have been matched with the high degree node to yield the same number of releases.

So in this case we still expect finding the maximum matching to be good enough.
% \paragraph{We Can Restrict Our Attention to Matching.}
\begin{comment}
The above analysis leads us to conclude that, since there are
relatively few triangles in meshing graphs, we can achieve almost the
full benefits of solving \textsc{MinCliqueCover} by only solving
\textsc{Matching}.
\end{comment}


\subsection{Theoretical Guarantees}
\label{subsec:analysis}
Since we need to perform meshing at runtime, it is essential that our
algorithm for finding strings to mesh be as efficient as possible. It
would be far too costly in both time and memory overhead to actually
construct the meshing graph and run an existing matching algorithm on
it. Instead, the \sm algorithm (shown in Figure \ref{fig:meshalg})
performs meshing without the need for explicitly constructing the
meshing graph.

For further efficiency, we need to constrain the value of the
parameter $t$, which controls \Mesh{}'s space-time tradeoff. If $t$
were set as large as $n$, then \sm could, in the worst case,
exhaustively search all pairs of spans between the left and right
sets: a total of $n^2/4$ probes.  In practice, we want to choose a
significantly smaller value for $t$ so that \Mesh{} can always
complete the meshing process quickly without the need to search all
possible pairs of strings.


\begin{lemma}
Let $t=k/q$ where $k>1$ is some user defined parameter and $q$ is the global probability of two
spans meshing. \sm finds a matching
of size at least $n(1-e^{-2k})/4$ between the left and right span sets
with probability approaching 1 as $n\geq 2k/q$ grows.
\end{lemma}

\begin{proof}
%\sm in essence creates a bipartite meshing graph by splitting the span set in half.
Let $S_l=\{v_1, v_2, \ldots v_{n/2}\}$ and $S_r=\{u_1,
u_2, \ldots u_{n/2}\}$. Let $t=k/q$ where
$k>1$ is some arbitrary constant. For $u_i\in S_l$ and $i \leq j \leq
j+t$, we say $(u_i,v_j)$ is a \emph{good match} if all the following
properties hold: (1) there is an edge between $u_i$ and $v_j$, (2)
there are no edges between $u_i$ and $v_{j'}$ for $i\leq j'<j$, and
(3) there are no edges between $u_{i'}$ and $v_{j}$ for $i< i'\leq j$.

We observe that \sm finds any good match, although it may
also find additional matches. It therefore suffices to consider only
the number of good matches. The probability $(u_i,v_j)$ is a good
match is $q(1-q)^{2(j-i)}$ by appealing to the fact that the collection of edges under consideration is acyclic. Hence, $\Pr(u_i \mbox{ has a good match})$
is
\begin{align*}
r:= q \sum_{i=0}^{k/q-1} \lp 1-q\rparen ^{2i} = q \frac{1-(1-q)^{2k/q}}{1-(1-q)^2}
% \geq q \frac{1-e^{-2k}}{2q-q^2}
 > \frac{1-e^{-2k}}{2} \ .
\end{align*}

To analyze the number of good matches, define $X_i = 1$ iff $u_i$ has
a good match. Then, $\sum_i X_i$ is the number of good matches. By
linearity of expectation, the expected number of good matches is
$rn/2$. We decompose $\sum_i X_i$ into \[Z_0+Z_1+\ldots + Z_{t-1} ~~\mbox{
  where }~~ Z_{j} = \sum_{i\equiv j \bmod t} X_i \ .\] Since each
$Z_j$ is a sum of $n/(2t)$ independent variables, by the Chernoff
bound,\\ $P\lp Z_j < \lp 1-\epsilon \rparen E[Z_j]\rparen \leq \exp\lp -
\epsilon^2 r n/(4t)\rparen$.  By the union bound,
$$P\lp X < \lp 1-\epsilon \rparen rn/2\rparen \leq t \exp\lp - \epsilon^2 r
n/(4t)\rparen$$ and this becomes arbitrarily small as $n$ grows.
\end{proof}

In the worst case, the algorithm checks $nk/2q$ pairs. For our
implementation of \Mesh, we use a static value of $t = 64$; this value
enables the guarantees of Lemma 5.1 in cases where significant meshing
is possible.  As Section~\ref{sec:evaluation} shows, this value for
$t$ results in effective memory compaction with modest performance
overhead.




\subsection{New Lower Bound for Maximum Matching Size}
\label{subsec:lowerbound}

In this section, we develop a bound for the size of the maximum matching in a graph that can easily be estimated in the context of meshing graphs.  As meshing may be costly to perform, this lower bound is useful as it can be used to predict the magnitude of compaction achievable before committing to the process.  In the case where little compaction is possible, it is often better not to try to mesh and instead conserve resources for other tasks. The quantity we introduce will always lower bound the size of  the maximum matching and will typically be relatively close to the size of the maximum matching.  For example, if we want to release 30\% of our active spans through meshing, but the bound suggests a release of less than 5\% is possible, we can infer that the maximum matching on the graph is small and meshing is currently not worth attempting.

Our approach is based on extending a result by McGregor and Vorotnikova \cite{McGregorV16}. Let $\ndegree(u)$ be the degree of node $u$ in a graph. They considered the quantity
$\sum_{e\in E} 1/\max(\ndegree(u),\ndegree(v))$
and showed that it is at most a factor $3/2$ larger than the maximum matching in the graph and at most a factor $4$ smaller in the case of planar graphs. These bounds were tight.  For example, on a complete graph on three nodes, the quantity is $3/2$ while $M$ is 1. Meshing graphs are very unlikely to be planar but are likely be almost regular, i.e., most degrees are roughly similar.
We need to extend the above bound such that we can guarantee that it never exceeds the size of the maximum matching while also being a good estimate for the graphs that are likely to arise as meshing graphs.

%\begin{theorem}
%Define $W(G)$ to be a fractional matching on graph $G = (V,E)$ such that $$W = \sum_{(u,v) \in E} \min\lp %\frac{1}{\ndegree\lp u\rparen },\frac{1}{\ndegree\lp v\rparen }\rparen $$
%Then $W\leq \frac{3}{2} M$ where $M$ is the cardinality of the maximum matching on $G$.
%If $G$ is bipartite, then $W \leq M$.
%\end{theorem}

%For general graphs, $W$ is not always a lower bound on the maximum matching (hence the need for the $\frac{3}{2}$ factor).  For example, on a complete graph on three nodes, $W$ is $\frac{3}{2}$ while $M$ is 1.

%\begin{figure}[h]
%\includegraphics[scale = .7]{figures/triangle_matching.png}
%\centering
%\caption{W is not a lower bound for the maximum matching on a triangle.}
%\end{figure}
One simple approach is to scale the above quantity by a factor of $2/3$,
but this can result in a poor approximation for the size of the
maximum matching for some graphs of interest. Instead, we take a more
nuanced approach. Specifically, we prove the following theorem (proof omitted
due to space constraints):
%in the
%Appendix:

%While it is tight for this example, for larger non-bipartite graphs it seems like scaling $W$ down by %$\frac{2}{3}$ is overkill.  For instance, for a complete graph on an odd number of nodes k, $W = %{{k}\choose{2}} \frac{1}{k-1} = \frac{k}{2}$, while $M = \frac{k-1}{2}$.
%The following theorem provides a tight lower bound for this example.


\begin{theorem}
\[
\W=\sum_{e\in E} \frac{1}{\max(\ndegree(u), \ndegree(v))+I[\min(\ndegree(u),\ndegree(v))>1]}\leq M \ .
\]
%where $f(1,\cdot)=f(\cdot,1)=f(2,3)=f(3,2)=0$ and $f(\cdot,\cdot)=1$ otherwise. Then, $\sum_{e\in E} w_e \leq M$.
% . Then, $\sum_{e\in E} w_e \leq M$.
\end{theorem}

\begin{proof}
We begin by showing that the simpler quantity $W$ is a lower bound of the maximum matching $M$.
\[
W = \sum_{e\in E} \frac{1}{\max(\ndegree\lp u\rparen, \ndegree\lp v\rparen )+1}\ .
\]
%For each edge $e = (u,v) \in G$, define $\W$$(e)=min\lp \frac{1}{\ndegree\lp u\rparen +1},\frac{1}{\ndegree\lp %v\rparen +1}\rparen$.  L
Let $U$ be an arbitrary set of $t$ nodes in $G$ where $t$ is odd.   Define $$W(U) = \sum_{u,v \in U} \frac{1}{\max(\ndegree\lp u\rparen, \ndegree\lp v\rparen )+1} \ . $$ As a corollary of Edmonds Matching Polytope Theorem, it can be shown that $W \leq M$ if  $W(U) \leq (|U|-1)/2$. We can argue this as follows:

\begin{align*}
W(U) &= \sum_{(u,v) \in U} \min \left(\frac{1}{\ndegree(u)+1},\frac{1}{\ndegree(v)+1}\right )\\
& \leq \sum_{(u,v) \in U}\frac{1}{2}\lp \frac{1}{\ndegree\lp u \rparen +1}+\frac{1}{\ndegree\lp v\rparen +1}\rparen\\
&\leq \sum_{(u,v) \in U}\frac{1}{2}\lp \frac{1}{\ndegree_U\lp u\rparen +1}+\frac{1}{\ndegree_U\lp v\rparen +1}\rparen\\
&= \frac{1}{2} \sum_{u \in U} \frac{\ndegree_U\lp u\rp }{\ndegree_U\lp u\rp +1} \leq  \frac{1}{2} \lp \frac{t-1}{t}\rp  t = \frac{t-1}{2}
\end{align*}
where $\ndegree_U(u)$ in the number of neighbors of $u$ in the set $U$.
The second line follows from the fact that the minimum of two quantities is bounded above by their average.  The third line follows from the fact that the degree of any node in a subgraph is bounded above by its degree in the original graph.  The fourth line follows from summing over nodes instead of edges, and then reasoning that in the worst case U is a clique and so $\degree_U(u) = t-1$ for all u $u$.

\iffalse
Recall that we prove in Section~\ref{subsec:lowerbound} that when
\[
w_e=\min\lp \frac{1}{\ndegree\lp u\rp+1 },\frac{1}{\ndegree\lp v\rp+1 }\rp
~~\mbox{ and }~~\W=\sum_{e\in E} w_e \ .\]
then $\W$ is a lower bound on the cardinality of the maximum matching M(G).
\fi


In some cases $W$ is too conservative; it assigns little weight to edges which it could safely have assigned much more.  For example, if $e$ is isolated (meaning its endpoints have degree 1), $W(e) = 1/2$.  However, it is always safe to assign weight 1 to $e$, since $M(G-e) = M(G)-1.$  If we modified our rule for $W$ so that for any edge $e= (u,v)$ s.t. $\deg(u)=\deg(v) = 1$ we assigned weight $ \min(1/\deg(u),1/\deg(v))$ instead of $ \min(1/\deg(u)+1,1/\deg(v)+1)$, we would always assign weight 1 to isolated edges.

In fact, a more general rule is true.  For any edge $e= (u,v)$, if either $\deg(u) or \deg(v) = 1$ then we may assign it weight $\min(1/\deg(u),1/\deg(v))$.

\iffalse
At this point it is natural to ask whether we can remove the $+1$s from the denominator for other edges.  We cannot of course do so for $\deg(u) = \deg(v) = k >1, k \in 2\mathbb{Z}$ since for a clique on k+1 nodes this rule would result in total weight $(k+1)/2$.\\

However, we can remove the $+1$ under certain circumstances. We show that if one of the endpoints has degree one then this is the case.
\fi

%prove several rules of this form.  For each rule we specify values of $\deg(u)$ and $\deg(v)$ for which the %$+1$ can be removed from the denominator of the edge weight.
%First we restate the claim that isolated edges may safely be assigned weight 1.
%\begin{lemma}
%Define $\W(G)$ as follows:\\
%For each $(u,v) \in G$,\\
%if $\deg(u) = \deg(v) = 1$, $\W(u,v) =  \min(\frac{1}{\deg(u)},\frac{1}{\deg(v)})$.\\
%else $\W(u,v) =  \min(\frac{1}{\deg(u)+1},\frac{1}{\deg(v)+1})$.
%Then $\W(G) \leq M(G)$.
%\end{lemma}
%\begin{proof}
%Already proven above.
%\end{proof}
%Next we will amend $\W$ to include the following rule: when one endpoint has degree 1 and the other has %degree k, assign weight $\frac{1}{k}$.

Define $\W$ as follows:
\[\W=\sum_{(u,v)\in E} \W(u,v) \]
where
\[
\W(u,v)=\frac{1}{\max(\ndegree\lp u\rp, \ndegree\lp v\rp )+I[\min(\ndegree(u),\ndegree(v))> 1]}\]

We now show that $\W \leq M$. We have proven that $W(U)\leq (t-1)/2$ on any odd-size subgraph $U$, $|U| = t$.  Define subsets $U_1$ and $U_2$ of $U$ such that $U_1 \cup U_2 = U$.  $U_1$ is the set of all nodes in $U$ of degree 1 and all nodes in $U$ adjacent to a node of degree 1, and $U_2$ is the set of all other nodes in $U$.  Let $G(U_2)$ denote the subgraph of $U$ induced by $U_2$, and let $|U_1| = x$ where x is even.  Then $W(G(U_2)) = \W(G(U_2)) \leq (t-x-1)/2$.  So to complete our proof we must show that all remaining edges (call them $E'$) have total weight $\leq x/2$.

Assume WLOG that there are no isolated edges in $G$ (if there are, we can group them with $G(U_2)$ and retain the $(t-x-1)/2$ bound).

\begin{align*}
\W(E') & \leq \frac{1}{2} \sum_{u,v \in E'} \lp \frac{1}{\deg(u)}+\frac{1}{\deg(v)}\rp\\
& = \mathlarger{\sum_k} \Big(\sum_{v \in U_1, \deg(v) = k} \frac{\delta}{k} + \frac{k-\delta}{2(k+1)}\Big)
\end{align*}
where $\delta$ denotes the number of degree 1 nodes adjacent to $v$.\\
Let $f(\delta) = \delta/k + (k-\delta)/(2(k+1))$.  We are interested in finding the maximum value $f(\delta)/(\delta+1)$ can take on; if it can never take a value greater than $1/2$ then $\W(E')$ cannot be greater than $x/2$.
$$\frac{\partial\frac{f(\delta)}{\delta+1}}{\partial \delta} = \frac{2-k}{2k(\delta+1)^2}$$
which is always negative for $k \geq 2$.  So, $f(\delta)/(\delta+1)$ is maximized at $\delta = 0$, so  $f(\delta)/(\delta+1) \leq k/(2(k+1)) < 1/2.$

There is one final detail we have not considered: $x$ might be odd.  In this case, $|U_2|$ is even and we can't appeal to Theorem 3 to say that $\W(G(U_2)) \leq (t-x-1)/2$.  However, we can simply remove one node $\node'$ from $U_2$; the resulting odd-size subgraph has weight at most $(t-x-2)/2$.  Since $\W$ is a valid fractional matching, the weight assigned to all edges adjacent to $\node'$ cannot exceed 1, so we can say that $\W(G(U_2)) \leq (t-x)/2$.  Now we must show that $\W(E') \leq (x-1)/2$.

We have shown that the edge weight per node in $U_1$ cannot exceed $1/2$.  $\W(E')$ is minimized when there is exactly 1 node of degree 1, with a degree k neighbor. In this case, the only edge in $E'$ will be assigned weight $k/(2(k+1))< 1/2$.  $\W(E') \leq 1/2 + (x-2)/2 = (x-1)/2$ and the theorem is proven.
\end{proof}


\paragraph*{A remark on estimating $\W$.}  If we wish to use this lower bound to decide whether to begin meshing by predicting the size of the maximum matching, we cannot compute it exactly because we do not know the degrees of the nodes in the graph.  However, we do know the degree distribution of the graph and so it is possible to calculate the expected value of $\W$.





%\subsection{Summary}
%\label{subsec:theorysummary}

\subsection{Summary of Analytical Results}
We show the problem of meshing is reducible to a graph problem,
\textsc{MinCliqueCover}.  While solving this problem is infeasible, we
show that probabilistically, we can do nearly as well by finding the
maximum \textsc{Matching}, a much easier graph problem. We analyze our
meshing algorithm as an approximation to the maximum matching on a
random meshing graph, and argue that it succeeds with high
probability.  Finally, we prove a new lower bound on the maximum matching for graphs based on degree distribution.
As a corollary of these results, \Mesh breaks the Robson
bounds with high probability.

%for a broad class of applications.
%As long as programs do not alter
%programs that do not alter their allocation behavior in response to
%addresses the allocator returns.


\begin{comment}
  An adversary attempting to create an
unmeshable memory state requires the capability to position objects at
the same locations in multiple spans.  But an address-oblivious
adversary can only do this with low probability - since allocation
offsets within a span are random, the best it can do is free some
objects from each span and hope that some of the remaining objects
reside at the same offset.  Since objects are distributed uniformly at
random from within spans, our analytical results hold and therefore
there is either a large maximum matching on the resulting meshing
graph (indicating significant compaction is possible) or spans are so
full that fragmentation is not significant.
\end{comment}


\section{Summary of Evaluation}
\label{sec:evaluation}


% \end{figure*}
Our evaluation answers the following questions: Does \Mesh reduce
overall memory usage with reasonable performance overhead?
($\S$\ref{sec:evaluation:overallmemory}) Does randomization
provide empirical benefits beyond its analytical guarantees?
($\S$\ref{sec:evaluation:practical})

% (3) Is \Mesh nearly as effective as custom defragmentation solutions?

%% Our evaluation shows that \Mesh doesn't increase memory consumption in
%% the average case, that it significantly reduces memory consumtion
%% under fragmentation, and that it does not impose an unrealistic CPU
%% overhead.

%% We evaluate \Mesh in three ways.  We show that under a workload
%% derived from the Redis test suite, \Mesh is able to recover from
%% fragmentation in ways that other allocators can not match.  Next we
%% show that running the Redis server with mesh doesn't impose a
%% performance overhead on the workload generated by
%% \texttt{redis-benchmark}. Finally, we run SPECint 2006 with mesh and
%% show that it imposes only a modest overhead (geomean: 15\%).

\subsection{Experimental Setup}
\label{subsec:memory-use}

We perform all experiments on a MacBook Pro with 16 GiB of RAM and an
Intel i7-5600U, running Linux 4.18 and Ubuntu Bionic. We use glibc
2.26 and jemalloc 3.6.0 for SPEC2006, Redis 4.0.2, and Ruby 2.5.1.
Two builds of Firefox 57.0.4 were compiled as release builds, one with
its internal allocator disabled to allow the use of alternate
allocators via \texttt{LD\_PRELOAD}.  SPEC was compiled with clang
version 4.0 at the \texttt{-O2} optimization level, and \Mesh was
compiled with gcc 8 at the \texttt{-O3} optimization level and with
link-time optimization (\texttt{-flto}).  We primarily compare Mesh
results to the default allocator for each test (jemalloc for Firefox
and Redis, glibc for SPEC and Ruby). Where appropriate, we also include
results from tcmalloc (gperftools 2.5) and Hoard (git commit
\texttt{a0e46aa1}); when omitted, it is because their performance is virtually identical
to jemalloc and glibc.

\textbf{Measuring memory usage:} To accurately measure the memory
usage of an application over time, we developed a Linux-based utility,
\texttt{mstat}\footnote{\texttt{mstat} is open source, and available at \url{https://github.com/bpowers/mstat}}, that runs a program in a new memory control
group~\cite{redhat:cgroups}. \texttt{mstat} polls the resident-set size
(RSS) and kernel memory usage statistics for all processes in the
control group at a constant frequency.  This enables us to account for
the memory required for larger page tables (due to meshing) in our
evaluation. We have verified that \texttt{mstat} does not perturb
performance results.
%We verified that a full run of SPECint 2006 with benchmarks run under
%\texttt{mstat} has the same performance as run without, so the
%additional accounting being performed does not perturb performance
%results.

\begin{comment}
\begin{figure*}[t]
  \centering\small\textbf{~~~~~~~~~SPECint 2006 Relative Mean RSS}\\[0.3em]
  \includegraphics{plots/spec_mem_mean.pdf}
  \caption{\textbf{\Mesh maintains low average memory consumption.}  The SPEC benchmark
    suite does not suffer from serious external fragmentation that
    \Mesh can recover from, but it maintains lower average memory consumption than glibc.
  For \texttt{403.gcc}, \Mesh reduces mean RSS by 15\% (from 220MB to 188MB).}
  \label{fig:spec-mean-rel}
\end{figure*}
\end{comment}

\begin{comment}
\begin{figure*}[!t]
  \centering\small\textbf{~~~~~~~~~SPECint 2006 Absolute Mean RSS}\\[0.3em]
  \includegraphics{plots/spec_mem_mean_mb.pdf}
  \caption{\textbf{\Mesh average memory consumption (absolute).}}
  \label{fig:spec-mean}
\end{figure*}
\end{comment}

\begin{comment}
\begin{figure*}[!t]
  \centering\small\textbf{~~~~~~~~~SPECint 2006 Performance}\\[0.3em]
  \includegraphics{plots/spec_speed.pdf}
  \caption{\textbf{\Mesh imposes modest runtime overhead.} Performance is
    normalized to glibc. Overall, \Mesh's execution time overhead
    is 15\% higher than glibc's (geometric mean).}
  \label{fig:spec-speed}
\end{figure*}
\end{comment}



\subsection{Memory Savings and Performance Overhead}
\label{sec:evaluation:overallmemory}

We evaluate \Mesh{}'s impact on memory consumption and runtime across
the Firefox web browser, the Redis data structure store, and the
SPECint2006 benchmark suite.

\subsubsection{Firefox}
\label{sec:firefox}

Firefox is an especially challenging application for memory reduction
since it has been the subject of a five year effort to reduce its
memory footprint~\cite{awsy}. To evaluate \Mesh{}'s impact on
Firefox's memory consumption under realistic conditions, we measure
Firefox's RSS while running the Speedometer 2.0 benchmark.
Speedometer was constructed by engineers working on the Google Chrome
and Apple Safari web browsers to simulate the patterns in use on
websites today, stressing a number of browser subsystems like DOM
APIs, layout, CSS resolution and the JavaScript engine.  In Firefox,
most of these subsystems are multi-threaded, even for a single
page~\cite{ff:quantum}.  The benchmark comprises a number of small
``todo'' apps written in a number of different languages and styles,
with a final score computed as the geometric mean of the time taken by
the executed tests.

\begin{figure}[t!]% {.4\linewidth}
  \centering
  \vtop{%
    \centering
    \vskip0pt
    \hbox{%
      \includegraphics[width=\textwidth]{Chapters/mesh/plots/firefox-speedometer.pdf}
    }%
  }
  \caption{\textbf{Firefox:} \Mesh decreases mean heap size by 16\%
    over the course of the Speedometer 2.0 benchmark compared with the
    version of jemalloc bundled with Firefox, with less than a 1\%
    change in the reported Speedometer score (\S\ref{sec:firefox}).
    \label{fig:firefox-heap}}
\end{figure}

We test Firefox in single-process mode (disabling content sandboxing,
which spawns multiple processes) under the \texttt{mstat} tool to
record memory usage over time. Our test opens a tab, loads the
Speedometer page from a local server, waits 2 seconds, and then
automatically executes the test.  We record the reported score
at the end of the benchmark run and calculate average memory
usage recorded by \texttt{mstat}.  We tested both a standard release
build of Firefox, along with a release build that did not bundle
Mozilla's fork of jemalloc (hereafter referred to as
\texttt{mozjemalloc}) and instead directly called \texttt{malloc}-related
functions, with \Mesh included via \texttt{LD\_PRELOAD}.  We report
the average resident set size over the course of the benchmark and a
15 second cooldown period afterward, collecting three runs per
allocator.

\Mesh reduces the memory consumption of Firefox by 16\% compared to
Firefox's bundled jemalloc allocator. \Mesh requires 530 MB ($\sigma =
22.4$ MB) to complete the benchmark, while the Mozilla allocator needs
632 MB ($\sigma = 25.3$ MB). Mesh spent a total of 135 ms meshing over
the course of the benchmark, with a maximum meshing latency of 7.5 ms
and average meshing latency of 0.2 ms.  This result shows that \Mesh
can effectively reduce memory overhead even in widely used and heavily
optimized applications. \Mesh achieves this savings with less than a
1\% reduction in performance (measured as the score reported by
Speedometer).  Hoard and tcmalloc improved Speedometer performance
relative to jemalloc by 5.4 and 6.0\% respectively, while increasing
average heap size by 48.0\% and 8.6\%.

Figure~\ref{fig:firefox-heap} shows memory usage over the course of a
Speedometer benchmark run under \Mesh and the default jemalloc
allocator.  While memory usage under both peaks to similar levels,
Mesh is able to keep heap size consistently lower.


\subsubsection{Redis}
\label{redis-section}


Redis is a widely-used in-memory data structure server.  Redis 4.0
introduced a feature called ``active
defragmentation''~\cite{jemalloc:exposehints,redis:announcement}.
Redis calculates a fragmentation ratio (RSS over sum of active
allocations) once a second.  If this ratio is too high, it triggers a
round of active defragmentation. This involves making a fresh copy of
Redis's internal data structures and freeing the old ones. Active
defragmentation relies on allocator-specific APIs in jemalloc both for
gathering statistics and for its ability to perform allocations that
bypass thread-local caches, increasing the likelihood they will be
contiguous in memory.

\begin{figure}[t!]
  \centering
  \vtop{%
    \centering
    \vskip0pt
    \hbox{%
      \includegraphics[width=\textwidth]{Chapters/mesh/plots/redis-lru.pdf}
    }%
  }
  \caption{\textbf{Redis:} \Mesh automatically achieves significant
    memory savings (39\%), obviating the need for its custom,
    application-specific ``defragmentation'' routine
    (\S\ref{redis-section}).
    \label{fig:redis-results}}
\end{figure}


%This defragmentation
%does not require scanning the heap for pointers because Redis's memory
%primarily consists of a large hash table associating keys with values.


%% A case where active defragmentation is useful is when redis is used as
%% an LRU cache with a maximum heap size set.  If there is a phase shift
%% in the size of objects being added to the cache, redis can suffer from
%% external fragmentation, requiring a significantly larger
%% working-set-size compared to live size of redis allocations.

%% Redis tracks the size of allocations (either through the use of a
%% \texttt{malloc\_usable\_size} API, or by prepending the size to every
%% allocation

We adapt a benchmark from the official Redis test suite to measure how
\Mesh's automatic compaction compares with Redis's active
defragmentation, as well as against the standard glibc allocator. This
benchmark runs for a total of 7.5 seconds, regardless of allocator. It
configures Redis to act as an LRU cache with a maximum of 100 MB of
objects (keys and values).  The test then allocates 700,000 random
keys and values, where the values have a length of 240 bytes.
Finally, the test inserts 170,000 new keys with values of length 492.
Our only change from the original Redis test is to increase the value
sizes in order to place all allocators on a level playing field with
respect to \emph{internal} fragmentation; the chosen values of 240 and
492 bytes ensure that tested allocators use similar size classes for
their allocations. We test \Mesh with Redis in two configurations:
with meshing always on and with meshing disabled, both without any
input or coordination from the redis-server application.

%% Our only change from the original Redis test is to
%% increase the key size in order to place all allocators on a level
%% playing field with respect to \emph{internal} fragmentation; both
%% Hoard and DieHard use power-of-two size classes.

Figure~\ref{fig:redis-results} shows memory usage over time for Redis
under \Mesh, as well as under jemalloc with Redis's
``activedefrag'' enabled, as measured by \texttt{mstat}
(\S\ref{subsec:memory-use}).  The ``activedefrag'' configuration
enables active defragmentation after all objects have been added to
the cache.

Using \Mesh automatically and portably achieves the same heap size
reduction (39\%) as Redis's active defragmentation.  During most of
the 7.5s of this test Redis is idle; Redis only triggers active
defragmentation during idle periods. With \Mesh, insertion takes
1.76s, while with Redis's default of jemalloc, insertion takes 1.72s.
Redis under Hoard and tcmalloc has the same average heap size as Mesh
with meshing disabled (under 2\% difference), and both allocators are
similarly within 2\% of the insertion speed of jemalloc.  \Mesh's
compaction is additionally \textit{significantly} faster than Redis's
active defragmentation. During execution with \Mesh{}, a total of
0.23s are spent meshing (the longest pause is 22 ms), while active
defragmentation accounts for 1.49s ($5.5\times$ slower). This high
latency may explain why Redis disables ``activedefrag'' by default.

%\Mesh's always-on defragmentation comes at a cost, the Redis server
%running with \Mesh is 29\% slower than the default jemalloc (which
%performs no defragmentation).

\subsubsection{SPEC Benchmarks}

%Figure~\ref{fig:spec-mean-rel} presents mean memory consumption (RSS)
%across the SPECint 2006 benchmark suite, normalized to glibc.

%Figure~\ref{fig:spec-mean} presents absolute average memory
%usage.

%% We show that meshing can yield substantial memory savings across a
%% suite of benchmarks and real-world applications, with little runtime
%% overhead. For one of the most allocation-heavy benchmarks in the
%% SPECint suite, \texttt{perlbench}, \Mesh reduces peak RSS by 13\%
%% (across the suite, it decreases average memory consumption by 2.5\%).
%% In longer-lived, memory-intensive applications, \Mesh saves more
%% memory. Replacing Firefox's default allocator with \Mesh reduces
%% average memory consumption in Speedometer 2, a modern browser
%% benchmark, by 11\% (from 651MB to 579MB) with a 6\% reduction in
%% performance.  For the Redis data structure server, \Mesh reduces
%% resident-set size (RSS) by 34\% compared to conventional allocators in
% a fragmentation-heavy workload.


Most of the SPEC benchmarks are not particularly compelling targets
for \Mesh because they have small overall footprints and do not
exercise the memory allocator. For example, while the
\texttt{401.bzip2} benchmark has one of the higher heap size averages
at 665 MB, the difference in both runtime and average heap size
between the fastest + slowest allocators is under 1\%.

Across the entire SPECint 2006 benchmark suite, \Mesh modestly
decreases average memory consumption (geomean: $-2.4$\%) vs. glibc,
while imposing minimal execution time overhead (geomean: 0.7\%).

%% Many Mesh benchmarks are

However, for applications that are both allocation-intensive (many
calls to \texttt{malloc} and \texttt{free}) and which have large footprints,
\Mesh is able to substantially reduce peak memory consumption. The
most allocator-sensitive benchmark is \texttt{400.perlbench}, a Perl
benchmark that performs a number of e-mail related tasks including
spam detection (SpamAssassin). Peak RSS with glibc, jemalloc, and
Hoard is 664 MB, 614 MB, and 732 MB respectively. \Mesh reduces peak
RSS to 564 MB (a 15\% reduction relative to glibc) while increasing
runtime overhead by only 3.9\%.


%\Mesh significantly reduces memory consumption for the
%allocation-heavy SPEC application (\texttt{400.perlbench}: 13\%) while
%imposing a minimal performance penalty.


% As Figure~\ref{fig:spec-speed} shows,
\begin{comment}
though in some cases, its overhead is high. We observe that these
cases correspond with the cases when DieHard also imposes significant
overhead; since both DieHard and \Mesh are randomized memory managers,
we attribute this overhead to degraded locality.
\end{comment}



\subsection{Empirical Value of Randomization}
\label{sec:evaluation:practical}

Randomization is key to \Mesh{}'s analytic guarantees; we evaluate
whether it also can have an observable empirical impact on its ability
to reclaim space. To do this, we test three configurations of \Mesh:
(1) meshing disabled, (2) meshing enabled but randomization disabled,
and (3) \Mesh with both meshing and randomization enabled (the
default).

We tested these configurations with Firefox and Redis, and found no
significant differences when randomization was disabled; we believe
that this is due to the highly irregular (effectively random)
allocation patterns that these applications exhibit. We hypothesized
that a more regular allocation pattern would be more challenging for a
non-randomized baseline. To test this hypothesis, we wrote a synthetic
microbenchmark with a regular allocation pattern in Ruby. Ruby is an
interpreted programming language popular for implementing web
services, including GitHub, AirBnB, and the original version of
Twitter.  Ruby makes heavy use of object-oriented and functional
programming paradigms, making it allocation-intensive.  Ruby is
garbage collected, and while the standard MRI Ruby implementation
(written in C) has a custom GC arena for small objects, large objects
(like strings) are allocated directly with \texttt{malloc}.

Our Ruby microbenchmark repeatedly performs a sequence of string
allocations and deallocations, simulating the effect of accumulating
results from an API and periodically filtering some out. It allocates
a number of strings of a fixed size, then retaining references 25\% of
the strings while dropping references to the rest.  Each iteration the
length of the strings is doubled.  The test requires only a fixed 128
MB to hold the string contents.

\begin{figure}[t!]% {.4\linewidth}
  \centering
  \vtop{%
    \centering
    \vskip0pt
    \hbox{%
      \includegraphics[width=\textwidth]{Chapters/mesh/plots/ruby-frag.pdf}
    }%
  }
  \caption{\textbf{Ruby benchmark:} \Mesh is able to decrease mean heap
    size by 18\% compared to \Mesh with randomization disabled and
    non-compacting allocators ($\S$\ref{sec:evaluation:practical}).
    \label{fig:ruby-frag}}
\end{figure}

Figure~\ref{fig:ruby-frag} presents the results of running this
application with the three variants of \Mesh{} and jemalloc; for this
benchmark, jemalloc and glibc are essentially indistinguishable. With
meshing disabled, \Mesh exhibits similar runtime and heap size to
jemalloc. With meshing enabled but randomization disabled, \Mesh
imposes a 4\% runtime overhead and yields only a modest 3\% reduction in
heap size.

Enabling randomization in \Mesh increases the time overhead to 10.7\%
compared to jemalloc, but the use of randomization lets it
significantly reduce the mean heap size over the execution time of the
microbenchmark (a 19\% reduction). The additional runtime overhead is
due to the additional system calls and memory copies induced by the
meshing process.  This result demonstrates that randomization is not
just useful for providing analytical guarantees but can also be
essential for meshing to be effective in practice.

\subsection{Summary of Empirical Results}




For a number of memory-intensive applications, including aggressively
space-optimized applications like Firefox, \Mesh can substantially
reduce memory consumption (by 16\% to 39\%) while imposing a modest
impact on runtime performance (e.g., around 1\% for Firefox and
SPECint 2006). We find that \Mesh{}'s randomization can enable
substantial space reduction in the face of a regular allocation
pattern.

%We find that \Mesh{}'s use of randomization can be
%to reclaim memory

%slightly memory consumption across the SPEC benchmark
%suite and low performance overhead. For two real,
%memory-intensive applications (Firefox and Redis), \Mesh's compaction
%reduces memory consumption significantly with minimal performance
%impact.

%  both reach a steady state where the occupancy of spans containing smaller (240 byte) objects is between 60-65\%, and there are very few spans under half full.


\begin{comment}
  \begin{figure*}[!t]
  \centering\small\textbf{~~~~~~~~~SPECint 2006 Relative Peak RSS}\\[0.3em]
  \includegraphics{plots/spec_mem_peak.pdf}
  \caption{\textbf{\Mesh has little impact on peak memory consumption.}
    The meshing algorithm requires additional bookkeeping and
    temporary allocations within the allocator, but these do not
    translate to increased memory usage across SPEC, and sometimes reduce
    it. \Mesh reduces \texttt{400.perlbench}'s peak memory consumption
    by 13.7\% (from 663MB to 575MB).}
  \label{fig:spec-peak-rel}
\end{figure*}
\end{comment}


\input{Chapters/mesh/discussion}

\section{Related Work}
\label{sec:related-work}

\paragraph*{Hound:}
\label{sec:hound}
Hound is a memory leak detector for C/C++ applications that introduced
meshing (a.k.a. ``virtual compaction''), a mechanism that \Mesh{}
leverages~\cite{1542521}. Hound combines an age-segregated heap with
data sampling to precisely identify leaks. Because Hound cannot
reclaim memory until every object on a page is freed, it relies on a
heuristic version of meshing to prevent catastrophic memory
consumption. Hound is unsuitable as a replacement general-purpose
allocator; it lacks both \Mesh's theoretical guarantees and space and
runtime efficiency (Hound's repository is missing files and it does
not build, precluding a direct empirical comparison here). The Hound
paper reports a geometric mean slowdown of $\approx 30\%$ for
SPECint2006 (compared to \Mesh{}'s 0.7\%), slowing one benchmark
(\texttt{xalancbmk}) by almost $10\times$. Hound also generally
\emph{increases} memory consumption, while \Mesh often substantially
decreases it.

%Finally, Hound was designed for
%32-bit Linux applications, making more direct comparisons to Mesh
%(which is focused on 64-bit applications) hard.


\paragraph*{Compaction for C/C++:}
Previous work has described a variety of manual and compiler-based
approaches to support compaction for C++. Detlefs shows that if
developers use annotations in the form of smart pointers, C++ code can
also be managed with a relocating garbage
collector~\cite{detlefs:1992:gc}.  Edelson introduced GC support
through a combination of automatically generated smart pointer classes
and compiler transformations that support relocating
GC~\cite{edelson:1992:precompilingcgc}. Google's Chrome uses an
application-specific compacting GC for C++ objects called Oilpan that
depends on the presence of a single event
loop~\cite{google:oilpan}. Developers must use a variety of smart
pointer classes instead of raw pointers to enable GC and
relocation. This effort took years. Unlike these approaches, \Mesh is
fully general, works for unmodified C and C++ binaries, and does not
require programmer or compiler support; its compaction approach is
orthogonal to GC.

CouchDB and Redis implement \emph{ad hoc} best-effort compaction,
which they call ``defragmentation''.  These work by iterating through
program data structures like hash tables, copying each object's
contents into freshly-allocated blocks (in the hope they will be
contiguous), updating pointers, and then freeing the old
objects~\cite{jemalloc:exposehints,redis:announcement}. This
application-specific approach is not only inefficient (because it may
copy objects that are already densely packed) and brittle (because it
relies on internal allocator behavior that may change in new
releases), but it may also be ineffective, since the allocator cannot
ensure that these objects are actually contiguous in memory. Unlike
these approaches, \Mesh performs compaction efficiently and its
effectiveness is guaranteed.

\paragraph*{Compacting garbage collection in managed languages:}
%% Recently Baker et al. showed that with compiler support you could
%% achieve accurate GC for C and C++~\cite{baker:2009:accurategc}.
Compacting garbage collection has long been a feature of languages
like LISP and
Java~\cite{hansen:1969:compaction,fenichel:1969:compaction}. Contemporary
runtimes like the Hotspot JVM~\cite{microystems2006memory}, the .NET
VM~\cite{microsoft:dotnet-gc}, and the SpiderMonkey JavaScript
VM~\cite{mozilla:spidermonkey-compaction} all implement compaction as
part of their garbage collection algorithms. \Mesh{} brings the
benefits of compaction to C/C++; in principle, it could also be used
to automatically enable compaction for language implementations that
rely on non-compacting collectors.

\paragraph*{Bounds on Partial Compaction:}
Cohen and Petrank prove upper and lower bounds on defragmentation via
partial compaction~\cite{Cohen:2017:LPC:3050768.2994597,
  Cohen:2013:LPC:2491956.2491973}. In their setting, corresponding to
managed environments, \emph{every} object \emph{may} be relocated to
any free memory location; they ask what space savings can be achieved
if the memory manager is only allowed to relocate a bounded number of
objects. By contrast, \Mesh{} is designed for unmanaged languages
where objects \emph{cannot} be arbitrarily relocated.

%\paragraph{Hardware support:}
%Seshadri \emph{et al.} present a hardware proposal that would allow
%for ``page overlays'' at a cache-line
%granularity~\cite{Seshadri:2015:POE:2749469.2750379}. \Mesh could use
%such a facility instead of page remapping to mesh pages containing
%objects of a cache line size (e.g., 64 bytes) or larger.

\paragraph*{PCM fault mitigation:}
Ipek \emph{et al.} use a technique similar to meshing to address the
degradation of phase-change memory (PCM) over the lifetime of a
device~\cite{ipek:2010:dynamic-replication}.  The authors introduce
dynamically replicated memory (DRM), which uses pairs of PCM pages
with non-overlapping bit failures to act as a single page of
(non-faulty) storage.  When the memory controller reports a page with
new bit failures, the OS attempts to pair it with a complementary
page. A random graph analysis is used to justify this greedy
algorithm.

DRM operates in a qualitatively different domain than \Mesh.  In DRM,
the OS occasionally attempts to pair newly faulty pages
against a list of pages with static bit failures.  This process is
incremental and local.  In \Mesh, the occupancy of spans in the heap
is more dynamic and much less local. \Mesh solves a full,
non-incremental version of the meshing problem each cycle.
Additionally, in DRM, the random graph describes an error model rather
than a design decision; additionally, the paper's analysis is flawed.
The paper erroneously claims that the resulting graph is a simple
random graph; in fact, its edges are not independent (as we show in
\S\ref{subsec:matching}).  This invalidates the claimed performance
guarantees, which depend on properties of simple random graphs. In
contrast, we prove the efficacy of our original \sm algorithm for
\Mesh using a careful random graph analysis.

\begin{comment}
  To avoid catastrophic memory consumption,
Hound employs virtual compaction, a non-randomized, best-effort form
of meshing. Their approach depends on heuristics and does not employ
randomization or come with any guarantees of fragmentation
recovery.
  Hound is not intended to be
space-efficient (and it is not) but rather to find leaks. Hound
identifies leaks by segregating objects by allocation site, protects
``cold'' pages, and delays reuse of pages until every object on that
page has been freed.
\end{comment}

%% Virtual memory operations have additionally been use for compaction in
%% a Java garbage collector~\cite{wegiel:2008:mapping-collector} in a
%% novel collector, as well as in a C allocator that traded address space
%% usage for reliability and security~\cite{1346296}.

%Microsoft's C++/CLI? C++ with GC'ed, moveable objects.


% http://www.filpizlo.com/papers/baker-ccpe09-accurate.pdf


%Theory?

%% Objects with similar lifetimes tend to appear
%% together~\cite{wilson:1995:survey}.


\section{Conclusion}
\label{sec:conclusion}

This chapter introduces \Mesh{}, a memory allocator that efficiently
performs \textit{compaction without relocation} to save memory for
unmanaged languages.  We show analytically that \Mesh{} provably
avoids catastrophic memory fragmentation with high probability, and
empirically show that \Mesh{} can substantially reduce memory
fragmentation for memory-intensive applications written in C/C++ with
low runtime overhead. %In future work, we plan to explore integrating \Mesh{} into language runtimes that do not currently support compaction, such as Go and Rust.

We have released \Mesh as an open source
project; it can be used with arbitrary C and C++ Linux and Mac OS X
binaries and can be downloaded at
\url{http://libmesh.org}.


\iffalse

%% contents suppressed with 'anonymous'
\begin{acks}
  %% Commands \grantsponsor{<sponsorID>}{<name>}{<url>} and
  %% \grantnum[<url>]{<sponsorID>}{<number>} should be used to
  %% acknowledge financial support and will be used by metadata
  %% extraction tools.
  This material is based upon work supported by the
  \grantsponsor{GS100000001}{National Science
    Foundation}{http://dx.doi.org/10.13039/100000001} under Grant
  No.~\grantnum{GS100000001}{1637536}.  Any opinions, findings, and
  conclusions or recommendations expressed in this material are those
  of the author and do not necessarily reflect the views of the
  National Science Foundation.
\end{acks}

\fi

%\bibliography{Chapters/mesh/emery,Chapters/mesh/mesh}

%% \clearpage
%% \pagebreak
\section{Appendix}
\label{sec:appendix}
\newtheorem*{remark}{Remark}

\newcommand{\rp}{\right)}


% TODO: reorganize!

\subsection{Proof of Theorem \ref{thm:polytime}}

\begin{comment}
\begin{proof}
We assume without loss of generality that $S$ does not contain the all-zero string $\str_0$; if it does, since $\str_0$ can be meshed with any other string and so can always be released, we can solve the meshing problem for $S \setminus \str_0$ and then mesh each instance of $\str_0$ arbitrarily.

Rather than reason directly about the Min Clique Cover problem on some bounded-length meshing graph $G$, let us consider the equivalent problem of coloring $\bar{G}$, the complement of $G$.  $\bar{G}$ has an edge between  every pair of two nodes whose strings do not mesh.

$\bar{G}$'s node set $N$ can be partitioned into at most $2^b-1$ subsets $N_1, N_2, ... N_{2^b-1}$ such that $\forall i$, all nodes in $N_i$ represent the same string $\str_i$.  The induced subgraph of $N_i$ is a clique since all its nodes have a 1 in the same position and so cannot be pairwise meshed.  Further, all nodes in $N_i$ has the same neighbors since they all represent the same string.

Since $N_i$ is a clique, at most one node in $N_i$ may be colored with any color.  Fix some coloring on $\bar{G}$.  Swapping the colors of two nodes in $N_i$ does not change the validity of the coloring since these nodes have the same neighbor set.  We can therefore unambiguously represent a valid coloring of $\bar{G}$ merely by indicating in which cliques each color appears.

With $2^b$ cliques and a maximum of $n$ colors, there are at most $\lp n+1 \rp ^{2^{2^b}}$ such colorings on the graph.  This is polynomial in $n$ since $b$ is fixed, so we can simply check each coloring for validity (a coloring is valid iff no color appears in two cliques whose string representations mesh).  Finally, the algorithm returns a valid coloring with the lowest number of colors out of all valid colorings discovered.
\end{proof}
\end{comment}


We assume without loss of generality that $S$ does not contain the all-zero string $\str_0$; if it does, since $\str_0$ can be meshed with any other string and so can always be released, we can solve the meshing problem for $S \setminus \str_0$ and then mesh each instance of $\str_0$ arbitrarily.

Let $\bar G = (V, \bar E)$ be the complement of $G$.  Solving \textsc{MinCliqueCover} on $G(S)$ is equivalent to solving \textsc{Coloring} on $\bar G$.

\begin{lemma}
There are a polynomial number of colorings on $\bar G$.
\end{lemma}
\begin{proof}
Partition $V$ into $V_1, V_2, ... V_k$ such that $V_i = \{u \vert str(u) = s_i\}$.  Note that the induced subgraph of $\bar G$ on each $V_i$ is a clique, and further that $k \leq 2^b-1$.

Since $V_i$ is a clique, at most one node in $V_i$ may be colored with any color.  Fix some coloring on $\bar{G}$.  Swapping the colors of two nodes in $V_i$ does not change the validity of the coloring since these nodes have the same neighbor set.  We can therefore unambiguously represent a valid coloring of $\bar{G}$ merely by indicating in which cliques each color appears.

With $2^b$ cliques and a maximum of $|V| = n$ colors, there are at most $\lp n+1 \rp ^{2^{2^b}}$ such colorings on the graph.
\end{proof}

As a result of the lemma, we can check all colorings in polynomial time and return the minimum valid coloring.

\begin{remark}
This proof relies on the assumption that $b$, string length, is fixed.  This constraint limits meshing graphs to a strict subset of all possible graphs.  If $b = \bigo(n^2)$, any graph can be expressed as a meshing graph and so \textsc{MinCliqueCover} is NP-Hard in this case.
\end{remark}

\subsection{Experimental Confirmation of Maximum Matching/Min Clique Cover Convergence}

In Section~\ref{subsec:matching}, we argue that we can approximate the solution to \textsc{MinCliqueCover} on meshing graphs with high probability by instead solving \textsc{MaximumMatching}.

We experimentally verify this result by generating many random constant occupancy graphs and, for each graph, comparing the size of the maximum matching to the size of a greedy (non-optimal) solution for \textsc{MinCliqueCover}.  The results are summarized in Figure~\ref{plot:const}.

\begin{figure}[h]
\includegraphics[scale = .5]{figures/const_match_comp.pdf}
\centering
\caption{\textbf{Min Clique Cover and Max Matching solutions converge.} The average size of Min Clique Cover and Max Matching for randomly generated constant occupancy meshing graphs, plotted against span occupancy.  Note that for sufficiently high-occupancy spans, Min Clique Cover and Max Matching are nearly equal.}
\label{plot:const}
\end{figure}



When we instead assume bits are 1 independently with probability p, we expect the graph to have many more triangles.  For $p = r/b = 10/32, n = 1000$, the expected number of triangles is roughly 36,000.  However, we can see experimentally that these graphs behave quite similarly in Figure~\ref{plot:indep}.


\begin{figure}[h]
\includegraphics[scale = .5]{figures/ind_match_comp.pdf}
\centering
\caption{\textbf{Converge still holds for independent bits assumption.} The average size of Min Clique Cover and Max Matching for randomly generated constant occupancy meshing graphs, plotted against span occupancy.  Note that for sufficiently high-occupancy spans, Min Clique Cover and Max Matching are nearly equal.}
\label{plot:indep}
\end{figure}

While constant occupancy graphs are fairly regular, independent bit graphs may not be.  Since strings have different occupancies, nodes which correspond to strings with relatively low occupancy will tend to have significantly higher degree than other nodes in the graph.  Meanwhile, other nodes may have strings with high occupancy, and therefore only have a few edges (probably with low-occupancy nodes).  So while there are many triangles, when the graph is sparse enough, even meshing cliques of size 3 and 4 will likely "abandon" adjacent high-occupancy nodes, one of which could have been matched with the high degree node to yield the same number of releases.

So in this case we still expect finding the maximum matching to be good enough.

\iffalse
\subsection{Proof of Theorem 4.4}
\begin{proof}
Since spans all have identical occupancies, we expect the meshing graph to be nearly regular - the degree of each node is binomially distributed with mean $\lp n-1\rp q$.  When there are many objects per span, the degree distribution will be tightly concentrated around this mean.  For example, when $b = 128, n = 1000, q = .05$, the mean degree is 50 and less than 5\% of nodes have a degree greater than 60.

When we split the span set into half, and only consider pairs between the halves, we essentially remove half of the edges from the meshing graph.  To see this, observe that there are ${{n}\choose{2}} \approx n^2/2$ span pairs, and we fail to consider $2*{{n/2}\choose{2}} \approx n^2/4$ of them.  Note also that we fail to consider exactly $n/2 -1$ pairs for each span.  We therefore expect both the total number of edges in the meshing graph and the degree of each node to decrease by half.

How does the maximum matching of a graph change when you evenly sparsify the graph by half?  If the average degree is high enough, we expect this to not make much difference at all - each node in expectation still has many edges from which a matching can be built.

We can formalize this argument by first assuming that every node in the graph has degree at least $d$ and at most $\lp 1 + \epsilon \rp d$.  As we prove in Section~\ref{subsec:lowerbound}, the maximum matching of a graph $G = \lp V, E\rp$ is lower bounded by

$$W\lp G \rp = \sum_{\lp u,v \rp \in V} \min \lp \frac{1}{\degree \lp u \rp +1}, \frac{1}{\degree \lp v \rp +1} \rp$$

With our degree assumption we can further argue

\begin{align*}
W\lp G \rp \geq \sum_{\lp u,v \rp \in V} \frac{1}{\lp 1+\epsilon \rp d+1}
 \geq \frac{nd/2}{\lp 1+\epsilon \rp d+1}
\geq \frac{n}{2\lp 1+\epsilon \rp}
\end{align*}
and $W\leq n/2$.
%\lp G \rp &\leq \sum_{\lp u,v \rp \in V} \frac{1}{d+1} \leq \frac{n}{2}$
%\begin{align*}
%W\lp G \rp &\leq \sum_{\lp u,v \rp \in V} \frac{1}{d+1} \leq \frac{n}{2}
%\end{align*}

Note how these upper and lower bounds are independent of $d$ and $m$.  Thus, we decrease the maximum matching by at most a $1/\lp 1+\epsilon \rp$ factor by splitting the span set.
\end{proof}
\fi

\subsection{Proof of Theorem 4.5}
\begin{proof}
We begin by showing that the simpler quantity $W$ is a lower bound of the maximum matching $M$.
\[
W = \sum_{e\in E} \frac{1}{\max(\degree\lp u\rp, \degree\lp v\rp )+1}\ .
\]
%For each edge $e = (u,v) \in G$, define $\W$$(e)=min\lp \frac{1}{\degree\lp u\rp +1},\frac{1}{\degree\lp %v\rp +1}\rp$.  L
Let $U$ be an arbitrary set of $t$ nodes in $G$ where $t$ is odd.   Define $$W(U) = \sum_{u,v \in U} \frac{1}{\max(\degree\lp u\rp, \degree\lp v\rp )+1} \ . $$ As a corollary of Edmonds Matching Polytope Theorem, it can be shown that $W \leq M$ if  $W(U) \leq (|U|-1)/2$. We can argue this as follows:

\begin{align*}
W(U) &= \sum_{(u,v) \in U} \min \left(\frac{1}{\degree(u)+1},\frac{1}{\degree(v)+1}\right )\\
& \leq \sum_{(u,v) \in U}\frac{1}{2}\lp \frac{1}{\degree\lp u \rp +1}+\frac{1}{\degree\lp v\rp +1}\rp\\
&\leq \sum_{(u,v) \in U}\frac{1}{2}\lp \frac{1}{\degree_U\lp u\rp +1}+\frac{1}{\degree_U\lp v\rp +1}\rp\\
&= \frac{1}{2} \sum_{u \in U} \frac{\degree_U\lp u\rp }{\degree_U\lp u\rp +1} \leq  \frac{1}{2} \lp \frac{t-1}{t}\rp  t = \frac{t-1}{2}
\end{align*}
where $\degree_U(u)$ in the number of neighbors of $u$ in the set $U$.
The second line follows from the fact that the minimum of two quantities is bounded above by their average.  The third line follows from the fact that the degree of any node in a subgraph is bounded above by its degree in the original graph.  The fourth line follows from summing over nodes instead of edges, and then reasoning that in the worst case U is a clique and so $\degree_U(u) = t-1$ for all u $u$.

\iffalse
Recall that we prove in Section~\ref{subsec:lowerbound} that when
\[
w_e=\min\lp \frac{1}{\degree\lp u\rp+1 },\frac{1}{\degree\lp v\rp+1 }\rp
~~\mbox{ and }~~\W=\sum_{e\in E} w_e \ .\]
then $\W$ is a lower bound on the cardinality of the maximum matching M(G).
\fi


In some cases $W$ is too conservative; it assigns little weight to edges which it could safely have assigned much more.  For example, if $e$ is isolated (meaning its endpoints have degree 1), $W(e) = 1/2$.  However, it is always safe to assign weight 1 to $e$, since $M(G-e) = M(G)-1.$  If we modified our rule for $W$ so that for any edge $e= (u,v)$ s.t. $\deg(u)=\deg(v) = 1$ we assigned weight $ \min(1/\deg(u),1/\deg(v))$ instead of $ \min(1/\deg(u)+1,1/\deg(v)+1)$, we would always assign weight 1 to isolated edges.

In fact, a more general rule is true.  For any edge $e= (u,v)$, if either $\deg(u) or \deg(v) = 1$ then we may assign it weight $\min(1/\deg(u),1/\deg(v))$.

\iffalse
At this point it is natural to ask whether we can remove the $+1$s from the denominator for other edges.  We cannot of course do so for $\deg(u) = \deg(v) = k >1, k \in 2\mathbb{Z}$ since for a clique on k+1 nodes this rule would result in total weight $(k+1)/2$.\\

However, we can remove the $+1$ under certain circumstances. We show that if one of the endpoints has degree one then this is the case.
\fi

%prove several rules of this form.  For each rule we specify values of $\deg(u)$ and $\deg(v)$ for which the %$+1$ can be removed from the denominator of the edge weight.
%First we restate the claim that isolated edges may safely be assigned weight 1.
%\begin{lemma}
%Define $\W(G)$ as follows:\\
%For each $(u,v) \in G$,\\
%if $\deg(u) = \deg(v) = 1$, $\W(u,v) =  \min(\frac{1}{\deg(u)},\frac{1}{\deg(v)})$.\\
%else $\W(u,v) =  \min(\frac{1}{\deg(u)+1},\frac{1}{\deg(v)+1})$.
%Then $\W(G) \leq M(G)$.
%\end{lemma}
%\begin{proof}
%Already proven above.
%\end{proof}
%Next we will amend $\W$ to include the following rule: when one endpoint has degree 1 and the other has %degree k, assign weight $\frac{1}{k}$.

Define $\W$ as follows:
\[\W=\sum_{(u,v)\in E} \W(u,v) \]
where
\[
\W(u,v)=\frac{1}{\max(\degree\lp u\rp, \degree\lp v\rp )+I[\min(\degree(u),\degree(v))> 1]}\]

We now show that $\W \leq M$. We have proven that $W(U)\leq (t-1)/2$ on any odd-size subgraph $U$, $|U| = t$.  Define subsets $U_1$ and $U_2$ of $U$ such that $U_1 \cup U_2 = U$.  $U_1$ is the set of all nodes in $U$ of degree 1 and all nodes in $U$ adjacent to a node of degree 1, and $U_2$ is the set of all other nodes in $U$.  Let $G(U_2)$ denote the subgraph of $U$ induced by $U_2$, and let $|U_1| = x$ where x is even.  Then $W(G(U_2)) = \W(G(U_2)) \leq (t-x-1)/2$.  So to complete our proof we must show that all remaining edges (call them $E'$) have total weight $\leq x/2$.

Assume WLOG that there are no isolated edges in $G$ (if there are, we can group them with $G(U_2)$ and retain the $(t-x-1)/2$ bound).

\begin{align*}
\W(E') & \leq \frac{1}{2} \sum_{u,v \in E'} \lp \frac{1}{\deg(u)}+\frac{1}{\deg(v)}\rp\\
& = \mathlarger{\sum_k} \Big(\sum_{v \in U_1, \deg(v) = k} \frac{\delta}{k} + \frac{k-\delta}{2(k+1)}\Big)
\end{align*}
where $\delta$ denotes the number of degree 1 nodes adjacent to $v$.\\
Let $f(\delta) = \delta/k + (k-\delta)/(2(k+1))$.  We are interested in finding the maximum value $f(\delta)/(\delta+1)$ can take on; if it can never take a value greater than $1/2$ then $\W(E')$ cannot be greater than $x/2$.
$$\frac{\partial\frac{f(\delta)}{\delta+1}}{\partial \delta} = \frac{2-k}{2k(\delta+1)^2}$$
which is always negative for $k \geq 2$.  So, $f(\delta)/(\delta+1)$ is maximized at $\delta = 0$, so  $f(\delta)/(\delta+1) \leq k/(2(k+1)) < 1/2.$

There is one final detail we have not considered: $x$ might be odd.  In this case, $|U_2|$ is even and we can't appeal to Theorem 3 to say that $\W(G(U_2)) \leq (t-x-1)/2$.  However, we can simply remove one node $\node'$ from $U_2$; the resulting odd-size subgraph has weight at most $(t-x-2)/2$.  Since $\W$ is a valid fractional matching, the weight assigned to all edges adjacent to $\node'$ cannot exceed 1, so we can say that $\W(G(U_2)) \leq (t-x)/2$.  Now we must show that $\W(E') \leq (x-1)/2$.

We have shown that the edge weight per node in $U_1$ cannot exceed $1/2$.  $\W(E')$ is minimized when there is exactly 1 node of degree 1, with a degree k neighbor. In this case, the only edge in $E'$ will be assigned weight $k/(2(k+1))< 1/2$.  $\W(E') \leq 1/2 + (x-2)/2 = (x-1)/2$ and the theorem is proven.
\end{proof}

\iffalse
Finally, we further amend $\W$ to include the following rule:  when one endpoint has degree 2 and the other degree 3, assign weight $\frac{1}{3}$.

\begin{lemma}
Define $\W(G)$ as follows:\\
For each $(u,v) \in G$,\\
if $\deg(u) = 1$, or if $\deg(u) = 2$ and $\deg(v) = 3$, $\W(u,v) =  \min(\frac{1}{\deg(u)},\frac{1}{\deg(v)})$.\\
else $\W(u,v) =  \min(\frac{1}{\deg(u)+1},\frac{1}{\deg(v)+1})$.

Then $\W(G) \leq M(G)$.
\end{lemma}
\begin{proof}
Omitted for space.
\end{proof}
\fi


%\end{document}

 % Vertex

\chapter{Mesh}

\iffalse
%% For double-blind review submission, w/o CCS and ACM Reference (max submission space)
%\documentclass[sigplan,review,anonymous]{acmart}\settopmatter{printfolios=true,printccs=false,printacmref=false}
%\documentclass[sigplan]{acmart}\settopmatter{printfolios=true,printccs=false,printacmref=false}
%% For double-blind review submission, w/ CCS and ACM Reference
%\documentclass[sigplan,review,anonymous]{acmart}\settopmatter{printfolios=true}
%% For single-blind review submission, w/o CCS and ACM Reference (max submission space)
%\documentclass[sigplan,review]{acmart}\settopmatter{printfolios=true,printccs=false,printacmref=false}
%% For single-blind review submission, w/ CCS and ACM Reference
%\documentclass[sigplan,review]{acmart}\settopmatter{printfolios=true}
%% For final camera-ready submission, w/ required CCS and ACM Reference
\documentclass[sigplan,screen]{acmart}\settopmatter{}

\setcopyright{acmlicensed}
\acmPrice{15.00}
\acmDOI{10.1145/3314221.3314582}
\acmYear{2019}
\copyrightyear{2019}
\acmISBN{978-1-4503-6712-7/19/06}
\acmConference[PLDI '19]{Proceedings of the 40th ACM SIGPLAN Conference on Programming Language Design and Implementation}{June 22--26, 2019}{Phoenix, AZ, USA}
\acmBooktitle{Proceedings of the 40th ACM SIGPLAN Conference on Programming Language Design and Implementation (PLDI '19), June 22--26, 2019, Phoenix, AZ, USA}


%% \startPage{1}

%% Bibliography style
\bibliographystyle{ACM-Reference-Format}
%% Citation style
%\citestyle{acmauthoryear}  %% For author/year citations
%\citestyle{acmnumeric}     %% For numeric citations
%\setcitestyle{nosort}      %% With 'acmnumeric', to disable automatic
                            %% sorting of references within a single citation;
                            %% e.g., \cite{Smith99,Carpenter05,Baker12}
                            %% rendered as [14,5,2] rather than [2,5,14].
%\setcitesyle{nocompress}   %% With 'acmnumeric', to disable automatic
                            %% compression of sequential references within a
                            %% single citation;
                            %% e.g., \cite{Baker12,Baker14,Baker16}
                            %% rendered as [2,3,4] rather than [2-4].


%%%%%%%%%%%%%%%%%%%%%%%%%%%%%%%%%%%%%%%%%%%%%%%%%%%%%%%%%%%%%%%%%%%%%%
%% Note: Authors migrating a paper from traditional SIGPLAN
%% proceedings format to PACMPL format must update the
%% '\documentclass' and topmatter commands above; see
%% 'acmart-pacmpl-template.tex'.
%%%%%%%%%%%%%%%%%%%%%%%%%%%%%%%%%%%%%%%%%%%%%%%%%%%%%%%%%%%%%%%%%%%%%%

\pdfinfoomitdate=1
\pdftrailerid{}

\usepackage[utf8]{inputenc}
\usepackage[english]{babel}

%\usepackage{minted}
%\RequirePackage[outputdir=latex.out,cache=false]{minted}
\RequirePackage[cache=false]{minted}
\usepackage{microtype}
\usepackage{hyperref}
\usepackage{xspace}
\usepackage{graphicx}
\usepackage{url}
\usepackage{amsmath}
\usepackage{amssymb}
\usepackage{relsize}
\usepackage{tikz}
\usepackage{clrscode3e}
\usepackage{amsthm}
\usepackage{color}

\newtheorem{observation}{Observation}%[section]


\newcommand{\sm}{\proc{SplitMesher}\xspace}

%% \setmonofont[Scale=MatchLowercase]{DejaVu Sans Mono}
%% \setmonofont{DejaVu Sans Mono}

\microtypecontext{spacing=nonfrench}
\microtypesetup{
  final,
  tracking=true,
  kerning=true,
  spacing=true,
  factor=1100,
  stretch=20,
  shrink=20
}
\SetTracking{encoding={*}, shape=sc}{0} % No tracking for smallcaps

\newcommand{\Mesh}{\textsc{Mesh}\xspace}
\newcommand{\mesh}{Mesh\xspace}
\newcommand{\ndegree}{\textup{d}} % Command \deg already defined by some dependency
\newcommand{\comments}[1]{[#1]}


%% Some recommended packages.
\usepackage{booktabs}   %% For formal tables:
                        %% http://ctan.org/pkg/booktabs
%\usepackage{subcaption} %% For complex figures with subfigures/subcaptions
                        %% http://ctan.org/pkg/subcaption
%\captionsetup[subfloat]{font={small,sf}}
% Settings for figures and tables. Figure captions are placed below the figure,
% while table captions are placed above the table. All captions are sans-serif.
\RequirePackage[font={normalsize,sf,bf}]{caption}
\RequirePackage[position=bottom]{subfig}
\captionsetup[table]{aboveskip=0.5em, belowskip=0.5em}
\captionsetup[figure]{aboveskip=0.5em, belowskip=0em}
\captionsetup[subfloat]{font={small,sf}}
\setcounter{topnumber}{2}
\setcounter{dbltopnumber}{2}
\setcounter{bottomnumber}{2}
\setcounter{totalnumber}{4}
\renewcommand{\topfraction}{0.85}
\renewcommand{\dbltopfraction}{0.9}
\renewcommand{\bottomfraction}{0.85}
\renewcommand{\textfraction}{0.07}
\renewcommand{\floatpagefraction}{0.85}
\renewcommand{\dblfloatpagefraction}{0.85}

\setlength{\floatsep}{0.5em plus 0.2em minus 0.2em}
\setlength{\dblfloatsep}{0.5em plus 0.2em minus 0.2em}
\setlength{\textfloatsep}{0.5em plus 0.2em minus 0.2em}
\setlength{\dbltextfloatsep}{0.5em plus 0.2em minus 0.2em}

% Utility packages for floats and tables.
\RequirePackage{float}
\RequirePackage{graphicx}
\RequirePackage{booktabs}
\RequirePackage{multirow}



\begin{document}

%% Title information
\title[Mesh]{\Mesh: Compacting Memory Management \\ for C/C++ Applications} % Unmanaged Languages}


%% Author information NOOP
%% Contents and number of authors suppressed with 'anonymous'.
%% Each author should be introduced by \author, followed by
%% \authornote (optional), \orcid (optional), \affiliation, and
%% \email.
%% An author may have multiple affiliations and/or emails; repeat the
%% appropriate command.
%% Many elements are not rendered, but should be provided for metadata
%% extraction tools.

%% Author with single affiliation.
\author{Bobby Powers}
%\authornote{with author1 note}          %% \authornote is optional;
                                        %% can be repeated if necessary
%\orcid{nnnn-nnnn-nnnn-nnnn}             %% \orcid is optional
\affiliation{
  %\position{Position1}
  \department{College of Information and Computer Sciences}              %% \department is recommended
  \institution{University of Massachusetts Amherst}            %% \institution is required
  \streetaddress{140 Governors Drive}
  \city{Amherst}
  \state{MA}
  \postcode{01003}
  \country{USA}                    %% \country is recommended
}
\email{bpowers@cs.umass.edu}          %% \email is recommended


%% Author with single affiliation.
\author{David Tench}
%\authornote{with author1 note}          %% \authornote is optional;
                                        %% can be repeated if necessary
%\orcid{nnnn-nnnn-nnnn-nnnn}             %% \orcid is optional
\affiliation{
  %\position{Position1}
  \department{College of Information and Computer Sciences}              %% \department is recommended
  \institution{University of Massachusetts Amherst}            %% \institution is required
  \streetaddress{140 Governors Drive}
  \city{Amherst}
  \state{MA}
  \postcode{01003}
  \country{USA}                    %% \country is recommended
}
\email{dtench@cs.umass.edu}          %% \email is recommended



%% Author with single affiliation.
\author{Emery D. Berger}
%\authornote{with author1 note}          %% \authornote is optional;
                                        %% can be repeated if necessary
%\orcid{nnnn-nnnn-nnnn-nnnn}             %% \orcid is optional
\affiliation{
  %\position{Position1}
  \department{College of Information and Computer Sciences}              %% \department is recommended
  \institution{University of Massachusetts Amherst}            %% \institution is required
  \streetaddress{140 Governors Drive}
  \city{Amherst}
  \state{MA}
  \postcode{01003}
  \country{USA}                    %% \country is recommended
}
\email{emery@cs.umass.edu}          %% \email is recommended

%% Author with single affiliation.
\author{Andrew McGregor}
%\authornote{with author1 note}          %% \authornote is optional;
                                        %% can be repeated if necessary
%\orcid{nnnn-nnnn-nnnn-nnnn}             %% \orcid is optional
\affiliation{
  %\position{Position1}
  \department{College of Information and Computer Sciences}              %% \department is recommended
  \institution{University of Massachusetts Amherst}            %% \institution is required
  \streetaddress{140 Governors Drive}
  \city{Amherst}
  \state{MA}
  \postcode{01003}
  \country{USA}                    %% \country is recommended
}
\email{mcgregor@cs.umass.edu}          %% \email is recommended

\fi

\input{Chapters/mesh/abstract}

\iffalse
%% 2012 ACM Computing Classification System (CSS) concepts
%% Generate at 'http://dl.acm.org/ccs/ccs.cfm'.
\begin{CCSXML}
<ccs2012>
<concept>
<concept_id>10011007.10010940.10010941.10010949.10010950.10010953</concept_id>
<concept_desc>Software and its engineering~Allocation / deallocation strategies</concept_desc>
<concept_significance>500</concept_significance>
</concept>
</ccs2012>
\end{CCSXML}

\ccsdesc[500]{Software and its engineering~Allocation / deallocation strategies}
%% End of generated code


\keywords{Memory management, runtime systems, unmanaged languages}


%% \maketitle
%% Note: \maketitle command must come after title commands, author
%% commands, abstract environment, Computing Classification System
%% environment and commands, and keywords command.
\maketitle


\fi

% TODO: make sure to talk about virtual compaction from Hound in related work!
% TODO: Make repo public, put in redacted citation.
% TODO: Rust maybe later
% TODO: mention evaluation results





\section{Introduction}
\label{sec:introduction}

\begin{comment}
Memory consumption is a serious concern across the spectrum of modern
computing platforms, from mobile to desktop to datacenters. For
example, on low-end Android devices, Google reports that more than 99
percent of Chrome crashes are due to running out of memory when
attempting to display a web page~\cite{hara:stateofblink}. On
desktops, the Firefox web browser has been the subject of a five-year
effort to reduce its memory footprint~\cite{awsy}. In datacenters,
developers implement a range of techniques from custom allocators to
other \emph{ad hoc} approaches in an effort to increase memory
utilization~\cite{jemalloc:exposehints,redis:announcement}.
%% while
%% Google reports that 10\% of CPU cycles in their clusters are spent on
%% memory allocation~\cite{kanev:2015:warehouse-scale}.
\end{comment}



\begin{comment}
A key challenge is that, unlike in garbage-collected environments,
automatically reducing a C/C++ application's memory footprint
via compaction is not possible. Because the addresses of allocated
objects are directly exposed to programmers, C/C++ applications can
freely modify or hide addresses.  For example, a program may stash
addresses in integers, store flags in the low bits of aligned
addresses, perform arithmetic on addresses and later reference them,
or even store addresses to disk and later reload them.  This hostile
environment makes it impossible to safely relocate objects: if an
object is relocated, all pointers to its original location must be
updated. However, there is no way to safely update \emph{every}
reference when they are ambiguous, much less when they are absent.

Existing memory allocators for C/C++ employ a variety of
best-effort heuristics aimed at reducing average
fragmentation~\cite{johnstone:1998:fragmentation}. However, these
approaches are inherently limited. In a classic result, Robson showed
that all such allocators can suffer from catastrophic
memory fragmentation~\cite{robson:1977:worstcasefrag}. This increase
in memory consumption can be as high as the $\log$ of the ratio
between the largest and smallest object sizes allocated. For example,
for an application that allocates 16-byte and 128KB objects, it is
possible for it to consume $13\times$ more memory than required.
\end{comment}

%Embedded systems designed for the Internet-of-Things (IoT), such as the Raspberry Pi Zero W, ship with wireless networking, 3D graphics, and complete operating system stacks but only hundreds of megabytes of memory~\cite{rpi:zero}, placing memory at a premium.

%% TODO: talk about / cite Detlefs paper on precise GC for C/C++

%% address manipulations.  XOR encoding of linked lists

% These addresses can then be freely manipulated: it is legal for C and C++ programs to

% Here, we define fragmentation as the
%amount of memory actually consumed divided by the amount of memory
%actually needed (the total live size).

\begin{comment}
  The result is that C and C++ programs can suffer from fragmentation
(both internal and external). In a setting where compaction
is possible, fragmentation can always be eliminated by squeezing out
space between objects. Since relocating objects is impossible,
fragmentation in C and C++ can become problematic.
\end{comment}

\begin{comment}
  % Push to related work
In languages like LISP, Java and
~\cite{hansen:1969:compaction,fenichel:1969:compaction}, the runtime
system can, as part of garbage collection, periodically compact memory
by moving live objects together. Compaction reduces the working set of
an application and can ensure that an application's footprint never
exceeds some pre-established maximum. Contemporary runtimes like the
Hotspot JVM~\cite{microystems2006memory}, the .NET
VM~\cite{microsoft:dotnet-gc}, and the SpiderMonkey JavaScript
VM~\cite{mozilla:spidermonkey-compaction} implement compaction as part
of their garbage collection algorithms.
\end{comment}


% additionally: build invalid addresses but never dereference them

% emacs pickle state?

%For example, on modern
%systems with a 4 KiB page size and a 16-byte minimum object size, this
%factor corresponds to a worst-case fragmentation of approximately
%$7\times$. In practice, of course, fragmentation is rarely so extreme,
%b
%% TODO: Memory remains one of the scarcest resources across the spectrum of modern computing devices, ranging from servers to desktops to mobile platforms.

%% Not sure - For example, Google reports that \emph{99 percent} of Chrome crashes on low-end Android devices are caused by it running out of memory when attempting to display the page~\cite{hara:stateofblink}.

%% TODO: cost of memory on Amazon servers!

Despite nearly fifty years of conventional wisdom indicating that
compaction is impossible in unmanaged languages, this chapter shows that
it is not only possible but also practical. It introduces
\Mesh, a memory allocator that effectively and efficiently performs
compacting memory management to reduce memory usage in unmodified
C and C++ applications.

Crucially and counterintuitively, \Mesh performs compaction without
relocation; that is, without changing the addresses of objects. This
property is vital for compatibility with arbitrary C/C++
applications. To achieve this, \Mesh{} builds on a mechanism which we
call \emph{meshing}, first introduced by Novark et al.'s Hound memory
leak detector~\cite{1542521}. Hound employed meshing in an effort to avoid
catastrophic memory consumption induced by its memory-inefficient
allocation scheme, which can only reclaim memory when every object on
a page is freed. Hound first searches for pages whose live objects do
not overlap. It then copies the contents of one page onto the other,
remaps one of the \emph{virtual} pages to point to the single
\emph{physical} page now holding the contents of both pages, and
finally relinquishes the other physical page to the
OS. Figure~\ref{fig:meshing} illustrates meshing in action.

\begin{figure*}[t!]
  \centering
  \subfloat[\textbf{Before:} these pages are candidates for
      ``meshing'' because their allocated objects do not
      overlap.\vspace{2em}]{
      \includegraphics[width=0.45\textwidth]{Chapters/mesh/figures/mesh-diagram-1}
      \label{pre-meshing}
  }
  ~~~~~
  \centering
  \subfloat[\textbf{After:} both virtual pages now point to the
      first physical page; the second page is now freed.]{
      \includegraphics[width=0.45\textwidth]{Chapters/mesh/figures/mesh-diagram-2}
      \label{post-meshing}
  }

  \caption{\textbf{\Mesh{} in action.} \Mesh{} employs novel
    randomized algorithms that let it efficiently find and then
    ``mesh'' candidate pages within \emph{spans} (contiguous 4K pages)
    whose contents do not overlap.  In this example, it increases
    memory utilization across these pages from 37.5\% to 75\%, and
    returns one physical page to the OS (via \texttt{munmap}),
    reducing the overall memory footprint. \Mesh{}'s randomized
    allocation algorithm ensures meshing's effectiveness with high
    probability.}

  \label{fig:meshing}
\end{figure*}

\Mesh{} overcomes two key technical challenges of meshing that previously made
it both inefficient and potentially entirely ineffective. First,
Hound's search for pages to mesh involves a linear scan of pages on
calls to \texttt{free}. While this search is more efficient than a
naive $O(n^2)$ search of all possible pairs of pages, it remains
prohibitively expensive for use in the context of a general-purpose
allocator. Second, Hound offers no guarantees that \emph{any} pages
would ever be meshable.  Consider an application that happens to
allocate even one object in the same offset in every page. That layout
would preclude meshing altogether, eliminating the possibility of
saving any space.

\Mesh makes meshing both efficient and provably effective (with high
probability) by combining it with two novel randomized
algorithms. First, \Mesh uses a space-efficient randomized
allocation strategy that effectively scatters objects within each
virtual page, making the above scenario provably exceedingly
unlikely. Second, \Mesh incorporates an efficient randomized
algorithm that is guaranteed with high probability to quickly find
candidate pages that are likely to mesh. These two algorithms work in
concert to enable formal guarantees on \Mesh's effectiveness. Our
analysis shows that \Mesh breaks the above-mentioned Robson worst
case bounds for fragmentation with high
probability~\cite{robson:1977:worstcasefrag}, as memory reclaimed by meshing is available for use by any size class.
This ability to redistribute memory from one size class to another enables Mesh to adapt to changes in an application's allocation behavior in a way other segregated-fit allocators cannot.


We implement \Mesh as a library for C/C++ applications running on
Linux or Mac OS X. \Mesh{} interposes on memory management operations,
making it possible to use it without code changes or
recompilation by setting the appropriate environment variable to load
the \Mesh{} library (e.g., \texttt{export
  LD\_PRELOAD=libmesh.so} on Linux). Our evaluation demonstrates that
our implementation of \Mesh{} is both fast and efficient in
practice. It generally matches the performance of state-of-the-art
allocators while guaranteeing the absence of catastrophic
fragmentation with high probability. In addition, it occasionally
yields substantial space savings: replacing the standard allocator
with \Mesh{} automatically reduces memory consumption by 16\%
(Firefox) to 39\% (Redis).
%In
%general, the longer-lived and more memory-intensive the application,
%the more memory \Mesh can save.


\subsection{Contributions}
\label{sec:contributions}

This chapter describes the \Mesh system, focusing on its core meshing algorithm.  It presents theoretical results that guarantee \Mesh{}'s efficiency and effectiveness with high probability (\S\ref{sec:theory}).  Other components of the \Mesh system design and empirical evaluation of its performance are briefly summarized to contextualize the algorithm and analysis.

\begin{comment}
\begin{itemize}

\item It introduces \textbf{\Mesh}, a novel memory allocator that acts
  as a plug-in replacement for \texttt{malloc}. \Mesh{} combines
  remapping of virtual to physical pages (meshing) with randomized
  allocation and search algorithms to enable safe and effective
  \emph{compaction without relocation} for C/C++
  (\S\ref{sec:meshing}, \S\ref{sec:algorithms},
  \S\ref{sec:allocator}).

\item It presents theoretical results that guarantee \Mesh{}'s
    efficiency and effectiveness with high probability (\S\ref{sec:theory}).

\item It evaluates \Mesh{}'s performance empirically, demonstrating \Mesh{}'s ability to reduce
    space consumption while generally imposing low runtime
    overhead (\S\ref{sec:evaluation}).

\end{itemize}
\end{comment}

\section{Overview}
\label{sec:meshing}

\begin{comment}
  \begin{figure}[!t]
\centering
\includegraphics[width=.3\textwidth]{figures/bitmap_bitstring}
\caption{The bitmaps managing the allocated space in a span
  (visualized as allocated objects in the span, top) can be
  represented as bitstrings of 0s and 1s (bottom), where a 1
  corresponds to an allocated object and 0 to free space.}
\label{fig:bitmap-bitstring}
\end{figure}
\end{comment}

This section provides a high-level overview of how \Mesh{} works and
gives some intuition as to how its algorithms and implementation
ensure its efficiency and effectiveness, before diving into detailed
description of \Mesh{}'s algorithms (\S\ref{sec:algorithms}),
implementation (\S\ref{sec:allocator}), and its theoretical analysis
(\S\ref{sec:theory}).


\subsection{Remapping Virtual Pages}

\Mesh{} enables compaction without relocating object addresses; it
depends only on hardware-level virtual memory support, which is
standard on most computing platforms like x86 and ARM64. \Mesh{} works
by finding pairs of pages and merging them together \emph{physically}
but not \emph{virtually}: this merging lets it relinquish
physical pages to the OS.

Meshing is only possible when no objects on the pages occupy the same
offsets.  A key observation is that as fragmentation increases (that
is, as there are more free objects), the likelihood of successfully
finding pairs of pages that mesh also increases.

Figure~\ref{fig:meshing} schematically illustrates the meshing
process. \Mesh{} manages memory at the granularity of \textit{spans},
which are runs of contiguous 4K pages (for purposes of illustration,
the figure shows single-page spans). Each span only contains
same-sized objects. The figure shows two spans of memory with low
utilization (each is under $40\%$ occupied) and whose allocations are at non-overlapping offsets.


\begin{comment}

More formally, we can say that $t$ spans
\textit{\mesh} if and only if the sum of each offset across the $t$
spans sum to 1 or less:

\begin{align}
  \forall k \in [0, b-1]. \sum_{0 \leq i \leq t} s_i[k] \leq 1
\end{align}

where $b$ is span length, and $s_i[k] = 1$ iff there is an object at the $k$th offset of span i.
\end{comment}

Meshing consolidates allocations from each span onto one physical span.
Each object in the resulting meshed span resides at the same offset as
it did in its original span; that is, its virtual addresses are
preserved, making meshing invisible to the application. Meshing then
updates the virtual-to-physical mapping (the page tables) for the
process so that both virtual spans point to the same physical
span. The second physical span is returned to the OS.  When average
occupancy is low, meshing can consolidate many pages, offering the
potential for considerable space savings.
% consolidated to a single span and all other spans can be reused.


\subsection{Random Allocation}

A key threat to meshing is that pages could contain objects at the
same offset, preventing them from being meshed. In the worst case, all
spans would have only one allocated object, each at the same offset,
making them non-meshable. \Mesh{} employs randomized allocation to
make this worst-case behavior exceedingly unlikely. It allocates
objects uniformly at random across all available offsets in a span. As
a result, the probability that all objects will occupy the same offset
is $\left({1}/{b}\right)^{n-1}$, where $b$ is the number of objects in
a span, and $n$ is the number of spans.

%Meshing employs random allocation to minimize the likelihood of this happening too often. The use of a randomized algorithm lets us reason about the effectiveness of meshing in expectation.

%Meshing introduces a limited amount of randomness to ensure that we
%can reason about meshing spans in expectation.

In practice, the resulting probability of being unable to mesh many
pages is vanishingly small. For example, when meshing 64 spans with
one 16-byte object allocated on each (so that the number of objects
$b$ in a 4K span is $256$), the likelihood of being unable to mesh any
of these spans is $10^{-152}$. To put this into perspective, there are
estimated to be roughly $10^{82}$ particles in the universe.

We use randomness to guide the design of \Mesh{}'s algorithms
(\S\ref{sec:algorithms}) and implementation (\S\ref{sec:allocator});
this randomization lets us prove robust guarantees of its performance
(\S\ref{sec:theory}), showing that \Mesh{} breaks the Robson bounds with
high probability.

% ,
%subject to an assumption that application allocation behavior does not
%depend on the addresses returned by the allocator.

%% TODO: should insert a treatment of Emery's ideas about address obliviousness somewhere.


\subsection{Finding Spans to Mesh}

Given a set of spans, our goal is to mesh them in a way that frees as
many physical pages as possible. We can think of this task as that of
partitioning the spans into subsets such that the spans in each subset
mesh. An optimal partition would minimize the number of such subsets.

%% TODO: fix ref to point at exact subsection in algorithms.

Unfortunately, as we show, optimal meshing is not feasible
(\S\ref{sec:theory}). Instead, the algorithms in
Section~\ref{sec:algorithms} present practical methods for finding
high-quality meshes under real-world time constraints. We show that
solving a simplified version of the problem (\S\ref{sec:algorithms})
is sufficient to achieve reasonable meshes with high probability
(\S\ref{sec:theory}).


\begin{comment}
Memory fragmentation, introduced in Section~\ref{sec:introduction},
occurs when the ratio of program-allocated memory to operating-system
reserved memory becomes high:

\begin{align*}
\text{Fragmentation} = \frac{\text{Reserved}}{\text{Allocated}}
\end{align*}

\end{comment}

%%\textit{XXX: I think it is worth noting that an allocator that stored a very high amount of metadata per allocation (e.g. tracking the   meshability graph explicitly) would be indistinguishable from a   highly-fragmented allocator from the perspective of the OS.  Not   sure how to work that in.}

%% Informally, fragmentation occurs because the application manages
%% memory in bytes, while the operating system manages memory at page
%% granularity -- 4 KiB on most current architectures.  Sparsely allocated
%% application addresses ``pin down'' otherwise vacant OS pages.

%% Many managed languages like Java and JavaScript do not experience
%% memory fragmentation due to the combination of language soundness and
%% garbage collection implementations.  Languages are designed in such a
%% way that implementations can enumerate and update all object
%% references, enabling garbage collectors to periodically compact live
%% objects.  This isn't possible for unmanaged languages like C, C++ and
%% Rust, as it is not possible to soundly enumerate all object
%% references.


%% XXX: we should unify talking about ``objects'' and ``allocations''
%% -- allocations is probably better for unmanaged languages where
%% objects have a specific interpretation.

%% Meshing works on \textit{spans} -- contiguous regions of memory where
%% the size of a span is a multiple of the page size between 4 KiB and
%% 128 KiB.  Each span allocates objects of a single-size only, for
%% example a 4 KiB span might hold 32 objects of size 128 bytes.  We can
%% represent a span by a \textit{bitstring}, as in
%% Figure~\ref{fig:bitmap-bitstring}, a string with a 1 for an allocated
%% object at that offset from the start of the span and 0 otherwise.  The
%% length of the bitstring is the number of objects that span holds.

%% This definition characterizes the constraints of the technique by
%% which meshing is possible, wherein two or more spans are meshed or
%% "stacked" on top of each other.

%% The layout and management of a program's heap guide how we consider
%% meshing.  In a running program, the heap is managed as a number of
%% different \textit{size classes} along with a region consisting of
%% large allocations.  Allocations are fulfilled from the smallest size
%% class they fit in (e.g. an allocation request for 50 bytes is
%% satisfied by the 64-byte size class), and objects larger than 16 KiB
%% are individually served from the large allocation region.

%% We treat each size class as an independent instance of the meshing
%% problem, and large allocations are not meshed.  As large allocations
%% are all many multiples of the page size significant fragmentation
%% between them does not exist.  The number of size-classes is fixed at
%% compilation time and constant during the execution of a program.

%% From here, we consider meshing as dealing with a single size-class,
%% and refer to all spans within this size class as $S$.  If we want to
%% mesh the entire heap, this means solving $n$ instances of the meshing
%% problem, where $n$ is the number of size classes.


%% Finally, meshing relies on the fact that there are two types of spans,
%% virtual and physical.  A \textit{virtual} span refers to the memory
%% addresses visible to the program being executed, while a
%% \textit{physical} span corresponds to the area in memory where
%% allocated objects live.  Meshing is concerned with minimizing the
%% count of in-use physical spans without modifying or moving virtual
%% spans.  As noted in Section~\ref{sec:introduction}, we cannot change
%% or modify virtual addresses returned from the allocator, as we do not
%% have a way to enumerate and update all references the program has
%% stored.  Since the last two digits of a virtual address directly
%% specify the offset of the referenced object in the span, objects
%% cannot be safely relocated to a different offset, leading to our

%% Allocated objects and free space within a span are tracked by the
%% memory allocator as a bitmap (see Section~\ref{sec:allocator}) -- the
%% in-memory representation of a bitstring.  For example, for objects of
%% size 32 and a span size of 4 KiB, the span can hold 128 32-byte
%% objects, so allocated objects are be tracked with a 128-bit bitmap.
%% Each allocated object in a running program has a unique
%% \textit{(bitmap, offset)} tuple.  Bitmaps are between 8 and 256-bits
%% in length.

%% We can therefore think of the meshing problem as that of partitioning
%% a set of equal-length bitstrings such that the bitstrings in each subset
%% mesh.  An optimal partition would minimize the number of such subsets.
%% This abstraction allows us to analyze the complexity of this computational
%% task in Section~\ref{sec:theory}.


\section{Algorithms \& System Design}
\label{sec:algorithms}

%This section provides an overview of \Mesh{}'s algorithms.
%Section~\ref{sec:allocator} provides a detailed description of \Mesh's
%implementation, while Section~\ref{sec:theory} presents a theoretical
%analysis of its effectiveness.

\Mesh{} comprises three main algorithmic components: allocation
(\S\ref{sec:allocation-algorithm}), deallocation
(\S\ref{sec:deallocation-algorithm}), and finding spans to mesh
(\S\ref{sec:meshing-algorithm}). Unless otherwise noted and without
loss of generality, all algorithms described here are per size class
(within spans, all objects are same size).

\subsection{Allocation}
\label{sec:allocation-algorithm}
% We want out allocator to enforce random spread of objects so that we can probabilistically guarantee meshing.  Also we want it to reuse memory when available.

Allocation in \Mesh{} consists of two steps: (1) finding a span to
allocate from, and (2) randomly allocating an object from that span.
\Mesh{} always allocates from a thread-local shuffle vector -- a
randomized version of a freelist %(described in detail in\S\ref{sec:shuffle-freelists}). 
The shuffle vector contains offsets
corresponding to the slots of a single span.  We call that span the
\emph{attached} span for a given thread.

If the shuffle vector is empty, \Mesh relinquishes the current
thread's attached span (if one exists) to the \emph{global heap}
(which holds all unattached spans), and asks it to select a new
span. If there are no partially full spans, the global heap returns a
new, empty span.  Otherwise, it selects a partially full span for
reuse. To maximize utilization, the global heap groups spans into bins
organized by decreasing occupancy (e.g., 75-99\% full in one bin,
50-74\% in the next). The global heap scans for the first non-empty
bin (by decreasing occupancy), and randomly selects a span from that
bin.

Once a span has been selected, the allocator adds the offsets
corresponding to the free slots in that span to the thread-local
shuffle vector (in a random order). \Mesh{} pops the first entry off
the shuffle vector and returns it.


\subsection{Deallocation}
\label{sec:deallocation-algorithm}

Deallocation behaves differently depending on whether the free is
local (the address belongs to the current thread's attached span),
remote (the object belongs to another thread's attached span), or if
it belongs to the global heap.

For local frees, \Mesh{} adds the object's offset onto the span's
shuffle vector in a random position and returns. For remote frees,
\Mesh{} atomically resets the bit in the corresponding index in a
bitmap associated with each span. Finally, for an object belonging to
the global heap, \Mesh{} marks the object as free, updates the span's
occupancy bin; this action may additionally trigger meshing.
%, which we
%describe below.

%% Global frees similarly mark the space backing an allocation as free and open for reuse in the owning span, but additionally may trigger meshing.  Meshing happens after a global free if it has been more than a set period of time since the last meshing (settable both at program startup and later by the program through the \texttt{mallctl} API, by default a tenth of a second, and if the allocator thinks there is a reasonable chance of meshing reclaiming space.

%% TODO: free -- right now, we assume unlimited virtual addresses; max limit of ranges in kernel (discuss in implementation)


\subsection{Meshing}
\label{sec:meshing-algorithm}

When meshing, \Mesh{} randomly chooses pairs of spans and attempts to
mesh each pair. The meshing algorithm, which we call \sm
(Figure~\ref{fig:meshalg}), is designed both for practical
effectiveness and for its theoretical guarantees.  The parameter $t$,
which determines the maximum number of times each span is probed (line
\ref{li:outerloop}), enables space-time trade-offs. The parameter $t$
can be increased to improve mesh quality and therefore reduce space,
or decreased to improve runtime, at the cost of sacrificed meshing
opportunities. We empirically found that $t=64$ balances runtime and
meshing effectiveness, and use this value in our implementation.

\sm proceeds by iterating through $S_l$ and checking whether it can
mesh each span with another span chosen from $S_r$ (line
\ref{li:condition}).  If so, it removes these spans from their
respective lists and meshes them (lines
\ref{li:remove}--\ref{li:mesh}). \sm repeats until it has checked $t *
|S_l|$ pairs of spans; \S\ref{sec:meshing-implementation} describes
the implementation of \sm in detail.

\begin{comment}
  At this point, each span has been checked exactly once.  All spans not
removed from the lists are still candidates for meshing.  \sm loops
again through the lists, in the $j$th step comparing the $j$th span in
$S_l$ with the $j+1$th span in $S_r$.  Again, whenever it compares a
pair of spans that can mesh, it removes them from their respective
lists and meshes them.

\sm repeats this looping process $t$ times.  On the $i$th loop through
the lists, the $j$th element in $S_l$ is compared with the $j+i-1$th
element in $S_r$.  After this outer loop (line~\ref{li:outerloop}) has
completed, any remaining span has been compared with $t$ other spans,
each time failing to mesh. \sm stops here and makes no further effort
to mesh these spans.
\end{comment}


\iffalse
\begin{algorithm}[tb]
  \LinesNumbered
  \SetKwData{Frontier}{frontier}\SetKwData{This}{this}\SetKwData{OldNode}{$n_{old}$}\SetKwData{NewNode}{$n_{new}$}
  \SetKw{Continue}{continue} \SetKw{And}{and}
  \SetKwData{Null}{null}
  \SetKwData{Path}{path}
  \SetKwData{OldEdges}{$E_{old}$}
  \SetKwData{NewEdges}{$E_{new}$}
  \SetKwData{OldEdge}{$e_{old}$}
  \SetKwData{NewEdge}{$e_{new}$}
  \SetKwData{Length}{Length}
  \SetKwData{True}{true} \SetKwData{False}{false}
  \SetKwFunction{MakeTuple}{MakeTuple}
  \SetKwFunction{GetRoot}{GetRoot}\SetKwFunction{Enqueue}{enqueue}\SetKwFunction{Length}{length}\SetKwFunction{Dequeue}{dequeue}
  \SetKwFunction{HasVisited}{HasVisited}
  \SetKwFunction{MarkAsVisited}{MarkAsVisited}
  \SetKwFunction{MarkAsGrowing}{MarkAsGrowing}
  \SetKwFunction{MarkNotGrowing}{MarkNotGrowing}
  \SetKwFunction{RecordGrowingPath}{RecordGrowingPath}
  \SetKwFunction{IsGrowing}{IsGrowing}
  \SetKwInOut{Input}{Input}\SetKwInOut{Output}{Output}
  \Input{$G_{final} = (N, E)$}
  \Frontier$\leftarrow$ []\;
  $n_{root} \leftarrow$ \GetRoot{$G_{final}$}\;
  \For{$e = (n_1, n2) \in E : n_1 = n_{root}$}{
    \Frontier.\Enqueue{\MakeTuple{\Null, $e$}}
  }
  \While{\Frontier.\Length $\neq 0$}{
    $T \leftarrow$ \Frontier.\Dequeue{}\;
    $(T_{prev}, e) \leftarrow T$\;
    \eIf{\HasVisited{$e$}}{
      \Continue\;
    }{
      \MarkAsVisited{$e$}\;
    }
    $(n_{from}, n_{to}) \leftarrow e$\;

    \If{\IsGrowing{$n_{to}$}}{
      \RecordGrowingPath{$T$}\;
    }
    \ForEach{$e=(n_1,n_2) \in E : n_1 = n_{to}$}{
      \Frontier.\Enqueue{\MakeTuple{$T$, $e$}}\;
    }
  }
  \caption{FindLeakPaths, which records paths through the heap to leaking nodes. RecordGrowingPath recovers the entire path to the edge $e$ by following the linked list formed by the tuple $T$.}
  \label{algo:findpaths}
\end{algorithm}
\fi

\iffalse
\begin{algorithm}[tb]
  \SetAlgorithmName{\sm}{}{}
  \SetKwInOut{Input}{Input}
  \SetKwInOut{Output}{Output}
  \Input{randomly ordered list of spans S, $|S| = n$; parameter $t$}

  M $ \leftarrow$ []\;

  $S_l,$ $S_r = S[1:\frac{n}{2}],$ $S[\frac{n}{2}+1 : n]$\;
  \ForEach{$i \in [0, t-1]$}{
    len = length$(S_l)$\;
    \ForEach{$j \in [0, \frac{n}{2}-1]$}{
      \If{$S_l(j)$ and $S_r(j+i$ mod len) mesh}{
        $S_l \leftarrow S_l \textbackslash S_l(j)$\;
        $S_r \leftarrow S_r \textbackslash S_r(j+i$ mod len)\;
        M $\leftarrow$ M $\cup$ $S_l(j)$ $\cup$ $S_r(j+i$ mod len)\;
      }
    }
  }
  \Return M\;
  \TitleOfAlgo{Procedure splits span set into halves and probes for meshable pairs between halves.}
%  \label{alg:mesher}
\end{algorithm}
\fi

\begin{figure}[!t]
\begin{codebox}
    \Procname{$\sm(S,t)$}
%    \li M $\gets$ []
    \li $n \gets$ length$(S)$
    \li $S_l,$ $S_r \gets S[1:n/2],$ $S[n/2 +1 : n]$
    \li \For $(i = 0, i<t, i++)$ \label{li:outerloop}
        \li \Do
            $\mbox{len} = |S_l|$\;
            \li \For $(j = 0, j < \mbox{len}, j++)$ \label{li:innerloop}
                \li \Do
                    \If \proc{Meshable} $(S_l(j)$, $S_r(j+i$ \% $\mbox{len}$)) \Then \label{li:condition}
                        \li $S_l \leftarrow S_l \setminus S_l(j)$ \label{li:remove}
                        \li $S_r \leftarrow S_r \setminus S_r(j+i$ \% $\mbox{len}$)
%                        \li M $\leftarrow$ M $\cup$ $S_l(j)$ $\cup$ $S_r(j+i$ mod len)
                        \li \proc{mesh}($ S_l(j)$, $S_r(j+i$ \% $\mbox{len}$)) \label{li:mesh}
                    \End
                \End
        \End
%    \li \Return M
\end{codebox}
\caption{\textbf{Meshing random pairs of spans.} \sm splits the randomly ordered span list $S$ into halves, then probes pairs between halves for meshes.  Each span is probed up to $t$ times.}
\label{fig:meshalg}
\end{figure}


\subsection{Implementation}
\label{sec:allocator}

We implement \Mesh as a drop-in replacement memory allocator that
implements meshing for single or multi-threaded applications written
in C/C++. Its current implementation work for 64-bit Linux and Mac OS
X binaries. \Mesh can be explicitly linked against by passing
\texttt{-lmesh} to the linker at compile time, or loaded dynamically
by setting the \texttt{LD\_PRELOAD} (Linux) or
\texttt{DYLD\_INSERT\_LIBRARIES} (Mac OS X) environment variables to
point to the \Mesh{} library. When loaded, \Mesh interposes on
standard libc functions to replace all memory allocation functions.

\begin{comment}
\begin{figure}
  \includegraphics[width=.5\textwidth]{figures/global_heap}
  \caption{\textbf{Heap organization.} \Mesh's internal allocator
    manages memory for \Mesh-internal dynamic data structures.
    The global heap manages both large objects and the meshable arena that spans are allocated from.  Each thread has a
    local heap which satisfies small allocations from MiniHeaps (one
    per size class).}
  \label{fig:global-heap}
\end{figure}
\end{comment}

\Mesh combines traditional allocation strategies with meshing to
minimize heap usage.  Like most modern memory
allocators~\cite{Novark:2010:DSH:1866307.1866371,1134000,379232,evans2006scalable,ghemawattcmalloc},
\Mesh is a segregated-fit allocator. \Mesh{} employs fine-grained size
classes to reduce internal fragmentation due to rounding up to the
nearest size class. \Mesh{} uses the same size classes as those
used by jemalloc for objects 1024 bytes and
smaller~\cite{evans2006scalable}, and power-of-two size classes for
objects between 1024 and 16K.  Allocations are fulfilled from the
smallest size class they fit in (e.g., objects of size 33--48 bytes
are served from the 48-byte size class); objects larger than 16K are
individually fulfilled from the global arena.  Small objects are
allocated out of \textit{spans} (\S\ref{sec:meshing}), which are
multiples of the page size and contain between 8 and 256 objects of a
fixed size.  Having at least eight objects per span helps
amortize the cost of reserving memory from the global manager for
the current thread's allocator.

Objects of 4KB and larger are always page-aligned and span at least one
entire page. \Mesh does not consider these objects for meshing;
instead, the pages are directly freed to the OS.

\Mesh's heap organization consists of four main components.
\emph{MiniHeaps} track occupancy and other metadata for spans.
%(\S\ref{sec:miniheaps}).  
\textit{Shuffle vectors} enable efficient,
random allocation out of a MiniHeap. %(\S\ref{sec:shuffle-freelists}).
\textit{Thread local heaps} satisfy small-object allocation requests
without the need for locks or atomic operations in the common case.
%(\S\ref{sec:thread-local-heaps}). 
Finally, the \textit{global heap}
%(\S\ref{sec:global-heap})
 manages runtime state shared by all threads,
large object allocation, and coordinates meshing operations.
%(\S\ref{sec:meshing-implementation}).
We omit detailing discussion of these components of \Mesh in this document.


\iffalse

\subsection{MiniHeaps}
\label{sec:miniheaps}

MiniHeaps manage allocated physical spans of memory and are either
\emph{attached} or \emph{detached}.  An attached MiniHeap is owned by
a specific thread-local heap, while a detached MiniHeap is only
referenced through the global heap.  New small objects are
\textit{only} allocated out of attached MiniHeaps.

Each MiniHeap contains metadata that comprises span length, object
size, allocation bitmap, and the start addresses of any virtual spans
meshed to a unique physical span.  The number of objects that can be
allocated from a MiniHeap bitmap is \textit{objectCount = spanSize /
  objSize}.  The allocation bitmap is initialized to
\textit{objectCount} zero bits.

When a MiniHeap is attached to a
thread-local \emph{shuffle vector} (\S\ref{sec:shuffle-freelists}),
each offset that is unset in the MiniHeap's bitmap is added to the
shuffle vector, with that bit now atomically set to one in the bitmap.
This approach is designed to allow multiple threads to free objects
which keeping most memory allocation operations local in the common
case.

When an object is freed and the free is non-local
(\S\ref{sec:deallocation-algorithm}), the bit is reset.  When a new
MiniHeap is allocated, there is only one virtual span that points to
the physical memory it manages. After meshing, there may be multiple
virtual spans pointing to the MiniHeap's physical memory.


\begin{figure}[!t]
  \centering
  \subfloat[A shuffle vector for a span of size 8, where no objects have
      yet been allocated.]{
      \includegraphics[width=.4\textwidth]{Chapters/mesh/figures/shuffle-freelist_a}
  }
  \vspace{-0em}
  \subfloat[The shuffle vector after the first object has been allocated.]{
      \includegraphics[width=.4\textwidth]{Chapters/mesh/figures/shuffle-freelist_b}
  }
  \vspace{-0em}
  \subfloat[On \texttt{free}, the object's offset is pushed onto the
      front of the vector, the allocation index is updated, and the
      offset is swapped with a randomly chosen offset.]{
      \includegraphics[width=.4\textwidth]{Chapters/mesh/figures/shuffle-freelist_c}
  }
  \vspace{-0em}
  \subfloat[Finally, after the swap, new allocations proceed in a
      bump-pointer like fashion.]{
    \centering
    \includegraphics[width=.4\textwidth]{Chapters/mesh/figures/shuffle-freelist_d}
  }
  \vspace{0.5em}
  \caption{\textbf{Shuffle vectors} compactly enable fast random allocation.
    Indices (one byte each) are maintained in random order; allocation is
    popping, and deallocation is pushing plus a random swap (\S\ref{sec:shuffle-freelists}).}
    % When an object is freed and the MiniHeap     it was allocated from is still attached to the local thread,    its
  \label{fig:shuffle-freelists}
\end{figure}


%% TODO: restore locality section if and when we have locality results
\begin{comment}
\subsubsection{Locality}

At first blush, it may appear that \Mesh's use of randomization would
degrade locality and thus degrade performance. Like DieHard (but unlike
many other allocators), \Mesh does not seek to allocate consecutive
objects nearby. However, we expect this approach to have minimal
impact on locality in many cases. First, the increasing size of cache
lines means that inter-object locality is not a concern for most
objects. Modern 64-bit Intel processors have 64-byte cache lines, and
ARM devices have 32 or 64-byte cache lines, meaning that object
locality has little impact on L1 cache locality for objects of that
size or larger. We also expect TLB pressure to be unaffected compared
to other segregated-fit allocators with the same choice of size
classes; \Mesh allocates randomly \emph{within} spans. % \textbf{XXX forward ref to evaluation; we want to say primary cost is allocation mechanism, not locality effects.}
\end{comment}

\subsection{Shuffle Vectors}
\label{sec:shuffle-freelists}

Shuffle vectors are a novel data structure that lets \Mesh
perform randomized allocation out of a MiniHeap efficiently and
with low space overhead.

Previous memory allocators that have employed randomization (for
security or reliability) perform randomized allocation by random
probing into
bitmaps~\cite{1134000,Novark:2010:DSH:1866307.1866371}. In these
allocators, a memory allocation request chooses a random number in the
range $[0,\text{\textit{objectCount}}-1]$. If the associated bit is
zero in the bitmap, the allocator sets it to one and returns the
address of the corresponding offset. If the offset is already one,
meaning that the object is in use, a new random number is chosen and
the process repeated. Random probing allocates objects in $O(1)$
\emph{expected} time but requires overprovisioning memory by a
constant factor (e.g., $2\times$ more memory must be allocated than
needed). This overprovisioning is at odds with our goal of
\emph{reducing} space overhead.

Shuffle vectors solve this problem, combining low space overhead with
worst-case $O(1)$ running time for \texttt{malloc} and
\texttt{free}. Each comprises a fixed-size array consisting of all the
offsets from a span that are \textit{not} already allocated, and an
allocation index representing the head. Each vector is initially
randomized with the Knuth-Fischer-Yates
shuffle~\cite{knuth:1981:semi}, and its allocation index is set to
0. Allocation proceeds by selecting the next available number in the
vector, ``bumping'' the allocation index and returning the
corresponding address. Deallocation works by placing the freed object
at the front of the vector and performing one iteration of the shuffle
algorithm; this operation preserves randomization of the
vector. Figure~\ref{fig:shuffle-freelists} illustrates this process,
while Figure~\ref{fig:malloc} has pseudocode listings for
initialization, allocation, and deallocation.

%\vspace{-1.2em}
%\begin{align*}
 % \text{\textit{ptr}} = \text{\textit{spanStart + off*objSize}}
%\end{align*}

%This freelist is initialized with all of the available offsets $[0
%  .. $ objectCount $]$ and then sorted

%When an object is freed and the owning MiniHeap is still in use, a
%single iteration of the shuffle is performed to re-add the newly freed
%offset back to the list of available offsets.

Shuffle vectors impose far less space overhead than random
probing. First, with a maximum of 256 objects in a span, each offset
in the vector can be represented as an unsigned character (a single
byte). Second, because \Mesh needs only one shuffle vector per
attached MiniHeap, the amount of memory required for vectors is
$256c$, where $c$ is the number of size classes (24 in the current
implementation): roughly 2.8K per thread.  Finally, shuffle vectors
are only ever accessed from a single thread, and so do not require
locks or atomic operations.  While bitmaps must be operated on
atomically (frees may originate at any time from other threads),
shuffle vectors are only accessed from a single thread and do not
require synchronization or cache-line flushes.

\subsection{Thread Local Heaps}
\label{sec:thread-local-heaps}

All malloc and free requests from an application start at the thread's
local heap. Thread local heaps have shuffle vectors for each size
class, a reference to the global heap, and their own thread-local
random number generator.

Allocation requests are handled differently depending on the size of
the allocation.  If an allocation request is larger than 16K, it is
forwarded to the global heap for fulfillment
(\S\ref{sec:global-heap}).  Allocation requests 16K and smaller are
small object allocations and are handled directly by the shuffle
vector for the size class corresponding to the allocation request, as
in Figure~\ref{fig:malloc}a.  If the shuffle vector is empty, it is
refilled by requesting an appropriately sized MiniHeap from the global
heap.  This MiniHeap is a partially-full MiniHeap if one exists, or
represents a freshly-allocated span if no partially full ones are
available for reuse.  Frees, as in Figure~\ref{fig:malloc}d, first
check if the object is from an attached MiniHeap.  If so, it is
handled by the appropriate shuffle vector, otherwise it is passed to
the global heap to handle.

\subsection{Global Heap}
\label{sec:global-heap}

The global heap allocates MiniHeaps for thread-local heaps, handles
all large object allocations, performs non-local frees for both small
and large objects, and coordinates meshing.

\subsubsection{The Meshable Arena}
\label{sec:meshable-arena}

The global heap allocates meshable spans and large objects from a
single, global meshable arena. This arena contains two sets of bins
for same-length spans --- one set is for demand zero-ed spans (freshly
\texttt{mmap}ped), and the other for used spans --- and a mapping of
page offsets from the start of the arena to their owning MiniHeap
pointers.  Used pages are not immediately returned to the OS as they
are likely to be needed again soon, and reclamation is relatively
expensive. Only after 64MB of used pages have accumulated, or whenever
meshing is invoked, \Mesh{} returns pages to OS by calling
\texttt{fallocate} on the heap's file descriptor
(\S\ref{sec:page-table-updates}) with the
\texttt{FALLOC\_FL\_PUNCH\_HOLE} flag.

\begin{comment}
An allocation request for $k$ pages is fulfilled by searching through
the bins of free spans from index $k-1$ to $k_{\text{max}}$ for a free
span of at-least size $k$.  Bins smaller than $k_{\text{max}}$ only
contain spans of a single size, while $k_{\text{max}}$ contains spans
of length 256 or larger.  If no spans are available, the arena is
extended, the extension placed in $k_{\text{max}}$, and bins are
re-searched.  Once a resulting span has been found, if the span is
larger than the $k$ pages requested it is broken in two with the
remainder span placed back in an appropriate bin.  This scheme favors
dirty pages over clean pages to minimize the process's RSS.
\end{comment}

% TODO: mention how this is also how aligned allocations work

\begin{figure}[!t]
  \centering
  \subfloat{\lstinputlisting[language=C++, basicstyle=\footnotesize]{Chapters/mesh/all-code1}}
  \centering
  \subfloat{\lstinputlisting[language=C++, basicstyle=\footnotesize]{Chapters/mesh/all-code2}}
  %\input{Chapters/mesh/all-code1}
  %\input{Chapters/mesh/all-code2}
%  \input{./local-malloc.cc}
%  \input{./freelist-attach.cc}
%  \input{./miniheap-malloc.cc}
%  \input{./local-free.cc}
%  \input{./miniheap-free.cc}
  \caption{Pseudocode for \Mesh's core allocation and deallocation routines.}
  \label{fig:malloc}
\end{figure}



\subsubsection{MiniHeap Allocation}

Allocating a MiniHeap of size $k$ pages begins with requesting $k$
pages from the meshable arena.  The global allocator then allocates
and initializes a new MiniHeap instance from an internal allocator
that \Mesh uses for its own needs. This MiniHeap is kept live so long
as the number of allocated objects remains non-zero, and singleton
MiniHeaps are used to account for large object allocations.  Finally,
the global allocator updates the mapping of offsets to MiniHeaps for
each of the $k$ pages to point at the address of the new MiniHeap.

\subsubsection{Large Objects}

All large allocation requests (greater than 16K) are directly
handled by the global heap. Large allocation requests are rounded up
to the nearest multiple of the hardware page size (4K on x86\_64),
and a MiniHeap for 1 object of that size is requested, as detailed
above.  The start of the span tracked by that MiniHeap is returned to
the program as the result of the malloc call.

\subsubsection{Non-local Frees}

If \texttt{free} is called on a pointer that is not contained in an
attached MiniHeap for that thread, the free is handled by the global
heap.  Non-local frees occur when the thread that frees the object is
different from the thread that allocated it, or if there have been
sufficient allocations on the current thread that the original
MiniHeap was exhaused and a new MiniHeap for that size class was
attached.

Looking up the owning MiniHeap for a pointer is a constant time
operation. The pointer is checked to ensure it falls within the arena,
the arena start address is subtracted from it, and the result is
divided by the page size.  The resulting offset is then used to index
into a table of MiniHeap pointers. If the result is zero, the pointer
is invalid; otherwise, it points to a live MiniHeap.  This enables us
to catch certain application-level memory management errors, like
non-local double frees when a new object hasn't been allocated at the
same address.

Once the owning MiniHeap has been found, that MiniHeap's bitmap is
updated atomically in a compare-and-set loop.  If a free occurs for an
object where the owning MiniHeap is attached to a different thread,
the free atomically updates that MiniHeap's bitmap, but does not
update the other thread's corresponding shuffle vector.


\subsection{Meshing}
\label{sec:meshing-implementation}

\Mesh's implementation of meshing is guided by theoretical results
(described in detail in Section~\ref{sec:theory}) that enable it to
efficiently find a number of spans that can be meshed.

Meshing is rate limited by a configurable parameter, settable at
program startup and during runtime by the application through the
semi-standard \texttt{mallctl} API.  The default rate meshes at most
once every tenth of a second.  If the last meshing freed less than one
MB of heap space, the timer is not restarted until a subsequent
allocation is freed through the global heap.  This approach ensures that \Mesh
does not waste time searching for meshes when the application and heap
are in a steady state.

We implement the \sm algorithm from Section~\ref{sec:algorithms} in
C++ to find meshes.  Meshing proceeds one size class at a time.  Pairs
of mesh candidates found by \sm are recorded in a list, and after \sm
returns candidate pairs are meshed together \emph{en masse}.

Meshing spans together is a two step process. First, \Mesh{}
consolidates objects onto a single physical span. This consolidation
is straightforward: \Mesh{} copies objects from one span into the free
space of the other span, and updates MiniHeap metadata (like the
allocation bitmap).  Importantly, as \Mesh copies data at the physical
span layer, even though objects are moving in memory, no pointers or
data internal to moved objects or external references need to be
updated. Finally, \Mesh{} updates the process's virtual-to-physical mappings
to point all meshed virtual spans at the consolidated physical
span.

Physical memory reclaimed from meshing in one size class is able to be
used to satisfy future allocations in other size classes.

\subsubsection{Page Table Updates}
\label{sec:page-table-updates}

\Mesh updates the process's page tables via calls to \texttt{mmap}.
We exploit the fact that \texttt{mmap} lets the same offset in a file
(corresponding to a physical span) be mapped to multiple
addresses. \Mesh's arena, rather than being an anonymous mapping, as
in traditional \texttt{malloc} implementations, is instead a shared mapping
backed by a temporary file. This temporary file is obtained via the
\texttt{memfd\_create} system call and only exists in memory or on
swap.


\subsubsection{Concurrent Meshing}

Meshing takes place concurrently with the normal execution of other
program threads with \textit{no} stop-the-world phase required.  This
is similar to how concurrent relocation is implemented in low-latency
garbage collector algorithms like Pauseless and
C4~\cite{click:2005:pauseless, tene:2011:c4}, as described below.
\Mesh maintains two invariants throughout the meshing process: reads
of objects being relocated are always correct and available to
concurrently executing threads, and objects are never written to while
being relocated between physical spans.  The first invariant is maintained
through the atomic semantics of \texttt{mmap}, the second through a
write barrier.

\Mesh's write barrier is implemented with page protections and a
segfault trap handler.  Before relocating objects, \Mesh calls
\texttt{mprotect} to mark the virtual page where objects are being
copied from as read-only.  Concurrent reads succeed as normal.  If a
concurrent thread tries to write to an object being relocated, a
\Mesh-controlled segfault signal handler is invoked by a combination
of the hardware and operating system.  This handler waits on a lock
for the current meshing operation to complete, the last step of which
is remapping the source virtual span as read/write.  Once meshing is
done the handler checks if the address that triggered the segfault was
involved in a meshing operation; if so, the handler exits and the
instruction causing the write is re-executed by the CPU as normal
against the fully relocated object.


\subsubsection{Concurrent Allocation}

All thread-local allocation (on threads other than the one running
\sm) can proceed concurrently and independently with meshing, until
and unless a thread needs a fresh span to allocate from.  Allocation
only is performed from spans owned by a thread, and only spans owned
by the global manager are considered for meshing; spans have a single
owner.  The thread running \sm holds the global heap's lock while
meshing.  This lock also synchronizes transferring ownership of a span
from the global heap to a thread-local heap (or vice-versa).  If
another thread requires a new span to fulfill an allocation request,
the thread waits until the global manager finishes meshing and
releases the lock.

\subsection{Huge Pages}

Mesh's heap is not designed to be used in conjunction with transparent
huge pages, where the page table size used by the kernel and hardware
is 2MB rather than 4KB and the kernel runs a garbage collection-like
daemon to coalesce 4KB pages into 2MB pages.  Huge pages reduce TLB
pressure, but necessarily increase the granularity at which the kernel
manages physical memory on behalf of the process. This coarse
granularity is fundamentally at odds with Mesh's focus on minimizing
heap size.  Additionally, the \texttt{madvise} mechanism that Mesh and
other allocators like jemalloc use to return memory to the OS
interacts poorly with transparent huge pages on Linux (causing 2MB
pages to be split into 4KB pages), to the extent that major software
vendors and operators recommend disabling transparent huge pages
altogether~\cite{cloudera:thb,redis:thb,mongodb:thb,oracle:thb,nelson:thb}.
Applications that need to back datasets or data structures with huge
pages can still directly allocate (non-Mesh-managed) memory from Linux
using one of several interfaces~\cite{lwn:hp-interfaces}.

%% Mainstream CPUs have incorporated hierarchical TLBs for 4 KB pages,
%% reducing the performance impact of TLB
%% pressure~\cite{anandtech:nehelem-tlb,vtune:page-walk}.  Additionally,
%% with transparent huge pages on Linux, there is a high cost in finding
%% pages to physically relocate, in order to convert contiguous chunks of
%% memory from traditional to huge pages.  Many databases and server
%% workloads like redis~\cite{redis:thb}, MongoDB~\cite{mongodb:thb}, and
%% Oracle~\cite{oracle:thb} recommend disabling transparent huge pages.


\fi

\section{Analysis}
\label{sec:theory}


%\newenvironment{claim}[1]{\par\noindent\underline{Claim:}\space#1}{}
\newenvironment{claimproof}[1]{\par\noindent\underline{Proof:}\space#1}{\hfill $\blacksquare$}

%\newtheorem{problem}{Problem}

%% \theoremstyle{definition}
%% \newtheorem{definition}{Definition}[section]

%% \theoremstyle{theorem}
%% \newtheorem{theorem}{Theorem}[section]

%% \theoremstyle{lemma}
%% \newtheorem{lemma}{Lemma}[section]

\newcommand{\bigo}{\mathcal{O}}
\newcommand{\page}{\pi}
\newcommand{\str}{s}
\newcommand{\node}{\mathit {v}}
\newcommand{\W}{\mathcal {W}}
\newcommand{\lp}{\left(}
%\newcommand{\rparen}{\right)}
\newcommand{\rparen}{\right)}
\newcommand{\rp}{\right)}

%%caveat required for lemma 4.4 added below.  run by andrew before finalizing
This section shows that the \sm procedure described in
\S\ref{sec:meshing-algorithm} comes with strong formal guarantees on
the \textit{quality} of the meshing found along with bounds on its
\textit{runtime}.  In situations where significant meshing
opportunities exist (that is, when compaction is most desirable), \sm
finds with high probability an approximation arbitrarily close to
$1/2$ of the best possible meshing in $O\lp n/q\rparen$ time, where
$n$ is the number of spans and $q$ is the global probability of two
spans meshing.

To formally establish these bounds on quality and runtime, we show
that meshing can be interpreted as a graph problem, analyze its
complexity (\S\ref{subsec:graph}), show that we can do nearly as well
by solving an easier graph problem instead (\S\ref{subsec:matching}),
and prove that \sm approximates this problem with high probability
(\S\ref{subsec:analysis}).

% \item{Introduce an easily-computable \emph{a priori} lower bound for the effect of meshing (\S\ref{subsec:lowerbound})}
%\end{itemize}

\begin{figure}[!t]
  \centering
  \includegraphics[width=.8\textwidth]{Chapters/mesh/figures/graph-diagram.pdf}
%  \includegraphics[width=.33\textwidth]{figures/meshing_graph.pdf}
%  \vspace{-2em}
%\centering
\caption{\textbf{An example meshing graph.}  Nodes correspond to the spans represented by the strings \texttt{01101000}, \texttt{01010000}, \texttt{00100110}, and \texttt{00010000}.  Edges connect meshable strings (corresponding to non-overlapping spans).}\label{fig:exmesh}
\end{figure}


\subsection{Formal Problem Definitions}
\label{subsec:probdef}
Since \Mesh{} segregates objects based on size, we can limit our
analysis to compaction within a single size class without loss of
generality. For our analysis, we represent spans as binary strings of
length $b$, the maximum number of objects that the span can
store. Each bit represents the allocation state of a single object. We
represent each span $\page$ with string $\str$ such that $\str\lp
i\rparen = 1$ if $\page$ has an object at offset $i$, and 0 otherwise.

\begin{definition}
We say  two  strings $\str_{1}, \str_{2}$ \em{mesh} iff $\sum_i \str_{1}\lp i\rparen \cdot \str_{2} \lp i \rparen = 0$. More generally, a set of binary strings are said to mesh if every pair of strings in this set mesh.
%We say that k binary strings $\str_{1}, ... \str_{k}$ mesh iff $\sum_{j \in [1,k]} \str_{j}(i) \leq 1$ $\forall i \in [0,b-1]$.\\
%Equivalently, k binary strings mesh if each each of said strings are pairwise meshable with all of the other strings.
\end{definition}

When we mesh $k$ spans together, the objects scattered across those
$k$ spans are moved to a single span while retaining their offset from
the start of the span. The remaining $k-1$ spans are no longer needed
and are released to the operating system. We say that we ``release''
$k-1$ \emph{strings} when we mesh $k$ strings together.  Since our
goal is to empty as many physical spans as possible, we can
characterize our theoretical problem as follows:

\begin{problem}
Given a multi-set of $n$ binary strings of length $b$, find a meshing
that releases the maximum number of strings.
\end{problem}

% Note that the total number of strings released is equal to $n - \rho -
% \phi$, where $\rho$ is the number of total meshes performed, and $\phi$
% is the number of strings that remain unmeshed.

\paragraph*{A Formulation via Graphs:}
\label{subsec:graph}

We observe that an instance of the meshing problem, a string multi-set
$S$, can naturally be expressed via a graph $G(S)$ where there is a
node for every string in $S$ and an edge between two nodes iff the
relevant strings can be meshed. Figure~\ref{fig:exmesh} illustrates
this representation via an example.

%Given , we construct a meshing graph $G_S$ as follows:\\
%We add a node $\node$ to $G_S$ for each $\str \in S$.  We say node $\node$ represents $\str$.  For each %node pair $\node_1, \node_2$, we add edge $(\node_1, \node_2)$ iff $\node_1$ and $\node_2$ mesh.

%This meshing problem is reducible to min clique cover (MCC), an NP-Complete problem.  We implicitly show %this reduction through a straightforward graph interpretation of the meshing problem.

If a set of strings are meshable, then there is an edge between every
pair of the corresponding nodes: the set of corresponding nodes is a
\emph{clique}. We can therefore decompose the graph into $k$ disjoint
cliques iff we can free $n-k$ strings in the meshing
problem. Unfortunately, the problem of decomposing a graph into the
minimum number of disjoint cliques (\textsc{MinCliqueCover}) is in
general NP-hard. Worse, it cannot even be approximated up to a factor
$m^{1-\epsilon}$ unless $P=NP$~\cite{zuckerman07}.

While the meshing problem is reducible to {\textsc{MinCliqueCover}},
we have not shown that the meshing problem is NP-Hard.  The meshing
problem is indeed NP-hard for strings of arbitrary length, but in
practice string length is proportional to span size, which is
constant.

\begin{theorem}\label{thm:polytime}
The meshing problem for $S$, a multi-set of strings of constant length, is in $P$. %can be solved in polynomial time.
%is in P
%; equivalently, the min clique cover problem on meshing graphs generated from sets of strings of constant %length is in P.
\end{theorem}

\begin{proof}
We assume without loss of generality that $S$ does not contain the
all-zero string $\str_0$; if it does, since $\str_0$ can be meshed
with any other string and so can always be released, we can solve the
meshing problem for $S \setminus \str_0$ and then mesh each instance
of $\str_0$ arbitrarily.

Rather than reason about {\textsc{MinCliqueCover}} on a
 meshing graph $G$, we consider the equivalent
problem of coloring the complement graph $\bar{G}$ in which there is
an edge between every pair of two nodes whose strings do not mesh. The nodes of $\bar{G}$ can be partitioned into at most $2^b-1$
subsets $N_1 \ldots N_{2^b-1}$ such that all nodes in each
$N_i$ represent the same string $\str_i$.  The induced subgraph of
$N_i$ in $\bar{G}$ is a clique since all its nodes have a 1 in the same position and so cannot be pairwise meshed.  Further, all nodes in $N_i$ have the same set of neighbors.

Since $N_i$ is a clique, at most one node in $N_i$ may be colored with
any color.  Fix some coloring on $\bar{G}$.  Swapping the colors of
two nodes in $N_i$ does not change the validity of the coloring since
these nodes have the same neighbor set.  We can therefore
unambiguously represent a valid coloring of $\bar{G}$ merely by
indicating in which cliques each color appears.

With $2^b$ cliques and a maximum of $n$ colors, there are at most $\lp
n+1 \rparen ^{c}$ such colorings on the graph where $c=2^{2^b}$. This follows because each color used can be associated with a subset of $\{1, \ldots, 2^b\}$ corresponding to which of the cliques have node with this color; we call this subset a \emph{signature} and note there are $c$ possible signatures. A coloring can be therefore be associated with a multi-set of possible signatures where each signature has multiplicity between 0 and $n$; there are $(n+1)^c$ such multi-sets. This is polynomial in $n$ since $b$ is constant and hence $c$ is also constant. So we can simply check each coloring for validity (a coloring is valid iff no color appears in two cliques whose string representations mesh). The algorithm returns a valid coloring with the lowest number of colors from all valid colorings discovered.
\end{proof}

%Unfortunately, the exponent of $n$ runtime grows doubly exponentially in  $b$ and $b$ can be as large as 256.
%Such a runtime,
Note that the runtime of the above algorithm is at least exponential in the string length. While technically polynomial for constant string length, the running time of the above algorithm would obviously be prohibitive in practice and so we never employ it in \Mesh{}. Fortunately, as we show next, we can exploit the randomness in the strings to design a much faster algorithm.

\subsection{Simplifying the Problem: From \textsc{MinCliqueCover} to \textsc{Matching}}
\label{subsec:matching}
We leverage \Mesh{}'s random allocation to simplify meshing; this random
allocation implies a distribution over the graphs that exhibits useful
structural properties. We first make the following important observation:

\begin{observation}
Conditioned on the occupancies of the strings, edges in the meshing graph  are not three-wise independent.
\end{observation}

To see that edges are not three-wise independent consider three random
strings $s_1, s_2, s_3$ of length 4, each with exactly 2 ones. It is
impossible for these strings to all mesh mutually since if we know that $s_1$ and $s_2$ mesh,
and that $s_2$ and $s_3$ mesh, we know for certain that $s_1$ and
$s_3$ cannot mesh. More generally, conditioning on $s_1$ and $s_2$
meshing and $s_1$ and $s_3$ meshing decreases the probability that
$s_1$ and $s_3$ mesh.
%Note that even when we assume bits are 1 or 0
%independently, edges are still not independent.
%The existence of edge
%($s_4,s_5$) is weak evidence of $s_5$'s low occupancy, increasing the
%probability that edge ($s_5,s_6$) exists.
Below, we quantify this
effect to argue that we can mesh near-optimally by solving the much
easier \textsc{Matching} problem on the meshing graph (i.e.,
restricting our attention to finding cliques of size 2) instead of
\textsc{MinCliqueCover}. Another consequence of the above observation is that we will not be able to appeal to  theoretical results on the standard model of random graphs, \emph{Erd\H{o}s-Renyi graphs}, in which  each possible edge is present with some fixed probability and the edges are fully independent. Instead we will need new algorithms and proofs that only require independence of acyclic collections of edges.

\subsubsection{Triangles and Larger Cliques are Uncommon.}
Because of the dependencies across the edges present in a meshing
graph, we can argue that \emph{triangles} (and hence also larger
cliques) are relatively infrequent in the graph and certainly less
frequent than one would expect were all edges independent.  For
example, consider three strings $s_1, s_2, s_3\in \{0,1\}^b$ with
occupancies $r_1, r_2,$ and $r_3$, respectively. The probability they
mesh is
\[
{\binom{b-r_1}{r_2}} \big / {\binom{b}{r_2}} \times {\binom{b-r_{1}-r_2 }{r_3}} \big / {\binom{b}{r_3}} \ . \]

This value is significantly less than would have been the case if the
events corresponding to pairs of strings being meshable were
independent.
%In most practical cases, we will not be interested in meshing strings with low occupancy --- it is often better just to wait until %the last objects on these spans are freed.  For higher occupancies, the expected number of triangles is quite low.
For instance, if $b = 32, r_1=r_2=r_3 = 10$, this probability is so
low that even if there were $1000$ strings, the expected number of
triangles would be less than 2. In contrast, had all meshes been
independent, with the same parameters, there would have been $167$ triangles.

The above analysis suggests that we can focus on finding only cliques
of size 2, thereby solving \textsc{Matching} instead of
\textsc{MinCliqueCover}. The evaluation in
Section~\ref{subsec:exp} vindicates this approach, and we show a
strong accuracy guarantee for \textsc{Matching} in Section~\ref{subsec:analysis}.
% We also verify this empirically in the Appendix.

\subsection{Experimental Confirmation of Maximum Matching/Min Clique Cover Convergence}
\label{subsec:exp}

In Section~\ref{subsec:matching}, we argue that we can approximate the solution to \textsc{MinCliqueCover} on meshing graphs with high probability by instead solving \textsc{MaximumMatching}.

We experimentally verify this result by generating many random constant occupancy graphs and, for each graph, comparing the size of the maximum matching to the size of a greedy (non-optimal) solution for \textsc{MinCliqueCover}.  The results are summarized in Figure~\ref{plot:const}.

\begin{figure}[h]
\includegraphics[scale = 1]{Chapters/mesh/figures/const_match_comp.pdf}
\centering
\caption{\textbf{Min Clique Cover and Max Matching solutions converge.} The average size of Min Clique Cover and Max Matching for randomly generated constant occupancy meshing graphs, plotted against span occupancy.  Note that for sufficiently high-occupancy spans, Min Clique Cover and Max Matching are nearly equal.}
\label{plot:const}
\end{figure}



When we instead assume bits are 1 independently with probability p, we expect the graph to have many more triangles.  For $p = r/b = 10/32, n = 1000$, the expected number of triangles is roughly 36,000.  However, we can see experimentally that these graphs behave quite similarly in Figure~\ref{plot:indep}.


\begin{figure}[h]
\includegraphics[scale = 1]{Chapters/mesh/figures/ind_match_comp.pdf}
\centering
\caption{\textbf{Converge still holds for independent bits assumption.} The average size of Min Clique Cover and Max Matching for randomly generated constant occupancy meshing graphs, plotted against span occupancy.  Note that for sufficiently high-occupancy spans, Min Clique Cover and Max Matching are nearly equal.}
\label{plot:indep}
\end{figure}

While constant occupancy graphs are fairly regular, independent bit graphs may not be.  Since strings have different occupancies, nodes which correspond to strings with relatively low occupancy will tend to have significantly higher degree than other nodes in the graph.  Meanwhile, other nodes may have strings with high occupancy, and therefore only have a few edges (probably with low-occupancy nodes).  So while there are many triangles, when the graph is sparse enough, even meshing cliques of size 3 and 4 will likely "abandon" adjacent high-occupancy nodes, one of which could have been matched with the high degree node to yield the same number of releases.

So in this case we still expect finding the maximum matching to be good enough.
% \paragraph{We Can Restrict Our Attention to Matching.}
\begin{comment}
The above analysis leads us to conclude that, since there are
relatively few triangles in meshing graphs, we can achieve almost the
full benefits of solving \textsc{MinCliqueCover} by only solving
\textsc{Matching}.
\end{comment}


\subsection{Theoretical Guarantees}
\label{subsec:analysis}
Since we need to perform meshing at runtime, it is essential that our
algorithm for finding strings to mesh be as efficient as possible. It
would be far too costly in both time and memory overhead to actually
construct the meshing graph and run an existing matching algorithm on
it. Instead, the \sm algorithm (shown in Figure \ref{fig:meshalg})
performs meshing without the need for explicitly constructing the
meshing graph.

For further efficiency, we need to constrain the value of the
parameter $t$, which controls \Mesh{}'s space-time tradeoff. If $t$
were set as large as $n$, then \sm could, in the worst case,
exhaustively search all pairs of spans between the left and right
sets: a total of $n^2/4$ probes.  In practice, we want to choose a
significantly smaller value for $t$ so that \Mesh{} can always
complete the meshing process quickly without the need to search all
possible pairs of strings.


\begin{lemma}
Let $t=k/q$ where $k>1$ is some user defined parameter and $q$ is the global probability of two
spans meshing. \sm finds a matching
of size at least $n(1-e^{-2k})/4$ between the left and right span sets
with probability approaching 1 as $n\geq 2k/q$ grows.
\end{lemma}

\begin{proof}
%\sm in essence creates a bipartite meshing graph by splitting the span set in half.
Let $S_l=\{v_1, v_2, \ldots v_{n/2}\}$ and $S_r=\{u_1,
u_2, \ldots u_{n/2}\}$. Let $t=k/q$ where
$k>1$ is some arbitrary constant. For $u_i\in S_l$ and $i \leq j \leq
j+t$, we say $(u_i,v_j)$ is a \emph{good match} if all the following
properties hold: (1) there is an edge between $u_i$ and $v_j$, (2)
there are no edges between $u_i$ and $v_{j'}$ for $i\leq j'<j$, and
(3) there are no edges between $u_{i'}$ and $v_{j}$ for $i< i'\leq j$.

We observe that \sm finds any good match, although it may
also find additional matches. It therefore suffices to consider only
the number of good matches. The probability $(u_i,v_j)$ is a good
match is $q(1-q)^{2(j-i)}$ by appealing to the fact that the collection of edges under consideration is acyclic. Hence, $\Pr(u_i \mbox{ has a good match})$
is
\begin{align*}
r:= q \sum_{i=0}^{k/q-1} \lp 1-q\rparen ^{2i} = q \frac{1-(1-q)^{2k/q}}{1-(1-q)^2}
% \geq q \frac{1-e^{-2k}}{2q-q^2}
 > \frac{1-e^{-2k}}{2} \ .
\end{align*}

To analyze the number of good matches, define $X_i = 1$ iff $u_i$ has
a good match. Then, $\sum_i X_i$ is the number of good matches. By
linearity of expectation, the expected number of good matches is
$rn/2$. We decompose $\sum_i X_i$ into \[Z_0+Z_1+\ldots + Z_{t-1} ~~\mbox{
  where }~~ Z_{j} = \sum_{i\equiv j \bmod t} X_i \ .\] Since each
$Z_j$ is a sum of $n/(2t)$ independent variables, by the Chernoff
bound,\\ $P\lp Z_j < \lp 1-\epsilon \rparen E[Z_j]\rparen \leq \exp\lp -
\epsilon^2 r n/(4t)\rparen$.  By the union bound,
$$P\lp X < \lp 1-\epsilon \rparen rn/2\rparen \leq t \exp\lp - \epsilon^2 r
n/(4t)\rparen$$ and this becomes arbitrarily small as $n$ grows.
\end{proof}

In the worst case, the algorithm checks $nk/2q$ pairs. For our
implementation of \Mesh, we use a static value of $t = 64$; this value
enables the guarantees of Lemma 5.1 in cases where significant meshing
is possible.  As Section~\ref{sec:evaluation} shows, this value for
$t$ results in effective memory compaction with modest performance
overhead.




\subsection{New Lower Bound for Maximum Matching Size}
\label{subsec:lowerbound}

In this section, we develop a bound for the size of the maximum matching in a graph that can easily be estimated in the context of meshing graphs.  As meshing may be costly to perform, this lower bound is useful as it can be used to predict the magnitude of compaction achievable before committing to the process.  In the case where little compaction is possible, it is often better not to try to mesh and instead conserve resources for other tasks. The quantity we introduce will always lower bound the size of  the maximum matching and will typically be relatively close to the size of the maximum matching.  For example, if we want to release 30\% of our active spans through meshing, but the bound suggests a release of less than 5\% is possible, we can infer that the maximum matching on the graph is small and meshing is currently not worth attempting.

Our approach is based on extending a result by McGregor and Vorotnikova \cite{McGregorV16}. Let $\ndegree(u)$ be the degree of node $u$ in a graph. They considered the quantity
$\sum_{e\in E} 1/\max(\ndegree(u),\ndegree(v))$
and showed that it is at most a factor $3/2$ larger than the maximum matching in the graph and at most a factor $4$ smaller in the case of planar graphs. These bounds were tight.  For example, on a complete graph on three nodes, the quantity is $3/2$ while $M$ is 1. Meshing graphs are very unlikely to be planar but are likely be almost regular, i.e., most degrees are roughly similar.
We need to extend the above bound such that we can guarantee that it never exceeds the size of the maximum matching while also being a good estimate for the graphs that are likely to arise as meshing graphs.

%\begin{theorem}
%Define $W(G)$ to be a fractional matching on graph $G = (V,E)$ such that $$W = \sum_{(u,v) \in E} \min\lp %\frac{1}{\ndegree\lp u\rparen },\frac{1}{\ndegree\lp v\rparen }\rparen $$
%Then $W\leq \frac{3}{2} M$ where $M$ is the cardinality of the maximum matching on $G$.
%If $G$ is bipartite, then $W \leq M$.
%\end{theorem}

%For general graphs, $W$ is not always a lower bound on the maximum matching (hence the need for the $\frac{3}{2}$ factor).  For example, on a complete graph on three nodes, $W$ is $\frac{3}{2}$ while $M$ is 1.

%\begin{figure}[h]
%\includegraphics[scale = .7]{figures/triangle_matching.png}
%\centering
%\caption{W is not a lower bound for the maximum matching on a triangle.}
%\end{figure}
One simple approach is to scale the above quantity by a factor of $2/3$,
but this can result in a poor approximation for the size of the
maximum matching for some graphs of interest. Instead, we take a more
nuanced approach. Specifically, we prove the following theorem (proof omitted
due to space constraints):
%in the
%Appendix:

%While it is tight for this example, for larger non-bipartite graphs it seems like scaling $W$ down by %$\frac{2}{3}$ is overkill.  For instance, for a complete graph on an odd number of nodes k, $W = %{{k}\choose{2}} \frac{1}{k-1} = \frac{k}{2}$, while $M = \frac{k-1}{2}$.
%The following theorem provides a tight lower bound for this example.


\begin{theorem}
\[
\W=\sum_{e\in E} \frac{1}{\max(\ndegree(u), \ndegree(v))+I[\min(\ndegree(u),\ndegree(v))>1]}\leq M \ .
\]
%where $f(1,\cdot)=f(\cdot,1)=f(2,3)=f(3,2)=0$ and $f(\cdot,\cdot)=1$ otherwise. Then, $\sum_{e\in E} w_e \leq M$.
% . Then, $\sum_{e\in E} w_e \leq M$.
\end{theorem}

\begin{proof}
We begin by showing that the simpler quantity $W$ is a lower bound of the maximum matching $M$.
\[
W = \sum_{e\in E} \frac{1}{\max(\ndegree\lp u\rparen, \ndegree\lp v\rparen )+1}\ .
\]
%For each edge $e = (u,v) \in G$, define $\W$$(e)=min\lp \frac{1}{\ndegree\lp u\rparen +1},\frac{1}{\ndegree\lp %v\rparen +1}\rparen$.  L
Let $U$ be an arbitrary set of $t$ nodes in $G$ where $t$ is odd.   Define $$W(U) = \sum_{u,v \in U} \frac{1}{\max(\ndegree\lp u\rparen, \ndegree\lp v\rparen )+1} \ . $$ As a corollary of Edmonds Matching Polytope Theorem, it can be shown that $W \leq M$ if  $W(U) \leq (|U|-1)/2$. We can argue this as follows:

\begin{align*}
W(U) &= \sum_{(u,v) \in U} \min \left(\frac{1}{\ndegree(u)+1},\frac{1}{\ndegree(v)+1}\right )\\
& \leq \sum_{(u,v) \in U}\frac{1}{2}\lp \frac{1}{\ndegree\lp u \rparen +1}+\frac{1}{\ndegree\lp v\rparen +1}\rparen\\
&\leq \sum_{(u,v) \in U}\frac{1}{2}\lp \frac{1}{\ndegree_U\lp u\rparen +1}+\frac{1}{\ndegree_U\lp v\rparen +1}\rparen\\
&= \frac{1}{2} \sum_{u \in U} \frac{\ndegree_U\lp u\rp }{\ndegree_U\lp u\rp +1} \leq  \frac{1}{2} \lp \frac{t-1}{t}\rp  t = \frac{t-1}{2}
\end{align*}
where $\ndegree_U(u)$ in the number of neighbors of $u$ in the set $U$.
The second line follows from the fact that the minimum of two quantities is bounded above by their average.  The third line follows from the fact that the degree of any node in a subgraph is bounded above by its degree in the original graph.  The fourth line follows from summing over nodes instead of edges, and then reasoning that in the worst case U is a clique and so $\degree_U(u) = t-1$ for all u $u$.

\iffalse
Recall that we prove in Section~\ref{subsec:lowerbound} that when
\[
w_e=\min\lp \frac{1}{\ndegree\lp u\rp+1 },\frac{1}{\ndegree\lp v\rp+1 }\rp
~~\mbox{ and }~~\W=\sum_{e\in E} w_e \ .\]
then $\W$ is a lower bound on the cardinality of the maximum matching M(G).
\fi


In some cases $W$ is too conservative; it assigns little weight to edges which it could safely have assigned much more.  For example, if $e$ is isolated (meaning its endpoints have degree 1), $W(e) = 1/2$.  However, it is always safe to assign weight 1 to $e$, since $M(G-e) = M(G)-1.$  If we modified our rule for $W$ so that for any edge $e= (u,v)$ s.t. $\deg(u)=\deg(v) = 1$ we assigned weight $ \min(1/\deg(u),1/\deg(v))$ instead of $ \min(1/\deg(u)+1,1/\deg(v)+1)$, we would always assign weight 1 to isolated edges.

In fact, a more general rule is true.  For any edge $e= (u,v)$, if either $\deg(u) or \deg(v) = 1$ then we may assign it weight $\min(1/\deg(u),1/\deg(v))$.

\iffalse
At this point it is natural to ask whether we can remove the $+1$s from the denominator for other edges.  We cannot of course do so for $\deg(u) = \deg(v) = k >1, k \in 2\mathbb{Z}$ since for a clique on k+1 nodes this rule would result in total weight $(k+1)/2$.\\

However, we can remove the $+1$ under certain circumstances. We show that if one of the endpoints has degree one then this is the case.
\fi

%prove several rules of this form.  For each rule we specify values of $\deg(u)$ and $\deg(v)$ for which the %$+1$ can be removed from the denominator of the edge weight.
%First we restate the claim that isolated edges may safely be assigned weight 1.
%\begin{lemma}
%Define $\W(G)$ as follows:\\
%For each $(u,v) \in G$,\\
%if $\deg(u) = \deg(v) = 1$, $\W(u,v) =  \min(\frac{1}{\deg(u)},\frac{1}{\deg(v)})$.\\
%else $\W(u,v) =  \min(\frac{1}{\deg(u)+1},\frac{1}{\deg(v)+1})$.
%Then $\W(G) \leq M(G)$.
%\end{lemma}
%\begin{proof}
%Already proven above.
%\end{proof}
%Next we will amend $\W$ to include the following rule: when one endpoint has degree 1 and the other has %degree k, assign weight $\frac{1}{k}$.

Define $\W$ as follows:
\[\W=\sum_{(u,v)\in E} \W(u,v) \]
where
\[
\W(u,v)=\frac{1}{\max(\ndegree\lp u\rp, \ndegree\lp v\rp )+I[\min(\ndegree(u),\ndegree(v))> 1]}\]

We now show that $\W \leq M$. We have proven that $W(U)\leq (t-1)/2$ on any odd-size subgraph $U$, $|U| = t$.  Define subsets $U_1$ and $U_2$ of $U$ such that $U_1 \cup U_2 = U$.  $U_1$ is the set of all nodes in $U$ of degree 1 and all nodes in $U$ adjacent to a node of degree 1, and $U_2$ is the set of all other nodes in $U$.  Let $G(U_2)$ denote the subgraph of $U$ induced by $U_2$, and let $|U_1| = x$ where x is even.  Then $W(G(U_2)) = \W(G(U_2)) \leq (t-x-1)/2$.  So to complete our proof we must show that all remaining edges (call them $E'$) have total weight $\leq x/2$.

Assume WLOG that there are no isolated edges in $G$ (if there are, we can group them with $G(U_2)$ and retain the $(t-x-1)/2$ bound).

\begin{align*}
\W(E') & \leq \frac{1}{2} \sum_{u,v \in E'} \lp \frac{1}{\deg(u)}+\frac{1}{\deg(v)}\rp\\
& = \mathlarger{\sum_k} \Big(\sum_{v \in U_1, \deg(v) = k} \frac{\delta}{k} + \frac{k-\delta}{2(k+1)}\Big)
\end{align*}
where $\delta$ denotes the number of degree 1 nodes adjacent to $v$.\\
Let $f(\delta) = \delta/k + (k-\delta)/(2(k+1))$.  We are interested in finding the maximum value $f(\delta)/(\delta+1)$ can take on; if it can never take a value greater than $1/2$ then $\W(E')$ cannot be greater than $x/2$.
$$\frac{\partial\frac{f(\delta)}{\delta+1}}{\partial \delta} = \frac{2-k}{2k(\delta+1)^2}$$
which is always negative for $k \geq 2$.  So, $f(\delta)/(\delta+1)$ is maximized at $\delta = 0$, so  $f(\delta)/(\delta+1) \leq k/(2(k+1)) < 1/2.$

There is one final detail we have not considered: $x$ might be odd.  In this case, $|U_2|$ is even and we can't appeal to Theorem 3 to say that $\W(G(U_2)) \leq (t-x-1)/2$.  However, we can simply remove one node $\node'$ from $U_2$; the resulting odd-size subgraph has weight at most $(t-x-2)/2$.  Since $\W$ is a valid fractional matching, the weight assigned to all edges adjacent to $\node'$ cannot exceed 1, so we can say that $\W(G(U_2)) \leq (t-x)/2$.  Now we must show that $\W(E') \leq (x-1)/2$.

We have shown that the edge weight per node in $U_1$ cannot exceed $1/2$.  $\W(E')$ is minimized when there is exactly 1 node of degree 1, with a degree k neighbor. In this case, the only edge in $E'$ will be assigned weight $k/(2(k+1))< 1/2$.  $\W(E') \leq 1/2 + (x-2)/2 = (x-1)/2$ and the theorem is proven.
\end{proof}


\paragraph*{A remark on estimating $\W$.}  If we wish to use this lower bound to decide whether to begin meshing by predicting the size of the maximum matching, we cannot compute it exactly because we do not know the degrees of the nodes in the graph.  However, we do know the degree distribution of the graph and so it is possible to calculate the expected value of $\W$.





%\subsection{Summary}
%\label{subsec:theorysummary}

\subsection{Summary of Analytical Results}
We show the problem of meshing is reducible to a graph problem,
\textsc{MinCliqueCover}.  While solving this problem is infeasible, we
show that probabilistically, we can do nearly as well by finding the
maximum \textsc{Matching}, a much easier graph problem. We analyze our
meshing algorithm as an approximation to the maximum matching on a
random meshing graph, and argue that it succeeds with high
probability.  Finally, we prove a new lower bound on the maximum matching for graphs based on degree distribution.
As a corollary of these results, \Mesh breaks the Robson
bounds with high probability.

%for a broad class of applications.
%As long as programs do not alter
%programs that do not alter their allocation behavior in response to
%addresses the allocator returns.


\begin{comment}
  An adversary attempting to create an
unmeshable memory state requires the capability to position objects at
the same locations in multiple spans.  But an address-oblivious
adversary can only do this with low probability - since allocation
offsets within a span are random, the best it can do is free some
objects from each span and hope that some of the remaining objects
reside at the same offset.  Since objects are distributed uniformly at
random from within spans, our analytical results hold and therefore
there is either a large maximum matching on the resulting meshing
graph (indicating significant compaction is possible) or spans are so
full that fragmentation is not significant.
\end{comment}


\section{Summary of Evaluation}
\label{sec:evaluation}


% \end{figure*}
Our evaluation answers the following questions: Does \Mesh reduce
overall memory usage with reasonable performance overhead?
($\S$\ref{sec:evaluation:overallmemory}) Does randomization
provide empirical benefits beyond its analytical guarantees?
($\S$\ref{sec:evaluation:practical})

% (3) Is \Mesh nearly as effective as custom defragmentation solutions?

%% Our evaluation shows that \Mesh doesn't increase memory consumption in
%% the average case, that it significantly reduces memory consumtion
%% under fragmentation, and that it does not impose an unrealistic CPU
%% overhead.

%% We evaluate \Mesh in three ways.  We show that under a workload
%% derived from the Redis test suite, \Mesh is able to recover from
%% fragmentation in ways that other allocators can not match.  Next we
%% show that running the Redis server with mesh doesn't impose a
%% performance overhead on the workload generated by
%% \texttt{redis-benchmark}. Finally, we run SPECint 2006 with mesh and
%% show that it imposes only a modest overhead (geomean: 15\%).

\subsection{Experimental Setup}
\label{subsec:memory-use}

We perform all experiments on a MacBook Pro with 16 GiB of RAM and an
Intel i7-5600U, running Linux 4.18 and Ubuntu Bionic. We use glibc
2.26 and jemalloc 3.6.0 for SPEC2006, Redis 4.0.2, and Ruby 2.5.1.
Two builds of Firefox 57.0.4 were compiled as release builds, one with
its internal allocator disabled to allow the use of alternate
allocators via \texttt{LD\_PRELOAD}.  SPEC was compiled with clang
version 4.0 at the \texttt{-O2} optimization level, and \Mesh was
compiled with gcc 8 at the \texttt{-O3} optimization level and with
link-time optimization (\texttt{-flto}).  We primarily compare Mesh
results to the default allocator for each test (jemalloc for Firefox
and Redis, glibc for SPEC and Ruby). Where appropriate, we also include
results from tcmalloc (gperftools 2.5) and Hoard (git commit
\texttt{a0e46aa1}); when omitted, it is because their performance is virtually identical
to jemalloc and glibc.

\textbf{Measuring memory usage:} To accurately measure the memory
usage of an application over time, we developed a Linux-based utility,
\texttt{mstat}\footnote{\texttt{mstat} is open source, and available at \url{https://github.com/bpowers/mstat}}, that runs a program in a new memory control
group~\cite{redhat:cgroups}. \texttt{mstat} polls the resident-set size
(RSS) and kernel memory usage statistics for all processes in the
control group at a constant frequency.  This enables us to account for
the memory required for larger page tables (due to meshing) in our
evaluation. We have verified that \texttt{mstat} does not perturb
performance results.
%We verified that a full run of SPECint 2006 with benchmarks run under
%\texttt{mstat} has the same performance as run without, so the
%additional accounting being performed does not perturb performance
%results.

\begin{comment}
\begin{figure*}[t]
  \centering\small\textbf{~~~~~~~~~SPECint 2006 Relative Mean RSS}\\[0.3em]
  \includegraphics{plots/spec_mem_mean.pdf}
  \caption{\textbf{\Mesh maintains low average memory consumption.}  The SPEC benchmark
    suite does not suffer from serious external fragmentation that
    \Mesh can recover from, but it maintains lower average memory consumption than glibc.
  For \texttt{403.gcc}, \Mesh reduces mean RSS by 15\% (from 220MB to 188MB).}
  \label{fig:spec-mean-rel}
\end{figure*}
\end{comment}

\begin{comment}
\begin{figure*}[!t]
  \centering\small\textbf{~~~~~~~~~SPECint 2006 Absolute Mean RSS}\\[0.3em]
  \includegraphics{plots/spec_mem_mean_mb.pdf}
  \caption{\textbf{\Mesh average memory consumption (absolute).}}
  \label{fig:spec-mean}
\end{figure*}
\end{comment}

\begin{comment}
\begin{figure*}[!t]
  \centering\small\textbf{~~~~~~~~~SPECint 2006 Performance}\\[0.3em]
  \includegraphics{plots/spec_speed.pdf}
  \caption{\textbf{\Mesh imposes modest runtime overhead.} Performance is
    normalized to glibc. Overall, \Mesh's execution time overhead
    is 15\% higher than glibc's (geometric mean).}
  \label{fig:spec-speed}
\end{figure*}
\end{comment}



\subsection{Memory Savings and Performance Overhead}
\label{sec:evaluation:overallmemory}

We evaluate \Mesh{}'s impact on memory consumption and runtime across
the Firefox web browser, the Redis data structure store, and the
SPECint2006 benchmark suite.

\subsubsection{Firefox}
\label{sec:firefox}

Firefox is an especially challenging application for memory reduction
since it has been the subject of a five year effort to reduce its
memory footprint~\cite{awsy}. To evaluate \Mesh{}'s impact on
Firefox's memory consumption under realistic conditions, we measure
Firefox's RSS while running the Speedometer 2.0 benchmark.
Speedometer was constructed by engineers working on the Google Chrome
and Apple Safari web browsers to simulate the patterns in use on
websites today, stressing a number of browser subsystems like DOM
APIs, layout, CSS resolution and the JavaScript engine.  In Firefox,
most of these subsystems are multi-threaded, even for a single
page~\cite{ff:quantum}.  The benchmark comprises a number of small
``todo'' apps written in a number of different languages and styles,
with a final score computed as the geometric mean of the time taken by
the executed tests.

\begin{figure}[t!]% {.4\linewidth}
  \centering
  \vtop{%
    \centering
    \vskip0pt
    \hbox{%
      \includegraphics[width=\textwidth]{Chapters/mesh/plots/firefox-speedometer.pdf}
    }%
  }
  \caption{\textbf{Firefox:} \Mesh decreases mean heap size by 16\%
    over the course of the Speedometer 2.0 benchmark compared with the
    version of jemalloc bundled with Firefox, with less than a 1\%
    change in the reported Speedometer score (\S\ref{sec:firefox}).
    \label{fig:firefox-heap}}
\end{figure}

We test Firefox in single-process mode (disabling content sandboxing,
which spawns multiple processes) under the \texttt{mstat} tool to
record memory usage over time. Our test opens a tab, loads the
Speedometer page from a local server, waits 2 seconds, and then
automatically executes the test.  We record the reported score
at the end of the benchmark run and calculate average memory
usage recorded by \texttt{mstat}.  We tested both a standard release
build of Firefox, along with a release build that did not bundle
Mozilla's fork of jemalloc (hereafter referred to as
\texttt{mozjemalloc}) and instead directly called \texttt{malloc}-related
functions, with \Mesh included via \texttt{LD\_PRELOAD}.  We report
the average resident set size over the course of the benchmark and a
15 second cooldown period afterward, collecting three runs per
allocator.

\Mesh reduces the memory consumption of Firefox by 16\% compared to
Firefox's bundled jemalloc allocator. \Mesh requires 530 MB ($\sigma =
22.4$ MB) to complete the benchmark, while the Mozilla allocator needs
632 MB ($\sigma = 25.3$ MB). Mesh spent a total of 135 ms meshing over
the course of the benchmark, with a maximum meshing latency of 7.5 ms
and average meshing latency of 0.2 ms.  This result shows that \Mesh
can effectively reduce memory overhead even in widely used and heavily
optimized applications. \Mesh achieves this savings with less than a
1\% reduction in performance (measured as the score reported by
Speedometer).  Hoard and tcmalloc improved Speedometer performance
relative to jemalloc by 5.4 and 6.0\% respectively, while increasing
average heap size by 48.0\% and 8.6\%.

Figure~\ref{fig:firefox-heap} shows memory usage over the course of a
Speedometer benchmark run under \Mesh and the default jemalloc
allocator.  While memory usage under both peaks to similar levels,
Mesh is able to keep heap size consistently lower.


\subsubsection{Redis}
\label{redis-section}


Redis is a widely-used in-memory data structure server.  Redis 4.0
introduced a feature called ``active
defragmentation''~\cite{jemalloc:exposehints,redis:announcement}.
Redis calculates a fragmentation ratio (RSS over sum of active
allocations) once a second.  If this ratio is too high, it triggers a
round of active defragmentation. This involves making a fresh copy of
Redis's internal data structures and freeing the old ones. Active
defragmentation relies on allocator-specific APIs in jemalloc both for
gathering statistics and for its ability to perform allocations that
bypass thread-local caches, increasing the likelihood they will be
contiguous in memory.

\begin{figure}[t!]
  \centering
  \vtop{%
    \centering
    \vskip0pt
    \hbox{%
      \includegraphics[width=\textwidth]{Chapters/mesh/plots/redis-lru.pdf}
    }%
  }
  \caption{\textbf{Redis:} \Mesh automatically achieves significant
    memory savings (39\%), obviating the need for its custom,
    application-specific ``defragmentation'' routine
    (\S\ref{redis-section}).
    \label{fig:redis-results}}
\end{figure}


%This defragmentation
%does not require scanning the heap for pointers because Redis's memory
%primarily consists of a large hash table associating keys with values.


%% A case where active defragmentation is useful is when redis is used as
%% an LRU cache with a maximum heap size set.  If there is a phase shift
%% in the size of objects being added to the cache, redis can suffer from
%% external fragmentation, requiring a significantly larger
%% working-set-size compared to live size of redis allocations.

%% Redis tracks the size of allocations (either through the use of a
%% \texttt{malloc\_usable\_size} API, or by prepending the size to every
%% allocation

We adapt a benchmark from the official Redis test suite to measure how
\Mesh's automatic compaction compares with Redis's active
defragmentation, as well as against the standard glibc allocator. This
benchmark runs for a total of 7.5 seconds, regardless of allocator. It
configures Redis to act as an LRU cache with a maximum of 100 MB of
objects (keys and values).  The test then allocates 700,000 random
keys and values, where the values have a length of 240 bytes.
Finally, the test inserts 170,000 new keys with values of length 492.
Our only change from the original Redis test is to increase the value
sizes in order to place all allocators on a level playing field with
respect to \emph{internal} fragmentation; the chosen values of 240 and
492 bytes ensure that tested allocators use similar size classes for
their allocations. We test \Mesh with Redis in two configurations:
with meshing always on and with meshing disabled, both without any
input or coordination from the redis-server application.

%% Our only change from the original Redis test is to
%% increase the key size in order to place all allocators on a level
%% playing field with respect to \emph{internal} fragmentation; both
%% Hoard and DieHard use power-of-two size classes.

Figure~\ref{fig:redis-results} shows memory usage over time for Redis
under \Mesh, as well as under jemalloc with Redis's
``activedefrag'' enabled, as measured by \texttt{mstat}
(\S\ref{subsec:memory-use}).  The ``activedefrag'' configuration
enables active defragmentation after all objects have been added to
the cache.

Using \Mesh automatically and portably achieves the same heap size
reduction (39\%) as Redis's active defragmentation.  During most of
the 7.5s of this test Redis is idle; Redis only triggers active
defragmentation during idle periods. With \Mesh, insertion takes
1.76s, while with Redis's default of jemalloc, insertion takes 1.72s.
Redis under Hoard and tcmalloc has the same average heap size as Mesh
with meshing disabled (under 2\% difference), and both allocators are
similarly within 2\% of the insertion speed of jemalloc.  \Mesh's
compaction is additionally \textit{significantly} faster than Redis's
active defragmentation. During execution with \Mesh{}, a total of
0.23s are spent meshing (the longest pause is 22 ms), while active
defragmentation accounts for 1.49s ($5.5\times$ slower). This high
latency may explain why Redis disables ``activedefrag'' by default.

%\Mesh's always-on defragmentation comes at a cost, the Redis server
%running with \Mesh is 29\% slower than the default jemalloc (which
%performs no defragmentation).

\subsubsection{SPEC Benchmarks}

%Figure~\ref{fig:spec-mean-rel} presents mean memory consumption (RSS)
%across the SPECint 2006 benchmark suite, normalized to glibc.

%Figure~\ref{fig:spec-mean} presents absolute average memory
%usage.

%% We show that meshing can yield substantial memory savings across a
%% suite of benchmarks and real-world applications, with little runtime
%% overhead. For one of the most allocation-heavy benchmarks in the
%% SPECint suite, \texttt{perlbench}, \Mesh reduces peak RSS by 13\%
%% (across the suite, it decreases average memory consumption by 2.5\%).
%% In longer-lived, memory-intensive applications, \Mesh saves more
%% memory. Replacing Firefox's default allocator with \Mesh reduces
%% average memory consumption in Speedometer 2, a modern browser
%% benchmark, by 11\% (from 651MB to 579MB) with a 6\% reduction in
%% performance.  For the Redis data structure server, \Mesh reduces
%% resident-set size (RSS) by 34\% compared to conventional allocators in
% a fragmentation-heavy workload.


Most of the SPEC benchmarks are not particularly compelling targets
for \Mesh because they have small overall footprints and do not
exercise the memory allocator. For example, while the
\texttt{401.bzip2} benchmark has one of the higher heap size averages
at 665 MB, the difference in both runtime and average heap size
between the fastest + slowest allocators is under 1\%.

Across the entire SPECint 2006 benchmark suite, \Mesh modestly
decreases average memory consumption (geomean: $-2.4$\%) vs. glibc,
while imposing minimal execution time overhead (geomean: 0.7\%).

%% Many Mesh benchmarks are

However, for applications that are both allocation-intensive (many
calls to \texttt{malloc} and \texttt{free}) and which have large footprints,
\Mesh is able to substantially reduce peak memory consumption. The
most allocator-sensitive benchmark is \texttt{400.perlbench}, a Perl
benchmark that performs a number of e-mail related tasks including
spam detection (SpamAssassin). Peak RSS with glibc, jemalloc, and
Hoard is 664 MB, 614 MB, and 732 MB respectively. \Mesh reduces peak
RSS to 564 MB (a 15\% reduction relative to glibc) while increasing
runtime overhead by only 3.9\%.


%\Mesh significantly reduces memory consumption for the
%allocation-heavy SPEC application (\texttt{400.perlbench}: 13\%) while
%imposing a minimal performance penalty.


% As Figure~\ref{fig:spec-speed} shows,
\begin{comment}
though in some cases, its overhead is high. We observe that these
cases correspond with the cases when DieHard also imposes significant
overhead; since both DieHard and \Mesh are randomized memory managers,
we attribute this overhead to degraded locality.
\end{comment}



\subsection{Empirical Value of Randomization}
\label{sec:evaluation:practical}

Randomization is key to \Mesh{}'s analytic guarantees; we evaluate
whether it also can have an observable empirical impact on its ability
to reclaim space. To do this, we test three configurations of \Mesh:
(1) meshing disabled, (2) meshing enabled but randomization disabled,
and (3) \Mesh with both meshing and randomization enabled (the
default).

We tested these configurations with Firefox and Redis, and found no
significant differences when randomization was disabled; we believe
that this is due to the highly irregular (effectively random)
allocation patterns that these applications exhibit. We hypothesized
that a more regular allocation pattern would be more challenging for a
non-randomized baseline. To test this hypothesis, we wrote a synthetic
microbenchmark with a regular allocation pattern in Ruby. Ruby is an
interpreted programming language popular for implementing web
services, including GitHub, AirBnB, and the original version of
Twitter.  Ruby makes heavy use of object-oriented and functional
programming paradigms, making it allocation-intensive.  Ruby is
garbage collected, and while the standard MRI Ruby implementation
(written in C) has a custom GC arena for small objects, large objects
(like strings) are allocated directly with \texttt{malloc}.

Our Ruby microbenchmark repeatedly performs a sequence of string
allocations and deallocations, simulating the effect of accumulating
results from an API and periodically filtering some out. It allocates
a number of strings of a fixed size, then retaining references 25\% of
the strings while dropping references to the rest.  Each iteration the
length of the strings is doubled.  The test requires only a fixed 128
MB to hold the string contents.

\begin{figure}[t!]% {.4\linewidth}
  \centering
  \vtop{%
    \centering
    \vskip0pt
    \hbox{%
      \includegraphics[width=\textwidth]{Chapters/mesh/plots/ruby-frag.pdf}
    }%
  }
  \caption{\textbf{Ruby benchmark:} \Mesh is able to decrease mean heap
    size by 18\% compared to \Mesh with randomization disabled and
    non-compacting allocators ($\S$\ref{sec:evaluation:practical}).
    \label{fig:ruby-frag}}
\end{figure}

Figure~\ref{fig:ruby-frag} presents the results of running this
application with the three variants of \Mesh{} and jemalloc; for this
benchmark, jemalloc and glibc are essentially indistinguishable. With
meshing disabled, \Mesh exhibits similar runtime and heap size to
jemalloc. With meshing enabled but randomization disabled, \Mesh
imposes a 4\% runtime overhead and yields only a modest 3\% reduction in
heap size.

Enabling randomization in \Mesh increases the time overhead to 10.7\%
compared to jemalloc, but the use of randomization lets it
significantly reduce the mean heap size over the execution time of the
microbenchmark (a 19\% reduction). The additional runtime overhead is
due to the additional system calls and memory copies induced by the
meshing process.  This result demonstrates that randomization is not
just useful for providing analytical guarantees but can also be
essential for meshing to be effective in practice.

\subsection{Summary of Empirical Results}




For a number of memory-intensive applications, including aggressively
space-optimized applications like Firefox, \Mesh can substantially
reduce memory consumption (by 16\% to 39\%) while imposing a modest
impact on runtime performance (e.g., around 1\% for Firefox and
SPECint 2006). We find that \Mesh{}'s randomization can enable
substantial space reduction in the face of a regular allocation
pattern.

%We find that \Mesh{}'s use of randomization can be
%to reclaim memory

%slightly memory consumption across the SPEC benchmark
%suite and low performance overhead. For two real,
%memory-intensive applications (Firefox and Redis), \Mesh's compaction
%reduces memory consumption significantly with minimal performance
%impact.

%  both reach a steady state where the occupancy of spans containing smaller (240 byte) objects is between 60-65\%, and there are very few spans under half full.


\begin{comment}
  \begin{figure*}[!t]
  \centering\small\textbf{~~~~~~~~~SPECint 2006 Relative Peak RSS}\\[0.3em]
  \includegraphics{plots/spec_mem_peak.pdf}
  \caption{\textbf{\Mesh has little impact on peak memory consumption.}
    The meshing algorithm requires additional bookkeeping and
    temporary allocations within the allocator, but these do not
    translate to increased memory usage across SPEC, and sometimes reduce
    it. \Mesh reduces \texttt{400.perlbench}'s peak memory consumption
    by 13.7\% (from 663MB to 575MB).}
  \label{fig:spec-peak-rel}
\end{figure*}
\end{comment}


\input{Chapters/mesh/discussion}

\section{Related Work}
\label{sec:related-work}

\paragraph*{Hound:}
\label{sec:hound}
Hound is a memory leak detector for C/C++ applications that introduced
meshing (a.k.a. ``virtual compaction''), a mechanism that \Mesh{}
leverages~\cite{1542521}. Hound combines an age-segregated heap with
data sampling to precisely identify leaks. Because Hound cannot
reclaim memory until every object on a page is freed, it relies on a
heuristic version of meshing to prevent catastrophic memory
consumption. Hound is unsuitable as a replacement general-purpose
allocator; it lacks both \Mesh's theoretical guarantees and space and
runtime efficiency (Hound's repository is missing files and it does
not build, precluding a direct empirical comparison here). The Hound
paper reports a geometric mean slowdown of $\approx 30\%$ for
SPECint2006 (compared to \Mesh{}'s 0.7\%), slowing one benchmark
(\texttt{xalancbmk}) by almost $10\times$. Hound also generally
\emph{increases} memory consumption, while \Mesh often substantially
decreases it.

%Finally, Hound was designed for
%32-bit Linux applications, making more direct comparisons to Mesh
%(which is focused on 64-bit applications) hard.


\paragraph*{Compaction for C/C++:}
Previous work has described a variety of manual and compiler-based
approaches to support compaction for C++. Detlefs shows that if
developers use annotations in the form of smart pointers, C++ code can
also be managed with a relocating garbage
collector~\cite{detlefs:1992:gc}.  Edelson introduced GC support
through a combination of automatically generated smart pointer classes
and compiler transformations that support relocating
GC~\cite{edelson:1992:precompilingcgc}. Google's Chrome uses an
application-specific compacting GC for C++ objects called Oilpan that
depends on the presence of a single event
loop~\cite{google:oilpan}. Developers must use a variety of smart
pointer classes instead of raw pointers to enable GC and
relocation. This effort took years. Unlike these approaches, \Mesh is
fully general, works for unmodified C and C++ binaries, and does not
require programmer or compiler support; its compaction approach is
orthogonal to GC.

CouchDB and Redis implement \emph{ad hoc} best-effort compaction,
which they call ``defragmentation''.  These work by iterating through
program data structures like hash tables, copying each object's
contents into freshly-allocated blocks (in the hope they will be
contiguous), updating pointers, and then freeing the old
objects~\cite{jemalloc:exposehints,redis:announcement}. This
application-specific approach is not only inefficient (because it may
copy objects that are already densely packed) and brittle (because it
relies on internal allocator behavior that may change in new
releases), but it may also be ineffective, since the allocator cannot
ensure that these objects are actually contiguous in memory. Unlike
these approaches, \Mesh performs compaction efficiently and its
effectiveness is guaranteed.

\paragraph*{Compacting garbage collection in managed languages:}
%% Recently Baker et al. showed that with compiler support you could
%% achieve accurate GC for C and C++~\cite{baker:2009:accurategc}.
Compacting garbage collection has long been a feature of languages
like LISP and
Java~\cite{hansen:1969:compaction,fenichel:1969:compaction}. Contemporary
runtimes like the Hotspot JVM~\cite{microystems2006memory}, the .NET
VM~\cite{microsoft:dotnet-gc}, and the SpiderMonkey JavaScript
VM~\cite{mozilla:spidermonkey-compaction} all implement compaction as
part of their garbage collection algorithms. \Mesh{} brings the
benefits of compaction to C/C++; in principle, it could also be used
to automatically enable compaction for language implementations that
rely on non-compacting collectors.

\paragraph*{Bounds on Partial Compaction:}
Cohen and Petrank prove upper and lower bounds on defragmentation via
partial compaction~\cite{Cohen:2017:LPC:3050768.2994597,
  Cohen:2013:LPC:2491956.2491973}. In their setting, corresponding to
managed environments, \emph{every} object \emph{may} be relocated to
any free memory location; they ask what space savings can be achieved
if the memory manager is only allowed to relocate a bounded number of
objects. By contrast, \Mesh{} is designed for unmanaged languages
where objects \emph{cannot} be arbitrarily relocated.

%\paragraph{Hardware support:}
%Seshadri \emph{et al.} present a hardware proposal that would allow
%for ``page overlays'' at a cache-line
%granularity~\cite{Seshadri:2015:POE:2749469.2750379}. \Mesh could use
%such a facility instead of page remapping to mesh pages containing
%objects of a cache line size (e.g., 64 bytes) or larger.

\paragraph*{PCM fault mitigation:}
Ipek \emph{et al.} use a technique similar to meshing to address the
degradation of phase-change memory (PCM) over the lifetime of a
device~\cite{ipek:2010:dynamic-replication}.  The authors introduce
dynamically replicated memory (DRM), which uses pairs of PCM pages
with non-overlapping bit failures to act as a single page of
(non-faulty) storage.  When the memory controller reports a page with
new bit failures, the OS attempts to pair it with a complementary
page. A random graph analysis is used to justify this greedy
algorithm.

DRM operates in a qualitatively different domain than \Mesh.  In DRM,
the OS occasionally attempts to pair newly faulty pages
against a list of pages with static bit failures.  This process is
incremental and local.  In \Mesh, the occupancy of spans in the heap
is more dynamic and much less local. \Mesh solves a full,
non-incremental version of the meshing problem each cycle.
Additionally, in DRM, the random graph describes an error model rather
than a design decision; additionally, the paper's analysis is flawed.
The paper erroneously claims that the resulting graph is a simple
random graph; in fact, its edges are not independent (as we show in
\S\ref{subsec:matching}).  This invalidates the claimed performance
guarantees, which depend on properties of simple random graphs. In
contrast, we prove the efficacy of our original \sm algorithm for
\Mesh using a careful random graph analysis.

\begin{comment}
  To avoid catastrophic memory consumption,
Hound employs virtual compaction, a non-randomized, best-effort form
of meshing. Their approach depends on heuristics and does not employ
randomization or come with any guarantees of fragmentation
recovery.
  Hound is not intended to be
space-efficient (and it is not) but rather to find leaks. Hound
identifies leaks by segregating objects by allocation site, protects
``cold'' pages, and delays reuse of pages until every object on that
page has been freed.
\end{comment}

%% Virtual memory operations have additionally been use for compaction in
%% a Java garbage collector~\cite{wegiel:2008:mapping-collector} in a
%% novel collector, as well as in a C allocator that traded address space
%% usage for reliability and security~\cite{1346296}.

%Microsoft's C++/CLI? C++ with GC'ed, moveable objects.


% http://www.filpizlo.com/papers/baker-ccpe09-accurate.pdf


%Theory?

%% Objects with similar lifetimes tend to appear
%% together~\cite{wilson:1995:survey}.


\section{Conclusion}
\label{sec:conclusion}

This chapter introduces \Mesh{}, a memory allocator that efficiently
performs \textit{compaction without relocation} to save memory for
unmanaged languages.  We show analytically that \Mesh{} provably
avoids catastrophic memory fragmentation with high probability, and
empirically show that \Mesh{} can substantially reduce memory
fragmentation for memory-intensive applications written in C/C++ with
low runtime overhead. %In future work, we plan to explore integrating \Mesh{} into language runtimes that do not currently support compaction, such as Go and Rust.

We have released \Mesh as an open source
project; it can be used with arbitrary C and C++ Linux and Mac OS X
binaries and can be downloaded at
\url{http://libmesh.org}.


\iffalse

%% contents suppressed with 'anonymous'
\begin{acks}
  %% Commands \grantsponsor{<sponsorID>}{<name>}{<url>} and
  %% \grantnum[<url>]{<sponsorID>}{<number>} should be used to
  %% acknowledge financial support and will be used by metadata
  %% extraction tools.
  This material is based upon work supported by the
  \grantsponsor{GS100000001}{National Science
    Foundation}{http://dx.doi.org/10.13039/100000001} under Grant
  No.~\grantnum{GS100000001}{1637536}.  Any opinions, findings, and
  conclusions or recommendations expressed in this material are those
  of the author and do not necessarily reflect the views of the
  National Science Foundation.
\end{acks}

\fi

%\bibliography{Chapters/mesh/emery,Chapters/mesh/mesh}

%% \clearpage
%% \pagebreak
\section{Appendix}
\label{sec:appendix}
\newtheorem*{remark}{Remark}

\newcommand{\rp}{\right)}


% TODO: reorganize!

\subsection{Proof of Theorem \ref{thm:polytime}}

\begin{comment}
\begin{proof}
We assume without loss of generality that $S$ does not contain the all-zero string $\str_0$; if it does, since $\str_0$ can be meshed with any other string and so can always be released, we can solve the meshing problem for $S \setminus \str_0$ and then mesh each instance of $\str_0$ arbitrarily.

Rather than reason directly about the Min Clique Cover problem on some bounded-length meshing graph $G$, let us consider the equivalent problem of coloring $\bar{G}$, the complement of $G$.  $\bar{G}$ has an edge between  every pair of two nodes whose strings do not mesh.

$\bar{G}$'s node set $N$ can be partitioned into at most $2^b-1$ subsets $N_1, N_2, ... N_{2^b-1}$ such that $\forall i$, all nodes in $N_i$ represent the same string $\str_i$.  The induced subgraph of $N_i$ is a clique since all its nodes have a 1 in the same position and so cannot be pairwise meshed.  Further, all nodes in $N_i$ has the same neighbors since they all represent the same string.

Since $N_i$ is a clique, at most one node in $N_i$ may be colored with any color.  Fix some coloring on $\bar{G}$.  Swapping the colors of two nodes in $N_i$ does not change the validity of the coloring since these nodes have the same neighbor set.  We can therefore unambiguously represent a valid coloring of $\bar{G}$ merely by indicating in which cliques each color appears.

With $2^b$ cliques and a maximum of $n$ colors, there are at most $\lp n+1 \rp ^{2^{2^b}}$ such colorings on the graph.  This is polynomial in $n$ since $b$ is fixed, so we can simply check each coloring for validity (a coloring is valid iff no color appears in two cliques whose string representations mesh).  Finally, the algorithm returns a valid coloring with the lowest number of colors out of all valid colorings discovered.
\end{proof}
\end{comment}


We assume without loss of generality that $S$ does not contain the all-zero string $\str_0$; if it does, since $\str_0$ can be meshed with any other string and so can always be released, we can solve the meshing problem for $S \setminus \str_0$ and then mesh each instance of $\str_0$ arbitrarily.

Let $\bar G = (V, \bar E)$ be the complement of $G$.  Solving \textsc{MinCliqueCover} on $G(S)$ is equivalent to solving \textsc{Coloring} on $\bar G$.

\begin{lemma}
There are a polynomial number of colorings on $\bar G$.
\end{lemma}
\begin{proof}
Partition $V$ into $V_1, V_2, ... V_k$ such that $V_i = \{u \vert str(u) = s_i\}$.  Note that the induced subgraph of $\bar G$ on each $V_i$ is a clique, and further that $k \leq 2^b-1$.

Since $V_i$ is a clique, at most one node in $V_i$ may be colored with any color.  Fix some coloring on $\bar{G}$.  Swapping the colors of two nodes in $V_i$ does not change the validity of the coloring since these nodes have the same neighbor set.  We can therefore unambiguously represent a valid coloring of $\bar{G}$ merely by indicating in which cliques each color appears.

With $2^b$ cliques and a maximum of $|V| = n$ colors, there are at most $\lp n+1 \rp ^{2^{2^b}}$ such colorings on the graph.
\end{proof}

As a result of the lemma, we can check all colorings in polynomial time and return the minimum valid coloring.

\begin{remark}
This proof relies on the assumption that $b$, string length, is fixed.  This constraint limits meshing graphs to a strict subset of all possible graphs.  If $b = \bigo(n^2)$, any graph can be expressed as a meshing graph and so \textsc{MinCliqueCover} is NP-Hard in this case.
\end{remark}

\subsection{Experimental Confirmation of Maximum Matching/Min Clique Cover Convergence}

In Section~\ref{subsec:matching}, we argue that we can approximate the solution to \textsc{MinCliqueCover} on meshing graphs with high probability by instead solving \textsc{MaximumMatching}.

We experimentally verify this result by generating many random constant occupancy graphs and, for each graph, comparing the size of the maximum matching to the size of a greedy (non-optimal) solution for \textsc{MinCliqueCover}.  The results are summarized in Figure~\ref{plot:const}.

\begin{figure}[h]
\includegraphics[scale = .5]{figures/const_match_comp.pdf}
\centering
\caption{\textbf{Min Clique Cover and Max Matching solutions converge.} The average size of Min Clique Cover and Max Matching for randomly generated constant occupancy meshing graphs, plotted against span occupancy.  Note that for sufficiently high-occupancy spans, Min Clique Cover and Max Matching are nearly equal.}
\label{plot:const}
\end{figure}



When we instead assume bits are 1 independently with probability p, we expect the graph to have many more triangles.  For $p = r/b = 10/32, n = 1000$, the expected number of triangles is roughly 36,000.  However, we can see experimentally that these graphs behave quite similarly in Figure~\ref{plot:indep}.


\begin{figure}[h]
\includegraphics[scale = .5]{figures/ind_match_comp.pdf}
\centering
\caption{\textbf{Converge still holds for independent bits assumption.} The average size of Min Clique Cover and Max Matching for randomly generated constant occupancy meshing graphs, plotted against span occupancy.  Note that for sufficiently high-occupancy spans, Min Clique Cover and Max Matching are nearly equal.}
\label{plot:indep}
\end{figure}

While constant occupancy graphs are fairly regular, independent bit graphs may not be.  Since strings have different occupancies, nodes which correspond to strings with relatively low occupancy will tend to have significantly higher degree than other nodes in the graph.  Meanwhile, other nodes may have strings with high occupancy, and therefore only have a few edges (probably with low-occupancy nodes).  So while there are many triangles, when the graph is sparse enough, even meshing cliques of size 3 and 4 will likely "abandon" adjacent high-occupancy nodes, one of which could have been matched with the high degree node to yield the same number of releases.

So in this case we still expect finding the maximum matching to be good enough.

\iffalse
\subsection{Proof of Theorem 4.4}
\begin{proof}
Since spans all have identical occupancies, we expect the meshing graph to be nearly regular - the degree of each node is binomially distributed with mean $\lp n-1\rp q$.  When there are many objects per span, the degree distribution will be tightly concentrated around this mean.  For example, when $b = 128, n = 1000, q = .05$, the mean degree is 50 and less than 5\% of nodes have a degree greater than 60.

When we split the span set into half, and only consider pairs between the halves, we essentially remove half of the edges from the meshing graph.  To see this, observe that there are ${{n}\choose{2}} \approx n^2/2$ span pairs, and we fail to consider $2*{{n/2}\choose{2}} \approx n^2/4$ of them.  Note also that we fail to consider exactly $n/2 -1$ pairs for each span.  We therefore expect both the total number of edges in the meshing graph and the degree of each node to decrease by half.

How does the maximum matching of a graph change when you evenly sparsify the graph by half?  If the average degree is high enough, we expect this to not make much difference at all - each node in expectation still has many edges from which a matching can be built.

We can formalize this argument by first assuming that every node in the graph has degree at least $d$ and at most $\lp 1 + \epsilon \rp d$.  As we prove in Section~\ref{subsec:lowerbound}, the maximum matching of a graph $G = \lp V, E\rp$ is lower bounded by

$$W\lp G \rp = \sum_{\lp u,v \rp \in V} \min \lp \frac{1}{\degree \lp u \rp +1}, \frac{1}{\degree \lp v \rp +1} \rp$$

With our degree assumption we can further argue

\begin{align*}
W\lp G \rp \geq \sum_{\lp u,v \rp \in V} \frac{1}{\lp 1+\epsilon \rp d+1}
 \geq \frac{nd/2}{\lp 1+\epsilon \rp d+1}
\geq \frac{n}{2\lp 1+\epsilon \rp}
\end{align*}
and $W\leq n/2$.
%\lp G \rp &\leq \sum_{\lp u,v \rp \in V} \frac{1}{d+1} \leq \frac{n}{2}$
%\begin{align*}
%W\lp G \rp &\leq \sum_{\lp u,v \rp \in V} \frac{1}{d+1} \leq \frac{n}{2}
%\end{align*}

Note how these upper and lower bounds are independent of $d$ and $m$.  Thus, we decrease the maximum matching by at most a $1/\lp 1+\epsilon \rp$ factor by splitting the span set.
\end{proof}
\fi

\subsection{Proof of Theorem 4.5}
\begin{proof}
We begin by showing that the simpler quantity $W$ is a lower bound of the maximum matching $M$.
\[
W = \sum_{e\in E} \frac{1}{\max(\degree\lp u\rp, \degree\lp v\rp )+1}\ .
\]
%For each edge $e = (u,v) \in G$, define $\W$$(e)=min\lp \frac{1}{\degree\lp u\rp +1},\frac{1}{\degree\lp %v\rp +1}\rp$.  L
Let $U$ be an arbitrary set of $t$ nodes in $G$ where $t$ is odd.   Define $$W(U) = \sum_{u,v \in U} \frac{1}{\max(\degree\lp u\rp, \degree\lp v\rp )+1} \ . $$ As a corollary of Edmonds Matching Polytope Theorem, it can be shown that $W \leq M$ if  $W(U) \leq (|U|-1)/2$. We can argue this as follows:

\begin{align*}
W(U) &= \sum_{(u,v) \in U} \min \left(\frac{1}{\degree(u)+1},\frac{1}{\degree(v)+1}\right )\\
& \leq \sum_{(u,v) \in U}\frac{1}{2}\lp \frac{1}{\degree\lp u \rp +1}+\frac{1}{\degree\lp v\rp +1}\rp\\
&\leq \sum_{(u,v) \in U}\frac{1}{2}\lp \frac{1}{\degree_U\lp u\rp +1}+\frac{1}{\degree_U\lp v\rp +1}\rp\\
&= \frac{1}{2} \sum_{u \in U} \frac{\degree_U\lp u\rp }{\degree_U\lp u\rp +1} \leq  \frac{1}{2} \lp \frac{t-1}{t}\rp  t = \frac{t-1}{2}
\end{align*}
where $\degree_U(u)$ in the number of neighbors of $u$ in the set $U$.
The second line follows from the fact that the minimum of two quantities is bounded above by their average.  The third line follows from the fact that the degree of any node in a subgraph is bounded above by its degree in the original graph.  The fourth line follows from summing over nodes instead of edges, and then reasoning that in the worst case U is a clique and so $\degree_U(u) = t-1$ for all u $u$.

\iffalse
Recall that we prove in Section~\ref{subsec:lowerbound} that when
\[
w_e=\min\lp \frac{1}{\degree\lp u\rp+1 },\frac{1}{\degree\lp v\rp+1 }\rp
~~\mbox{ and }~~\W=\sum_{e\in E} w_e \ .\]
then $\W$ is a lower bound on the cardinality of the maximum matching M(G).
\fi


In some cases $W$ is too conservative; it assigns little weight to edges which it could safely have assigned much more.  For example, if $e$ is isolated (meaning its endpoints have degree 1), $W(e) = 1/2$.  However, it is always safe to assign weight 1 to $e$, since $M(G-e) = M(G)-1.$  If we modified our rule for $W$ so that for any edge $e= (u,v)$ s.t. $\deg(u)=\deg(v) = 1$ we assigned weight $ \min(1/\deg(u),1/\deg(v))$ instead of $ \min(1/\deg(u)+1,1/\deg(v)+1)$, we would always assign weight 1 to isolated edges.

In fact, a more general rule is true.  For any edge $e= (u,v)$, if either $\deg(u) or \deg(v) = 1$ then we may assign it weight $\min(1/\deg(u),1/\deg(v))$.

\iffalse
At this point it is natural to ask whether we can remove the $+1$s from the denominator for other edges.  We cannot of course do so for $\deg(u) = \deg(v) = k >1, k \in 2\mathbb{Z}$ since for a clique on k+1 nodes this rule would result in total weight $(k+1)/2$.\\

However, we can remove the $+1$ under certain circumstances. We show that if one of the endpoints has degree one then this is the case.
\fi

%prove several rules of this form.  For each rule we specify values of $\deg(u)$ and $\deg(v)$ for which the %$+1$ can be removed from the denominator of the edge weight.
%First we restate the claim that isolated edges may safely be assigned weight 1.
%\begin{lemma}
%Define $\W(G)$ as follows:\\
%For each $(u,v) \in G$,\\
%if $\deg(u) = \deg(v) = 1$, $\W(u,v) =  \min(\frac{1}{\deg(u)},\frac{1}{\deg(v)})$.\\
%else $\W(u,v) =  \min(\frac{1}{\deg(u)+1},\frac{1}{\deg(v)+1})$.
%Then $\W(G) \leq M(G)$.
%\end{lemma}
%\begin{proof}
%Already proven above.
%\end{proof}
%Next we will amend $\W$ to include the following rule: when one endpoint has degree 1 and the other has %degree k, assign weight $\frac{1}{k}$.

Define $\W$ as follows:
\[\W=\sum_{(u,v)\in E} \W(u,v) \]
where
\[
\W(u,v)=\frac{1}{\max(\degree\lp u\rp, \degree\lp v\rp )+I[\min(\degree(u),\degree(v))> 1]}\]

We now show that $\W \leq M$. We have proven that $W(U)\leq (t-1)/2$ on any odd-size subgraph $U$, $|U| = t$.  Define subsets $U_1$ and $U_2$ of $U$ such that $U_1 \cup U_2 = U$.  $U_1$ is the set of all nodes in $U$ of degree 1 and all nodes in $U$ adjacent to a node of degree 1, and $U_2$ is the set of all other nodes in $U$.  Let $G(U_2)$ denote the subgraph of $U$ induced by $U_2$, and let $|U_1| = x$ where x is even.  Then $W(G(U_2)) = \W(G(U_2)) \leq (t-x-1)/2$.  So to complete our proof we must show that all remaining edges (call them $E'$) have total weight $\leq x/2$.

Assume WLOG that there are no isolated edges in $G$ (if there are, we can group them with $G(U_2)$ and retain the $(t-x-1)/2$ bound).

\begin{align*}
\W(E') & \leq \frac{1}{2} \sum_{u,v \in E'} \lp \frac{1}{\deg(u)}+\frac{1}{\deg(v)}\rp\\
& = \mathlarger{\sum_k} \Big(\sum_{v \in U_1, \deg(v) = k} \frac{\delta}{k} + \frac{k-\delta}{2(k+1)}\Big)
\end{align*}
where $\delta$ denotes the number of degree 1 nodes adjacent to $v$.\\
Let $f(\delta) = \delta/k + (k-\delta)/(2(k+1))$.  We are interested in finding the maximum value $f(\delta)/(\delta+1)$ can take on; if it can never take a value greater than $1/2$ then $\W(E')$ cannot be greater than $x/2$.
$$\frac{\partial\frac{f(\delta)}{\delta+1}}{\partial \delta} = \frac{2-k}{2k(\delta+1)^2}$$
which is always negative for $k \geq 2$.  So, $f(\delta)/(\delta+1)$ is maximized at $\delta = 0$, so  $f(\delta)/(\delta+1) \leq k/(2(k+1)) < 1/2.$

There is one final detail we have not considered: $x$ might be odd.  In this case, $|U_2|$ is even and we can't appeal to Theorem 3 to say that $\W(G(U_2)) \leq (t-x-1)/2$.  However, we can simply remove one node $\node'$ from $U_2$; the resulting odd-size subgraph has weight at most $(t-x-2)/2$.  Since $\W$ is a valid fractional matching, the weight assigned to all edges adjacent to $\node'$ cannot exceed 1, so we can say that $\W(G(U_2)) \leq (t-x)/2$.  Now we must show that $\W(E') \leq (x-1)/2$.

We have shown that the edge weight per node in $U_1$ cannot exceed $1/2$.  $\W(E')$ is minimized when there is exactly 1 node of degree 1, with a degree k neighbor. In this case, the only edge in $E'$ will be assigned weight $k/(2(k+1))< 1/2$.  $\W(E') \leq 1/2 + (x-2)/2 = (x-1)/2$ and the theorem is proven.
\end{proof}

\iffalse
Finally, we further amend $\W$ to include the following rule:  when one endpoint has degree 2 and the other degree 3, assign weight $\frac{1}{3}$.

\begin{lemma}
Define $\W(G)$ as follows:\\
For each $(u,v) \in G$,\\
if $\deg(u) = 1$, or if $\deg(u) = 2$ and $\deg(v) = 3$, $\W(u,v) =  \min(\frac{1}{\deg(u)},\frac{1}{\deg(v)})$.\\
else $\W(u,v) =  \min(\frac{1}{\deg(u)+1},\frac{1}{\deg(v)+1})$.

Then $\W(G) \leq M(G)$.
\end{lemma}
\begin{proof}
Omitted for space.
\end{proof}
\fi


%\end{document}

 % Mesh





%% End of body
%%%%%%%%%%%%%%%%%%%%%%%%%%%%%%%%%%%%%%%%%%%%%%%%%%%%%%%%%%%%%%%%%%%%%%%%%%%%%%%

\appendix
\chapter{Scan-First Trees}
A scan first search tree (SFST) of a graph \cite{CheriyanKT93} is defined as follows: The tree is initially empty, all vertices except the root (chosen arbitrarily) are \emph{unmarked}, and 
 all vertices are \emph{unscanned}. At each step we \emph{scan} an marked but unscanned vertex. For each vertex $x$ that is being scanned, all edges from $x$ to unmarked neighbors of $x$ are added to the tree and the unmarked neighbors are marked. This continues until no marked but unscanned vertices remain.

\begin{theorem}
Any data stream algorithm that constructs a SFST with probability at least $3/4$ requires $\Omega(n^2)$ space.
\end{theorem}
\begin{proof}
The proof is by a reduction from the communication problem of indexing \cite{Ablayev96}. Suppose Alice has a binary string $x\in \{0,1\}^{n^2}$ indexed by $[n]\times [n]$ and Bob wants to compute $x_{i,j}$ for some index $(i,j)\in [n]\times [n]$ that is unknown to Alice. This requires $\Omega(n^2)$ bits to be communicated from Alice to Bob if Bob is to learn $x_{i,j}$ with probability at least $3/4$.
Suppose we have a data stream algorithm for constructing an SFST. Alice creates a graph on nodes $T\cup U\cup V \cup W$ where $T=\{t_1, \ldots, t_n\}, U=\{u_1, \ldots, u_n\}, V=\{v_1, \ldots, v_n\}$, and $W=\{w_1, \ldots, w_n\}$. She adds edges $\{t_k,u_\ell\}$ and $\{v_\ell,t_k\}$ for each $\ell,k$ such that $x_{\ell,k}=1$.  Alice runs the scan-first search algorithm and sends the contents of her memory to Bob. Bob adds the edge $\{u_i,v_i\}$. Note that any SFST includes all neighbors of $u_i$ or $v_i$. In particular, $x_{i,j}=1$ iff at least one of $\{t_j,u_i\}$ or $\{v_i,w_j\}$ is present in the SFST constructed. Hence, the algorithm must have used $\Omega(n^2)$ space.
\end{proof}


%\chapter{THE SECOND APPENDIX TITLE}
%...

%%
%% Beginning of back matter
\backmatter  %% <--- mandatory

%%
%% We don't support endnotes

%%
%% A bibliography is required.
\interlinepenalty=10000  % prevent split bibliography entries
\bibliographystyle{umassthesis}
\bibliography{dynamic}
\end{document}

%%% Local Variables: 
%%% mode: latex
%%% TeX-master: t
%%% End: 
