\section{Algorithms \& System Design}
\label{sec:algorithms}

%This section provides an overview of \Mesh{}'s algorithms.
%Section~\ref{sec:allocator} provides a detailed description of \Mesh's
%implementation, while Section~\ref{sec:theory} presents a theoretical
%analysis of its effectiveness.

\Mesh{} comprises three main algorithmic components: allocation
(\S\ref{sec:allocation-algorithm}), deallocation
(\S\ref{sec:deallocation-algorithm}), and finding spans to mesh
(\S\ref{sec:meshing-algorithm}). Unless otherwise noted and without
loss of generality, all algorithms described here are per size class
(within spans, all objects are same size).

\subsection{Allocation}
\label{sec:allocation-algorithm}
% We want out allocator to enforce random spread of objects so that we can probabilistically guarantee meshing.  Also we want it to reuse memory when available.

Allocation in \Mesh{} consists of two steps: (1) finding a span to
allocate from, and (2) randomly allocating an object from that span.
\Mesh{} always allocates from a thread-local shuffle vector -- a
randomized version of a freelist %(described in detail in\S\ref{sec:shuffle-freelists}). 
The shuffle vector contains offsets
corresponding to the slots of a single span.  We call that span the
\emph{attached} span for a given thread.

If the shuffle vector is empty, \Mesh relinquishes the current
thread's attached span (if one exists) to the \emph{global heap}
(which holds all unattached spans), and asks it to select a new
span. If there are no partially full spans, the global heap returns a
new, empty span.  Otherwise, it selects a partially full span for
reuse. To maximize utilization, the global heap groups spans into bins
organized by decreasing occupancy (e.g., 75-99\% full in one bin,
50-74\% in the next). The global heap scans for the first non-empty
bin (by decreasing occupancy), and randomly selects a span from that
bin.

Once a span has been selected, the allocator adds the offsets
corresponding to the free slots in that span to the thread-local
shuffle vector (in a random order). \Mesh{} pops the first entry off
the shuffle vector and returns it.


\subsection{Deallocation}
\label{sec:deallocation-algorithm}

Deallocation behaves differently depending on whether the free is
local (the address belongs to the current thread's attached span),
remote (the object belongs to another thread's attached span), or if
it belongs to the global heap.

For local frees, \Mesh{} adds the object's offset onto the span's
shuffle vector in a random position and returns. For remote frees,
\Mesh{} atomically resets the bit in the corresponding index in a
bitmap associated with each span. Finally, for an object belonging to
the global heap, \Mesh{} marks the object as free, updates the span's
occupancy bin; this action may additionally trigger meshing.
%, which we
%describe below.

%% Global frees similarly mark the space backing an allocation as free and open for reuse in the owning span, but additionally may trigger meshing.  Meshing happens after a global free if it has been more than a set period of time since the last meshing (settable both at program startup and later by the program through the \texttt{mallctl} API, by default a tenth of a second, and if the allocator thinks there is a reasonable chance of meshing reclaiming space.

%% TODO: free -- right now, we assume unlimited virtual addresses; max limit of ranges in kernel (discuss in implementation)


\subsection{Meshing}
\label{sec:meshing-algorithm}

When meshing, \Mesh{} randomly chooses pairs of spans and attempts to
mesh each pair. The meshing algorithm, which we call \sm
(Figure~\ref{fig:meshalg}), is designed both for practical
effectiveness and for its theoretical guarantees.  The parameter $t$,
which determines the maximum number of times each span is probed (line
\ref{li:outerloop}), enables space-time trade-offs. The parameter $t$
can be increased to improve mesh quality and therefore reduce space,
or decreased to improve runtime, at the cost of sacrificed meshing
opportunities. We empirically found that $t=64$ balances runtime and
meshing effectiveness, and use this value in our implementation.

\sm proceeds by iterating through $S_l$ and checking whether it can
mesh each span with another span chosen from $S_r$ (line
\ref{li:condition}).  If so, it removes these spans from their
respective lists and meshes them (lines
\ref{li:remove}--\ref{li:mesh}). \sm repeats until it has checked $t *
|S_l|$ pairs of spans.% \S\ref{sec:meshing-implementation} describes the implementation of \sm in detail.

\begin{comment}
  At this point, each span has been checked exactly once.  All spans not
removed from the lists are still candidates for meshing.  \sm loops
again through the lists, in the $j$th step comparing the $j$th span in
$S_l$ with the $j+1$th span in $S_r$.  Again, whenever it compares a
pair of spans that can mesh, it removes them from their respective
lists and meshes them.

\sm repeats this looping process $t$ times.  On the $i$th loop through
the lists, the $j$th element in $S_l$ is compared with the $j+i-1$th
element in $S_r$.  After this outer loop (line~\ref{li:outerloop}) has
completed, any remaining span has been compared with $t$ other spans,
each time failing to mesh. \sm stops here and makes no further effort
to mesh these spans.
\end{comment}


\iffalse
\begin{algorithm}[tb]
  \LinesNumbered
  \SetKwData{Frontier}{frontier}\SetKwData{This}{this}\SetKwData{OldNode}{$n_{old}$}\SetKwData{NewNode}{$n_{new}$}
  \SetKw{Continue}{continue} \SetKw{And}{and}
  \SetKwData{Null}{null}
  \SetKwData{Path}{path}
  \SetKwData{OldEdges}{$E_{old}$}
  \SetKwData{NewEdges}{$E_{new}$}
  \SetKwData{OldEdge}{$e_{old}$}
  \SetKwData{NewEdge}{$e_{new}$}
  \SetKwData{Length}{Length}
  \SetKwData{True}{true} \SetKwData{False}{false}
  \SetKwFunction{MakeTuple}{MakeTuple}
  \SetKwFunction{GetRoot}{GetRoot}\SetKwFunction{Enqueue}{enqueue}\SetKwFunction{Length}{length}\SetKwFunction{Dequeue}{dequeue}
  \SetKwFunction{HasVisited}{HasVisited}
  \SetKwFunction{MarkAsVisited}{MarkAsVisited}
  \SetKwFunction{MarkAsGrowing}{MarkAsGrowing}
  \SetKwFunction{MarkNotGrowing}{MarkNotGrowing}
  \SetKwFunction{RecordGrowingPath}{RecordGrowingPath}
  \SetKwFunction{IsGrowing}{IsGrowing}
  \SetKwInOut{Input}{Input}\SetKwInOut{Output}{Output}
  \Input{$G_{final} = (N, E)$}
  \Frontier$\leftarrow$ []\;
  $n_{root} \leftarrow$ \GetRoot{$G_{final}$}\;
  \For{$e = (n_1, n2) \in E : n_1 = n_{root}$}{
    \Frontier.\Enqueue{\MakeTuple{\Null, $e$}}
  }
  \While{\Frontier.\Length $\neq 0$}{
    $T \leftarrow$ \Frontier.\Dequeue{}\;
    $(T_{prev}, e) \leftarrow T$\;
    \eIf{\HasVisited{$e$}}{
      \Continue\;
    }{
      \MarkAsVisited{$e$}\;
    }
    $(n_{from}, n_{to}) \leftarrow e$\;

    \If{\IsGrowing{$n_{to}$}}{
      \RecordGrowingPath{$T$}\;
    }
    \ForEach{$e=(n_1,n_2) \in E : n_1 = n_{to}$}{
      \Frontier.\Enqueue{\MakeTuple{$T$, $e$}}\;
    }
  }
  \caption{FindLeakPaths, which records paths through the heap to leaking nodes. RecordGrowingPath recovers the entire path to the edge $e$ by following the linked list formed by the tuple $T$.}
  \label{algo:findpaths}
\end{algorithm}
\fi

\iffalse
\begin{algorithm}[tb]
  \SetAlgorithmName{\sm}{}{}
  \SetKwInOut{Input}{Input}
  \SetKwInOut{Output}{Output}
  \Input{randomly ordered list of spans S, $|S| = n$; parameter $t$}

  M $ \leftarrow$ []\;

  $S_l,$ $S_r = S[1:\frac{n}{2}],$ $S[\frac{n}{2}+1 : n]$\;
  \ForEach{$i \in [0, t-1]$}{
    len = length$(S_l)$\;
    \ForEach{$j \in [0, \frac{n}{2}-1]$}{
      \If{$S_l(j)$ and $S_r(j+i$ mod len) mesh}{
        $S_l \leftarrow S_l \textbackslash S_l(j)$\;
        $S_r \leftarrow S_r \textbackslash S_r(j+i$ mod len)\;
        M $\leftarrow$ M $\cup$ $S_l(j)$ $\cup$ $S_r(j+i$ mod len)\;
      }
    }
  }
  \Return M\;
  \TitleOfAlgo{Procedure splits span set into halves and probes for meshable pairs between halves.}
%  \label{alg:mesher}
\end{algorithm}
\fi

\begin{figure}[!t]
\begin{codebox}
    \Procname{$\sm(S,t)$}
%    \li M $\gets$ []
    \li $n \gets$ length$(S)$
    \li $S_l,$ $S_r \gets S[1:n/2],$ $S[n/2 +1 : n]$
    \li \For $(i = 0, i<t, i++)$ \label{li:outerloop}
        \li \Do
            $\mbox{len} = |S_l|$\;
            \li \For $(j = 0, j < \mbox{len}, j++)$ \label{li:innerloop}
                \li \Do
                    \If \proc{Meshable} $(S_l(j)$, $S_r(j+i$ \% $\mbox{len}$)) \Then \label{li:condition}
                        \li $S_l \leftarrow S_l \setminus S_l(j)$ \label{li:remove}
                        \li $S_r \leftarrow S_r \setminus S_r(j+i$ \% $\mbox{len}$)
%                        \li M $\leftarrow$ M $\cup$ $S_l(j)$ $\cup$ $S_r(j+i$ mod len)
                        \li \proc{mesh}($ S_l(j)$, $S_r(j+i$ \% $\mbox{len}$)) \label{li:mesh}
                    \End
                \End
        \End
%    \li \Return M
\end{codebox}
\caption{\textbf{Meshing random pairs of spans.} \sm splits the randomly ordered span list $S$ into halves, then probes pairs between halves for meshes.  Each span is probed up to $t$ times.}
\label{fig:meshalg}
\end{figure}
